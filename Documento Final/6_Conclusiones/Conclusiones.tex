%  ╔═╗┌─┐┌┐┌┌─┐┬  ┬ ┬┌─┐┬┌─┐┌┐┌┌─┐┌─┐
%  ║  │ │││││  │  │ │└─┐││ ││││├┤ └─┐
%  ╚═╝└─┘┘└┘└─┘┴─┘└─┘└─┘┴└─┘┘└┘└─┘└─┘

\chapter{Conclusiones y recomendaciones}
\section{Conclusiones}
En este trabajo se desarrollaron dos aplicaciones de \emph{software}: Una para realizar la calibración de calibradores
acústicos, y otra para sonómetros, de conformidad con los lineamientos del Anexo B de la \mbox{IEC 60942}~\citeyearpar{IEC_TC29_2017}
y de la \mbox{IEC 61672--3}~\citeyearpar{IEC_TC29_2013_3}, respectivamente.
Estas aplicaciones integran el control de los instrumentos de medición (patrones y auxiliares) que conforman el sistema de
calibración junto con las secuencias predeterminadas.
Particularmente, para sonómetros, también se integra el procesamiento de imágenes para el reconocimiento de los valores
indicados en la pantalla del sonómetro y el modelamiento de la variabilidad de un resultado usando cadenas de Markov.

Dados los resultados obtenidos, las aplicaciones desarrolladas mejoran significativamente los procesos
de calibración de los laboratorios, reduciendo el tiempo total de calibración en aproximadamente un $30\%$,
bajan la incertidumbre de medición (al incrementar los grados efectivos de libertad por el mayor número de muestras que
pueden obtenerse respecto a un proceso manual) y aumentan la robustez del procedimiento (asegurando los resultados contra
el error humano).

Para calibradores acústicos, la arquitectura de \emph{software} basada en el modelo de GRAFCET es una metodología
que permitió desarrollar una aplicación versátil e intuitiva, ofreciendo al usuario un mecanismo sencillo para identificar
y ordenar los pasos del proceso de calibración en etapas enumeradas.
A esto se suma que, tomar el GRAFCET como un tipo de paradigma de programación, facilita la implementación en Python
empleando multi-hilos.

El control de los instrumentos del estante de calibración mediante comandos SCPI es una de las herramientas fundamentales
en la aplicación desarrollada.
Utilizar estos comandos en las secuencias programadas hace innecesario que el usuario manipule directamente los controles
físicos de los instrumentos.
Esto contribuye al aseguramiento contra el error humano y aumenta la precisión de los parámetros de señal, ya que se pueden
ajustar y obtener todos los dígitos disponibles en el instrumento (lo que no es posible con la interfaz física).

El método de reconocimiento de imágenes implementado para la calibración de sonómetros siguió la estructura clásica de
adquisición (captura de fotogramas), preproceso (filtrado, \emph{padding} y escalamiento), segmentación
(binarización Otsu), extracción de características (descriptor SIFT simplificado), y reconocimiento (clasificador
bayesiano normal ingenuo).
Este sencillo método es suficiente para el reconocimiento en tiempo real con una base de datos
de entrenamiento relativamente pequeña y con el que se consiguen resultados confiables a un bajo costo computacional.

Finalmente, el modelamiento de la variabilidad de un resultado empleando cadenas de Markov, considerándola como un proceso
estocástico, permite estimar en el largo plazo cuál es el valor de medición tomando en cuenta la \emph{dinámica} del
cambio de un valor a otro (representada en las probabilidades de transición) y no solo como la media de todas las
muestras adquiridas.

\section{Recomendaciones y trabajo futuro}
Las aplicaciones fueron programadas con un manejo de excepciones básicas, sin embargo, conviene garantizar que estas
sean ejecutadas en las condiciones normales.
Como que las señales eléctricas enviadas o medidas se encuentren dentro del rango de los instrumentos utilizados,
lo cual se puede predecir con la sensibilidad del micrófono patrón y del micrófono del sonómetro, y con los límites
del rango lineal de medición del sonómetro.
Al usar la aplicación para la calibración de sonómetros, se recomienda adecuar un ambiente controlado para que la
iluminación, distancia focal, o ruido no afecte el reconocimiento de imágenes.

Tal como están desarrolladas las aplicaciones hasta ahora, cuentan con el funcionamiento básico necesario para llevar a
cabo las calibraciones de principio a fin.
Sin embargo, aún resta desarrollar utilidades que mejorarían las aplicaciones y las harían más adecuadas para el uso
cotidiano en los laboratorios.
Utilidades como guardar y cargar sesiones de calibración, retroceder, avanzar o navegar entre etapas, emitir certificado
de calibración y bloquear resultados.
El desarrollo de estas utilidades puede hacerse sobre lo que ya está, puesto que en el código se dejaron las bases para
esas funciones.
Por ejemplo, en la interfaz gráfica están los botones correspondientes, y la configuración básica de estos se ejecuta al
lanzar la aplicación.

Por ahora, las señales eléctricas son ajustadas y medidas sin tener en cuenta las correcciones proporcionadas en los
certificados de calibración de los patrones, por lo que los resultados aún no son trazables en el sentido estricto.
El trabajo posterior es entonces, aplicar las correcciones usando técnicas de interpolación o métodos numéricos a partir
de los resultados de calibración de los patrones.

Igualmente, la estimación de incertidumbre de medición en este trabajo llegó hasta la incertidumbre tipo A, pero para
reportar un resultado de medición completo, el trabajo debe complementarse estimando las demás componentes de incertidumbre
tipo B y calculando la incertidumbre expandida de medición.
Además, en este trabajo la incertidumbre tipo A se estimó a partir de la desviación experimental de la media, no del
valor esperado del proceso estocástico.
El trabajo futuro es tener en cuenta la distribución de probabilidad estacionaria del proceso estocástico en la estimación
de la incertidumbre tipo A y posteriormente en la determinación de los intervalos de confianza de la incertidumbre
expandida de medición, de tal modo que se relacione el modelamiento en cadenas de Markov con la incertidumbre de medición
asociada al valor de medición.