%\newpage
%\setcounter{page}{1}
% ---- PRIMERA PÁGINA DE PORTADA ----
\begin{center}
    \capstartfalse  % Omite el caption de la primera imagen
    \begin{figure}
        \centering
        \epsfig{file=0_Portada/EscudoUN,scale=1}
    \end{figure}
    \thispagestyle{empty}
    % \vspace*{0.2cm}
    \setstretch{1.5}\huge\textbf{Desarrollo de un procedimiento de calibración de sonómetros
    y calibradores acústicos de conformidad con las normas \mbox{IEC 61672--3} e \mbox{IEC 60942}}\\[4.0cm]
    \setstretch{1.1}
    \Large\textbf{Juan Felipe Maldonado Pedraza}\\[4.0cm]
    \small Universidad Nacional de Colombia\\
    Facultad de Ingeniería, Departamento de Ingeniería Eléctrica y Electrónica\\
    Bogotá, Colombia\\
    2021
\end{center}

\newpage{\pagestyle{empty}\cleardoublepage}
\newpage

% ---- SEGUNDA PÁGINA DE PORTADA ----
\begin{center}
    \thispagestyle{empty}
    \setstretch{1.3}\huge\textbf{Desarrollo de un procedimiento de calibración de sonómetros
    y calibradores acústicos de conformidad con las normas \mbox{IEC 61672--3} e \mbox{IEC 60942}}\\[3.0cm]
    \setstretch{1.1}
    \Large\textbf{Juan Felipe Maldonado Pedraza}\\[3.0cm]
    \small Trabajo de grado presentado como requisito parcial para optar al título de:\\
    \textbf{Master en Automatización Industrial}\\[2.0cm]
    Director:\\
    Ph. D.\ Leonardo Enrique Bermeo Clavijo\\[2.0cm]
    Línea de investigación:\\
    Automatización en procesos de metrología\\
    % Grupo de Investigación:\\
    % Nombrar el grupo en caso que sea posible\\[2.5cm]
    Universidad Nacional de Colombia\\
    Facultad de Ingeniería\\
    Bogotá, Colombia\\
    2022
\end{center}

\newpage{\pagestyle{empty}\cleardoublepage}

% ---- DEDICATORIA O LEMA ----
\newpage
\thispagestyle{empty}\normalsize \vspace*{3cm}

\begin{flushright}
    \begin{minipage}{8cm}
        \setlength{\parskip}{2em}  % Espacio entre párrafos
        \noindent
        \small
        \begin{center}
            ``Con Dios está la sabiduría y la fortaleza;\\suyo es el consejo y la inteligencia.''
        \end{center}
        \begin{flushright}
            Job 12:13 (RVG)
        \end{flushright}
    \end{minipage}
\end{flushright}

\newpage{\pagestyle{empty}\cleardoublepage}

% ---- AGRADECIMIENTOS ----
\newpage
\thispagestyle{empty}\normalsize \vspace*{3cm}

%\textbf{\LARGE Agradecimientos} \vspace{2em}
\chapter*{Agradecimientos}
\addcontentsline{toc}{chapter}{Agradecimientos}
Al Dios de mi supremo gozo, quien me ciñe de entendimiento y vigor para culminar trabajos como este.
A Él sea dado el honor y las gracias porque, a pesar de mí, su favor no me ha faltado jamás.

A mis hermanos en la fe y a mi familia, quienes toleraron mi ausencia y no obstante conté con su apoyo mientras
dedicaba mi tiempo a terminar este trabajo.

A mi director de trabajo de grado, por su importante orientación en el desarrollo del proyecto.

A todos ellos, gracias.

\newpage{\pagestyle{empty}\cleardoublepage}

% ---- RESUMEN ----
\newpage
%\textbf{\LARGE Resumen} \vspace{2em}
\chapter*{Resumen}
\addcontentsline{toc}{chapter}{Resumen}
En este documento se describe el desarrollo de un sistema de calibración de instrumentos acústicos de conformidad con
las
normas internacionales \mbox{IEC 60942} e \mbox{IEC 61672--3}.
Se inicia con una breve revisión de los sistemas desarrollados hasta ahora, analizando sus ventajas y oportunidades de
mejora.
En seguida se hace un estudio del marco normativo y de la naturaleza de los equipos bajo prueba, y se determina la
instrumentación necesaria para el sistema de calibración.
El sistema de calibración de sonómetros es controlado por una aplicación que emplea reconocimiento de imágenes, por lo
que después del marco normativo se explica el método diseñado para el reconocimiento de caracteres numéricos, y se
incluyen
resultados del procesamiento y desempeño del clasificador.
Luego, se introduce una sección en la que se describen los detalles de implementación de las aplicaciones de
\emph{software}
codificadas en Python;
particularmente, se presenta el modelo GRAFCET que fue la base del desarrollo de la aplicación para calibradores
acústicos.
A continuación, se explica el diseño e implementación del método propuesto para modelar la variabilidad de un valor
de medición
como un proceso estocástico usando cadenas de Markov y se muestra un ejemplo de una matriz de transición obtenida y el
cálculo del valor esperado.
Se concluye el documento subrayando los logros alcanzados, las recomendaciones de uso del sistema de calibración
y sugerencias de trabajo futuro.

\textbf{\small Palabras clave: Calibración, metrología, sonómetros, calibradores acústicos, automatización,
    visión de máquina, cadenas de Markov} \vspace{2em}

\vfill
\pagebreak

%\textbf{\LARGE \textsf{Abstract}} \vspace{2em}
\chapter*{Abstract}
This document describes the development of a calibration system for acoustic instruments in accordance with the
international standards \mbox{IEC 60942} and \mbox{IEC 61672--3}.
It begins with a brief review of the systems developed so far, analyzing their advantages and opportunities for
improvement.
Next, a study of the regulatory framework and the nature of the equipment under test is made, and the necessary
instrumentation for the calibration system is determined.
The sound level meter calibration system is controlled by an application that uses image recognition, so after the
normative framework, the method designed for the recognition of numerical characters is explained, and processing
results
and performance of the classifier are included.
Then, a section is introduced describing the implementation details of software applications coded in Python.
In particular, the GRAFCET model is presented, which was the basis for the development of the application for acoustic
calibrators.
Next, the design and implementation of the proposed method to model the variability of a measurement value as a
stochastic
process using Markov chains is explained, and an example of a transition matrix obtained and the calculation of the
expected value is shown.
The document is concluded highlighting the achievements, the recommendations for the use of the calibration system, and
suggestions for future work.

\textbf{\small Keywords: Calibration, metrology, sound level meters, acoustic calibrators, automation,
    computer vision, Markov chains}
\vfill