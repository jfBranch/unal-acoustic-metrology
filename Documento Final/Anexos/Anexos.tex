%  ╔═╗┌┐┌┌─┐─┐ ┬┌─┐┌─┐
%  ╠═╣│││├┤ ┌┴┬┘│ │└─┐
%  ╩ ╩┘└┘└─┘┴ └─└─┘└─┘

\begin{appendix}

\chapter{Anexo: Códigos de Python}

\begin{code}
	\caption{Código para presentar resultados del procesamiento de una muestra del número $5$ del conjunto de imágenes entrenamiento.}
	\label{code:test_samples_code}
	\centering
	\inputminted{python}{3_Reconocimiento/Codes/image_processing_test.py}
\end{code}

\begin{code}
	\caption{Código para realizar la extracción de características de los vectores de entrenamiento y para entrenar el clasificador.}
	\label{code:features_extracting}
	\centering
	\inputminted{python}{3_Reconocimiento/Codes/features_extracting.py}
\end{code}

\begin{code}
\caption{Código para realizar la extracción de características de los vectores de prueba y para realizar la clasificación.}
\label{code:sample_classifying}
\centering
\begin{minted}{python}
X = np.array(X)  # Convert test sample features vector to numpy array
# -------- DIMENSIONALITY REDUCTION BY KERNEL PCA --------- #
X = kpcaModel.transform(X)
# ------- CLASSIFY ---------#
y_hat = gnbClassifier.predict(X)

\end{minted}
\end{code}

\begin{code}
\caption{Código para generar la matriz de confusión del clasificador bayesiano normal.}
\label{code:confusion_matrix}
\centering
\begin{minted}{python}
from sklearn.metrics import confusion_matrix, ConfusionMatrixDisplay

# -------------- CLASSIFICATION ACCURACY --------------
cm = confusion_matrix(y_test, y_hat, labels=gnbClassifier.classes_)
disp = ConfusionMatrixDisplay(confusion_matrix=cm, display_labels=gnbClassifier.classes_)
disp.plot()
plt.show()
\end{minted}
\end{code}

\end{appendix}