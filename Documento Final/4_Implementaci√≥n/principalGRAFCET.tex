%! Author = Usuario
%! Date = 18/07/2022

% Preamble
\documentclass[12pt,spanish,fleqn,openany,letterpaper,pagesize,hidelinks]{scrbook}

% Packages
\usepackage[T1]{fontenc}
\usepackage[spanish]{babel}  % Escribir con acentos sin necesidad de comandos \'{}
\usepackage{setspace}  % Paquete para el ajuste de espacio entre líneas
\usepackage{fancyhdr}  % Paquete para el ajuste de formato de página
\usepackage{epsfig}
\usepackage{epic}
\usepackage{eepic}
\usepackage{amsmath}  % Paquete para diversas herramientas matemáticas
\usepackage{amssymb}  % Paquete para rotar tablas, figuras, etc.
\usepackage{mathtools}  % Paquete para el símbolo :=
\usepackage{kbordermatrix}  % Paquete para matrices con índices de filas y columnas
\usepackage{upgreek}  % Paquete para letras griegas en roman upright
\usepackage{threeparttable}
\usepackage{amscd}
\usepackage{here}
\usepackage{nicefrac}  % Paquete para fracciones en diagonal
\usepackage[binary-units=true,angle-mode=decimal]{siunitx}  % Paquete para unidades en SI
\sisetup{output-decimal-marker={,}}  % Establece la coma para decimales
\usepackage{tikz}
\usepackage{flowchart}  % Paquete con formas predefinidas de diagramas de flujo
\usetikzlibrary{arrows, arrows.meta}
\usetikzlibrary{automata}  % Paquete para dibujar autómatas
\usetikzlibrary{positioning}  % Paquete con herramientas para facilitar la ubicación de nodos
\usetikzlibrary{math}  % Paquete para definir variables
\usepackage{grafcet}  % Paquete para dibujar gráficos de etapas y transiciones
\usepackage[simplified]{pgf-umlcd}  % Paquete para dibujar diagramas UML
\usepackage[algochapter,linesnumbered,vlined,ruled,commentsnumbered]{algorithm2e}  % Paquete para escribir pseudocódigo
\usepackage{graphicx}  % Paquete para poder usar gráficos
\usepackage{tabularx}  % Paquete para poder usar tablas
\usepackage{longtable}
\usepackage{xcolor}  % Paquete para definir colores
\usepackage{caption}
\usepackage{subcaption}
\usepackage{listings}  % Paquete para entornos tipo listings
\usepackage[chapter]{minted}  % Paquete para códigos con formato
\usepackage{lscape}
\usepackage{lipsum}  % Paquete para simular párrafos de textos
% Paquete para habilitar hipervínculos habilitando el cambio de línea
\PassOptionsToPackage{hyphens}{url}\usepackage[colorlinks=true,linkcolor=blue,urlcolor=gray,citecolor=blue]{hyperref}
\usepackage[all]{hypcap}  % Paquete para ajustar las referencias a floats por encima de estos
\usepackage{natbib}  % Paquete mejorado para bibliografía
\usepackage{mathspec} % Importa el paquete para el uso de fuentes externas
\usepackage{fancyvrb}
\usepackage{rotating}

% Establece la fuente Ancizar Serif como la fuente principal
\setmainfont[Ligatures = TeX, UprightFont = {* Regular}, BoldFont = {* Bold},
    ItalicFont = {* Italic}, BoldItalicFont = {* Bold Italic}]{Ancizar Serif}
% Establece la fuente Ancizar Sans como la fuente Sans
\setsansfont[Ligatures = TeX, UprightFont = {* Regular}, BoldFont = {* Bold},
    ItalicFont = {* Italic}, BoldItalicFont = {* Bold Italic}]{Ancizar Sans}
% Establece l afuente JetBrains Mono como la fuente monoespaciada
\setmonofont[Ligatures = TeX, UprightFont = {* Regular}, BoldFont = {* Bold},
    ItalicFont = {* Italic}, BoldItalicFont = {* Bold Italic}]{JetBrains Mono}

% ---------- CONFIGURACIÓN DE PÁGINA ----------
\pagestyle{fancyplain}
\addtolength{\headwidth}{\marginparwidth}
\textheight22.5cm \topmargin0cm \textwidth16.5cm
\oddsidemargin0.5cm \evensidemargin-0.5cm

\fancyfoot{}

\unitlength1mm  % Define la unidad LE para Figuras
\mathindent1cm  % Define la distancia de las formulas al texto,  fleqn las descentra
\marginparwidth0cm
\parindent0cm  % Define la distancia de la primera linea de un parrafo a la margen

% Creación de formas para dibujo
\tikzstyle{print}=[trapezium, draw, text centered, trapezium left angle=60, trapezium right angle=120, minimum
height=2em]  % Trapezoide para el diagrama de flujo
\tikzstyle{conditional}=[diamond, draw, text centered, aspect=3]  % Rombo para el bloque de decisión.
\tikzstyle{arrow} = [thick,->,>=stealth]

\newcommand{\vect}[1]{\mathrm{\mathbf{#1}}}  % Abreviación de comando para notación de vectores y matrices
\newcommand{\defeq}{\vcentcolon=}  % Declaración de comando para el símbolo :=

% Document
\begin{document}
\begin{figure}[!hp]
    \caption{GRAFCET de la rutina principal de la calibración periódica de calibradores acústicos.}
    \label{fig:GRAFCET_principal}
    \begin{subfigure}[t]{0.48\textwidth}
        \centering
        \begin{tikzpicture}[font=\scriptsize]
            \EtapeInitTransition[0,0]{0}{}{\tiny \texttt{Start}}
            \MacroEtape[VT0]{M1}
            \node [right=0.7 of XM1, align=center]{\textit{``Medición preeliminar de} \\ \textit{sensibilidad del
            micrófono''}};
            \TransitionRecept[VXM1]{1}{\tiny \texttt{XM1}}
            \MacroEtape[VT1]{M2}
            \node [right=0.7 of XM2, align=center]{\textit{``Medición de ruido} \\ \textit{de fondo a 0°''}};
            \TransitionRecept[VXM2]{2}{\tiny \texttt{XM2}}
            \MacroEtape[VT2]{M3}
            \node[right=0.7 of XM3, align=center]{\textit{``Medición de} \\ \textit{magnitudes a 0°''}};
            \TransitionRecept[VXM3]{4}{\tiny \texttt{XM3}}
            \EtapeTransition{4}{\tiny $\texttt{RefCalPower} \defeq 0$}
            {\tiny $\overline{\texttt{RefCalPower}} \cdot \overline{\texttt{CoupledRefCal}} \cdot \texttt{Ok}$}
            \ActionActiv{X4}
            \Action{X4}{\tiny $\texttt{CoupledRefCal} \defeq 0$}
            \ActionEvenement{X4}{\tiny $\downarrow\texttt{RefCalPower}$}
            \node(linkXM5)[below=0.1 of T4]{M5};
            \draw[-Straight Barb](T4) -- (linkXM5);
            \MacroEtape[9,0]{M5}
            \node [right=0.7 of XM5, align=center]{``\textit{Medición de ruido} \\ \textit{de fondo a 120°''}};
            \TransitionRecept[VXM5]{5}{\tiny \texttt{XM5}}
            \MacroEtape[VT5]{M6}
            \node[right=0.7 of XM6, align=center]{\textit{``Medición de}\\\textit{magnitudes a 120°''}};
            \TransitionRecept[VXM6]{6}{\tiny \texttt{XM6}}
            \EtapeTransition{7}{\tiny \texttt{RefCalPower} $\defeq 0$}
            {\tiny $\overline{\texttt{RefCalPower}} \cdot \overline{\texttt{CoupledRefCal}} \cdot \texttt{Ok}$}
            \ActionActiv{X7}
            \Action{X7}{\tiny $\texttt{CoupledRefCal} \defeq 0$}
            \ActionEvenement{X7}{\tiny $\downarrow\texttt{RefCalPower}$}
            \MacroEtape[VT7]{M8}
            \node[right=0.7 of XM8, align=center]{\textit{``Medición de ruido}\\\textit{de fondo a 240°''}};
            \TransitionRecept[VXM8]{8}{\tiny \texttt{XM8}}
            \MacroEtape[VT8]{M9}
            \node[right=0.7 of XM9, align=center]{\textit{``Medición de}\\\textit{magnitudes a 240°°''}};
            \TransitionRecept[VXM9]{9}{\tiny \texttt{XM9}}
            \node(linkX10)[below=0.1 of T9]{10};
            \draw[-Straight Barb](T9) -- (linkX10);
        \end{tikzpicture}
    \end{subfigure}
    \\[-0.7cm]
    \begin{subfigure}[t]{0.48\textwidth}
        \centering
        \begin{tikzpicture}[font=\scriptsize]
            \EtapeTransition[0,0]{10}{\tiny $\texttt{RefCalPower} \defeq 0$}
            {\tiny $\overline{\texttt{RefCalPower}} \cdot \overline{\texttt{CoupledRefCal}} \cdot \texttt{Ok}$}
            \ActionActiv{X10}
            \Action{X10}{\tiny $\texttt{CoupledRefCal} \defeq 0$}
            \ActionEvenement{X10}{\tiny $\downarrow\texttt{RefCalPower}$}
            \MacroEtape[VT10]{M11}
            \node[right=0.7 of XM11, align=center]{\textit{``Medición de ruido}\\\textit{de fondo a 0°''}};
            \TransitionRecept[VXM11]{11}{\tiny \texttt{XM11}}
            \MacroEtape[VT11]{M12}
            \node[right=0.7 of XM12, align=center]{\textit{``Medición de}\\\textit{magnitudes a 0°''}};
            \TransitionRecept[VXM12]{11}{\tiny \texttt{XM12}}
            \EtapeTransition{13}{\tiny \texttt{CusCalPower} $\defeq 0$}
            {\tiny $\overline{\texttt{CusCalPower}} \cdot \overline{\texttt{CoupledCusCal}} \cdot \texttt{Ok}$}
            \ActionActiv{X13}
            \Action{X13}{\tiny $\texttt{CoupledCusCal} \defeq 0$}
            \ActionEvenement{X13}{\tiny $\downarrow\texttt{CusCalPower}$}
            \MacroEtape[VT13]{M14}
            \node[right=0.7 of XM14, align=center]{\textit{``Medición de ruido}\\\textit{de fondo a 120°''}};
            \TransitionRecept[VXM14]{14}{\tiny \texttt{XM14}}
            \node(linkXM15)[below=0.1 of T14]{M15};
            \draw[-Straight Barb](T14) -- (linkXM15);
            \MacroEtape[9,0]{M15}
            \node[right=0.7 of XM15, align=center]{\textit{``Medición de}\\\textit{de magnitudes a 120°''}};
            \TransitionRecept[VXM15]{15}{\tiny \texttt{XM15}}
            \EtapeTransition{16}{\tiny $\texttt{CusCalPower} \defeq 0$}
            {\tiny $\overline{\texttt{CusCalPower}} \cdot \overline{\texttt{CoupledCusCal}} \cdot \texttt{Ok}$}
            \ActionActiv{X16}
            \Action{X16}{\tiny $\texttt{CoupledCusCal} \defeq 0$}
            \ActionEvenement{X16}{\tiny $\downarrow\texttt{CusCalPower}$}
            \MacroEtape[VT16]{M17}
            \node[right=0.7 of XM17, align=center]{\textit{``Medición de ruido}\\\textit{de fondo a 240°''}};
            \TransitionRecept[VXM17]{17}{\tiny \texttt{XM17}}
            \MacroEtape[VT17]{M18}
            \node[right=0.7 of XM18, align=center]{\textit{``Medición de}\\\textit{de magnitudes a 240°''}};
            \TransitionRecept[VXM18]{18}{\tiny \texttt{XM18}}
            \Etape{19}
            \ActionX{X19}{\tiny $\texttt{CusCalPower} \defeq 0$}{}
            \ActionActiv{X19}
            \Action{X19}{\tiny $\texttt{CoupledCusCal} \defeq 0$}
            \ActionEvenement{X19}{\tiny $\downarrow\texttt{CusCalPower}$}
        \end{tikzpicture}
    \end{subfigure}
\end{figure}
\end{document}