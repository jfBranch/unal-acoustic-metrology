%! Author = Usuario
%! Date = 18/07/2022

% Preamble
\documentclass[12pt,spanish,fleqn,openany,letterpaper,pagesize,hidelinks]{scrbook}

% Packages
\usepackage[T1]{fontenc}
\usepackage[spanish]{babel}  % Escribir con acentos sin necesidad de comandos \'{}
\usepackage{setspace}  % Paquete para el ajuste de espacio entre líneas
\usepackage{fancyhdr}  % Paquete para el ajuste de formato de página
\usepackage{epsfig}
\usepackage{epic}
\usepackage{eepic}
\usepackage{amsmath}  % Paquete para diversas herramientas matemáticas
\usepackage{amssymb}  % Paquete para rotar tablas, figuras, etc.
\usepackage{mathtools}  % Paquete para el símbolo :=
\usepackage{kbordermatrix}  % Paquete para matrices con índices de filas y columnas
\usepackage{upgreek}  % Paquete para letras griegas en roman upright
\usepackage{threeparttable}
\usepackage{amscd}
\usepackage{here}
\usepackage{nicefrac}  % Paquete para fracciones en diagonal
\usepackage[binary-units=true,angle-mode=decimal]{siunitx}  % Paquete para unidades en SI
\sisetup{output-decimal-marker={,}}  % Establece la coma para decimales
\usepackage{tikz}
\usepackage{flowchart}  % Paquete con formas predefinidas de diagramas de flujo
\usetikzlibrary{arrows, arrows.meta}
\usetikzlibrary{automata}  % Paquete para dibujar autómatas
\usetikzlibrary{positioning}  % Paquete con herramientas para facilitar la ubicación de nodos
\usetikzlibrary{math}  % Paquete para definir variables
\usepackage{grafcet}  % Paquete para dibujar gráficos de etapas y transiciones
\usepackage[simplified]{pgf-umlcd}  % Paquete para dibujar diagramas UML
\usepackage[algochapter,linesnumbered,vlined,ruled,commentsnumbered]{algorithm2e}  % Paquete para escribir pseudocódigo
\usepackage{graphicx}  % Paquete para poder usar gráficos
\usepackage{tabularx}  % Paquete para poder usar tablas
\usepackage{longtable}
\usepackage{xcolor}  % Paquete para definir colores
\usepackage{caption}
\usepackage{subcaption}
\usepackage{listings}  % Paquete para entornos tipo listings
\usepackage[chapter]{minted}  % Paquete para códigos con formato
\usepackage{lscape}
\usepackage{lipsum}  % Paquete para simular párrafos de textos
% Paquete para habilitar hipervínculos habilitando el cambio de línea
\PassOptionsToPackage{hyphens}{url}\usepackage[colorlinks=true,linkcolor=blue,urlcolor=gray,citecolor=blue]{hyperref}
\usepackage[all]{hypcap}  % Paquete para ajustar las referencias a floats por encima de estos
\usepackage{natbib}  % Paquete mejorado para bibliografía
\usepackage{mathspec} % Importa el paquete para el uso de fuentes externas
\usepackage{fancyvrb}
\usepackage{rotating}

% Establece la fuente Ancizar Serif como la fuente principal
\setmainfont[Ligatures = TeX, UprightFont = {* Regular}, BoldFont = {* Bold},
				ItalicFont = {* Italic}, BoldItalicFont = {* Bold Italic}]{Ancizar Serif}
% Establece la fuente Ancizar Sans como la fuente Sans
\setsansfont[Ligatures = TeX, UprightFont = {* Regular}, BoldFont = {* Bold},
				ItalicFont = {* Italic}, BoldItalicFont = {* Bold Italic}]{Ancizar Sans}
% Establece l afuente JetBrains Mono como la fuente monoespaciada
\setmonofont[Ligatures = TeX, UprightFont = {* Regular}, BoldFont = {* Bold},
				ItalicFont = {* Italic}, BoldItalicFont = {* Bold Italic}]{JetBrains Mono}

% ---------- CONFIGURACIÓN DE PÁGINA ----------
\pagestyle{fancyplain}
\addtolength{\headwidth}{\marginparwidth}
\textheight22.5cm \topmargin0cm \textwidth16.5cm
\oddsidemargin0.5cm \evensidemargin-0.5cm

\fancyfoot{}

\unitlength1mm  % Define la unidad LE para Figuras
\mathindent1cm  % Define la distancia de las formulas al texto,  fleqn las descentra
\marginparwidth0cm
\parindent0cm  % Define la distancia de la primera linea de un parrafo a la margen

% Creación de formas para dibujo
\tikzstyle{print}=[trapezium, draw, text centered, trapezium left angle=60, trapezium right angle=120, minimum height=2em]  % Trapezoide para el diagrama de flujo
\tikzstyle{conditional}=[diamond, draw, text centered, aspect=3]  % Rombo para el bloque de decisión.
\tikzstyle{arrow} = [thick,->,>=stealth]

% Document
\begin{document}
\begin{figure}[h]
    \caption{}
    \label{fig:acoustic_calibrator_calibration_flowchart}
    \centering
    \begin{tikzpicture}[font=\scriptsize, minimum width=2cm, minimum height=0.5cm]
        \node (start) at (-5, 0) [draw, terminal]{INICIO};
        \node (process1) [draw, process, align=center, below=0.3cm of start]{Acoplar calibrador \\ acústico patrón};
        \node (process2) [draw, process, align=center, below=0.3cm of process1]{Verificar calibrador \\ acústico
        patrón \\ $\mathrm{Acople} = 1$};
        \node (intersection1) [draw, circle, fill=black, inner sep=0pt, minimum size=2pt, below=0.3cm of process2]{};
        \node (intersection2) [draw, circle, fill=black, inner sep=0pt, minimum size=2pt, right=2.6cm of
        intersection1]{};
        \node (process3) [draw, process, align=center, below=0.3cm of intersection1]{Medir ruido \\ de fondo $(N)$};
        \node (decision1) [draw, conditional, below=0.3cm of process3]{$N \le L_{\mathrm{spec}} - \qty{30}{\dB}$};
        \node (process4) [draw, process, align=center,
            below=0.3cm of decision1]{Medir magnitudes \\ Voltaje, frecuencia y THD+N \\ $\mathrm{Acople\,} += 1$};
        \node (decision2) [draw, conditional, below=0.3cm of process4]{$\mathrm{Acople} == 4$};
        \node (process5) [draw, process, align=center, right=1.2cm of decision1]{Desacoplar calibrador \\ patrón,
            rotarlo y \\ volverlo a acoplar};
        \node (process6) [draw, process, align=center, below=0.3cm of decision2]{Calcular promedios \\ de magnitudes};
        \node (ankor1) [below right=0.3cm and 4.1cm of process6]{};
        \node (ankor2) [right=6.8cm of start]{};
        \node (process7) [draw, process, align=center,
            below=0.3cm of ankor2]{Desacoplar calibrador\\ patrón, acoplar \\ calibrador del cliente. \\
            $\mathrm{Acople} = 1$};
        \node (intersection3) [draw, circle, fill=black, inner sep=0pt, minimum size=2pt, below=0.3cm of process7]{};
        \node (intersection4) [draw, circle, fill=black, inner sep=0pt, minimum size=2pt, right=2.6cm of
        intersection3]{};
        \node (process8) [draw, process, align=center, below=0.3cm of intersection3]{Medir ruido \\ de fondo $(N)$};
        \node (decision3) [draw, conditional, below=0.3cm of process8]{$N \le L_{\mathrm{spec}} - \qty{30}{\dB}$};
        \node (process9) [draw, process, align=center,
            below=0.3cm of decision3]{Medir magnitudes: \\ Nivel de presión acústica (por \\ comparación),
            frecuencia y THD+N \\ $\mathrm{Acople\,} += 1$};
        \node (decision4) [draw, conditional, below=0.3cm of process9]{$\mathrm{Acople} == 4$};
        \node (process10) [draw, process, align=center, right=1.2cm of decision3]{Desacoplar calibrador \\ del
        cliente, rotarlo y \\ volverlo a acoplar};
        \node (process11) [draw, process, align=center, below=0.3cm of decision4]{Calcular promedios \\ de magnitudes};
        \node (finish) [draw, terminal, below=0.3cm of process11]{FIN};

        \draw [arrow] (start.south) -- (process1);
        \draw [arrow] (process1) -- (process2);
        \draw [thick] (process2) -- (intersection1);
        \draw [arrow] (intersection1) -- (process3);
        \draw [arrow] (process3) -- (decision1);
        \draw [arrow] (decision1.south) -- node[at start, right]{Sí} (process4);
        \draw [arrow] (decision1.east) -| node[at start, above]{No} (intersection2);
        \draw [arrow] (intersection2) -- (intersection1);
        \draw [arrow] (process4) -- (decision2);
        \draw [arrow] (decision2.east) -| node[at start, above]{No}  (process5.south);
        \draw [arrow] (process5) |- (intersection2);
        \draw [arrow] (decision2.south) -- node[at start, right]{Sí} (process6);
        \draw [thick] (process6) |- (ankor1.center);
        \draw [thick] (ankor1.center) |- (ankor2.center);
        \draw [arrow] (ankor2.center) -- (process7);
        \draw [thick] (process7) -- (intersection3);
        \draw [arrow] (intersection4) -- (intersection3);
        \draw [arrow] (intersection3) -- (process8);
        \draw [arrow] (process8) -- (decision3);
        \draw [arrow] (decision3.east) -| node[at start, above]{No} (intersection4);
        \draw [arrow] (decision3.south) -- node[at start, right]{Sí} (process9);
        \draw [arrow] (process9) -- (decision4);
        \draw [arrow] (decision4.east) -| node[at start, above]{No} (process10);
        \draw [arrow] (process10) |- (intersection4);
        \draw [arrow] (decision4.south) -- node[at start, right]{Sí} (process11);
        \draw [arrow] (process11) -- (finish);
    \end{tikzpicture}
\end{figure}
\end{document}