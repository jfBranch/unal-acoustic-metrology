\documentclass[12pt,spanish,fleqn,openany,letterpaper,pagesize,hidelinks]{scrbook}

% IMPORTAR PAQUETES
%\usepackage[utf8]{inputenc}
\usepackage[T1]{fontenc}
\usepackage[spanish]{babel}  % Escribir con acentos sin necesidad de comandos \'{}
\usepackage{setspace}  % Paquete para el ajuste de espacio entre líneas
\usepackage{fancyhdr}  % Paquete para el ajuste de formato de página
\usepackage{epsfig}
\usepackage{epic}
\usepackage{eepic}
\usepackage{amsmath}  % Paquete para diversas herramientas matemáticas
\usepackage{amssymb}  % Paquete para rotar tablas, figuras, etc.
\usepackage{mathtools}  % Paquete para el símbolo :=
\usepackage{kbordermatrix}  % Paquete para matrices con índices de filas y columnas
\usepackage{upgreek}  % Paquete para letras griegas en roman upright
\usepackage{threeparttable}
\usepackage{amscd}
\usepackage{here}
\usepackage{nicefrac}  % Paquete para fracciones en diagonal
\usepackage[binary-units=true,angle-mode=decimal]{siunitx}  % Paquete para unidades en SI
\sisetup{output-decimal-marker={,}}  % Establece la coma para decimales
\usepackage{tikz}
\usepackage{flowchart}  % Paquete con formas predefinidas de diagramas de flujo
\usetikzlibrary{arrows, arrows.meta}
\usetikzlibrary{automata}  % Paquete para dibujar autómatas
\usetikzlibrary{positioning}  % Paquete con herramientas para facilitar la ubicación de nodos
\usetikzlibrary{math}  % Paquete para definir variables
\usepackage{grafcet}  % Paquete para dibujar gráficos de etapas y transiciones
\usepackage[simplified]{pgf-umlcd}  % Paquete para dibujar diagramas UML
\usepackage[algochapter,linesnumbered,vlined,ruled,commentsnumbered]{algorithm2e}  % Paquete para escribir pseudocódigo
\usepackage{graphicx}  % Paquete para poder usar gráficos
\usepackage{tabularx}  % Paquete para poder usar tablas
\usepackage{longtable}
\usepackage{xcolor}  % Paquete para definir colores
\usepackage{caption}
\usepackage{subcaption}
\usepackage{listings}  % Paquete para entornos tipo listings
\usepackage[chapter]{minted}  % Paquete para códigos con formato
\usepackage{lscape}
\usepackage{lipsum}  % Paquete para simular párrafos de textos
% Paquete para habilitar hipervínculos habilitando el cambio de línea
\PassOptionsToPackage{hyphens}{url}\usepackage[colorlinks=true,linkcolor=blue,urlcolor=gray,citecolor=blue]{hyperref}
\usepackage[all]{hypcap}  % Paquete para ajustar las referencias a floats por encima de estos
\usepackage{natbib}  % Paquete mejorado para bibliografía
\usepackage{mathspec} % Importa el paquete para el uso de fuentes externas
\usepackage{fancyvrb}
\usepackage{rotating}
\usepackage{gettitlestring}

% Establece la fuente Ancizar Serif como la fuente principal
\setmainfont[Ligatures = TeX, UprightFont = {* Regular}, BoldFont = {* Bold},
				ItalicFont = {* Italic}, BoldItalicFont = {* Bold Italic}]{Ancizar Serif}
% Establece la fuente Ancizar Sans como la fuente Sans
\setsansfont[Ligatures = TeX, UprightFont = {* Regular}, BoldFont = {* Bold},
				ItalicFont = {* Italic}, BoldItalicFont = {* Bold Italic}]{Ancizar Sans}
% Establece l afuente JetBrains Mono como la fuente monoespaciada
\setmonofont[Ligatures = TeX, UprightFont = {* Regular}, BoldFont = {* Bold},
				ItalicFont = {* Italic}, BoldItalicFont = {* Bold Italic}]{JetBrains Mono}


% ---------- CONFIGURACIÓN DE PÁGINA ----------
\pagestyle{fancyplain}
\addtolength{\headwidth}{\marginparwidth}
\textheight22.5cm \topmargin0cm \textwidth16.5cm
\oddsidemargin0.5cm \evensidemargin-0.5cm
\renewcommand{\chaptermark}[1]{\markboth{\thechapter\; #1}{}}
\renewcommand{\sectionmark}[1]{\markright{\thesection\; #1}}
\lhead[\fancyplain{}{\thepage}]{\fancyplain{}{\rightmark}}
\rhead[\fancyplain{}{\leftmark}]{\fancyplain{}{\thepage}}
\fancyfoot{}
\thispagestyle{fancy}
\unitlength1mm  % Define la unidad LE para Figuras
\mathindent1cm  % Define la distancia de las formulas al texto,  fleqn las descentra
\marginparwidth0cm
\parindent0cm  % Define la distancia de la primera linea de un parrafo a la margen
% Establece la distancia antes y después de las ecuaciones
\makeatletter
\g@addto@macro\normalsize{
  \setlength\abovedisplayskip{7pt}
  \setlength\belowdisplayskip{7pt}
  \setlength\abovedisplayshortskip{7pt}
  \setlength\belowdisplayshortskip{7pt}}
\makeatother
\allowdisplaybreaks  % Permite hacer saltos de página en entornos align
% Para tablas,  redefine el backslash en tablas donde se define la posición del texto en las
% casillas (con \centering \raggedright o \raggedleft)
\newcommand{\PreserveBackslash}[1]{\let\temp=\\#1\let\\=\temp}
\let\PBS=\PreserveBackslash

\setlength{\parskip}{1em}  % Espacio entre párrafos
\captionsetup[sub]{font=footnotesize}

% Configuración para permitir referenciar items en el entorno description
\makeatletter
\let\orgdescriptionlabel\descriptionlabel
\renewcommand*{\descriptionlabel}[1]{%
  \let\orglabel\label
  \let\label\@gobble
  \phantomsection
  \edef\@currentlabel{#1}%
  %\edef\@currentlabelname{#1}%
  \let\label\orglabel
  \orgdescriptionlabel{#1}%
}
\makeatother
% ----------------------------------------------------

% Creación de formas para dibujo
\tikzstyle{print}=[trapezium, draw, text centered, trapezium left angle=60, trapezium right angle=120, minimum height=2em]  % Trapezoide para el diagrama de flujo
\tikzstyle{conditional}=[diamond, draw, text centered, aspect=3]  % Rombo para el bloque de decisión.
\tikzstyle{arrow} = [thick,->,>=stealth]

% Configuraciones para los códigos de Python
\definecolor{background_color}{HTML}{dddddd}
\definecolor{rule_color}{HTML}{c9c7c7}
\definecolor{blue_sky}{HTML}{D9F1FA}
\definecolor{softOrange}{HTML}{FEF7DA}
\definecolor{soft_red}{HTML}{FFD4DE}

\newenvironment{code}{\captionsetup{type=listing, labelfont=bf}\vspace{15pt}}{\vspace{-20pt}}
% \DeclareFloatingEnvironment[fileext=frm,placement={!hbt},name=Code]{code}
% \captionsetup[code]{type=listing, labelfont=bf}
\setminted[python]{linenos=true, breaklines=true, breakafter=-/, fontsize=\tiny,
				   frame=lines, rulecolor=rule_color, framesep=2pt,
				   numbersep=8pt, python3=true, style=stata, tabsize=4, bgcolor=background_color}

% Definición de alias para referencias bibliográficas
\defcitealias{Scikit-learndevelopers2022}{scikit Developers, 2022}
\defcitealias{Bruel2021}{B\&K, 2021}
\defcitealias{IEC_TC29_2017}{IEC-TC29, 2017}
\defcitealias{Bruel2016}{B\&K, 2016}
\defcitealias{IEC_TC29_2013_1}{IEC-TC29, 2013}
\defcitealias{Keysight2015}{Keysight, 2015}
\defcitealias{Keysight2022}{Keysight, 2022}
\defcitealias{MHJSoftware2020}{MHJ-Software, 2020}

% SOBREESCRITURA Y DECLARACIÓN DE COMANDOS
\renewcommand{\theequation}{\thechapter-\arabic{equation}}
\renewcommand{\thefigure}{\textbf{\thechapter-\arabic{figure}}}
\renewcommand{\thetable}{\textbf{\thechapter-\arabic{table}}}
%\renewcommand{\thelstlisting}{\bfseries\thechapter-\arabic{code}}
\renewcommand{\baselinestretch}{1.1} % Espacio base entre lineas
\renewcommand{\tablename}{\textbf{Tabla}}
\renewcommand{\figurename}{\textbf{Figura}}
\renewcommand{\listtablename}{Lista de tablas}
\renewcommand{\listfigurename}{Lista de figuras}
\renewcommand{\contentsname}{Contenido}
\renewcommand{\listingscaption}{Código}
\newcommand{\clearemptydoublepage}{\newpage{\pagestyle{empty}\cleardoublepage}}
% Formato de números de línea en códigos
\renewcommand{\theFancyVerbLine}{\sffamily\textcolor[HTML]{808080}{\tiny\oldstylenums{\arabic{FancyVerbLine}}}}
\renewcommand*{\algorithmcfname}{Algoritmo}
% Comandos para ecuaciones
\newcommand{\vect}[1]{\mathrm{\mathbf{#1}}}  % Abreviación de comando para notación de vectores y matrices
\newcommand{\defeq}{\vcentcolon=}  % Declaración de comando para el símbolo :=
% Comandos para unidades especiales
\DeclareSIUnit\mVpp{mVpp}
\DeclareSIUnit\Vpp{Vpp}
\DeclareSIUnit\Vrms{Vrms}
\DeclareSIUnit\uVrms{\upmu Vrms}

%  ╦┌┐┌┬┌─┐┬┌─┐  ┌┬┐┌─┐┬    ┌┬┐┌─┐┌─┐┬ ┬┌┬┐┌─┐┌┐┌┌┬┐┌─┐
%  ║│││││  ││ │   ││├┤ │     │││ ││  │ ││││├┤ │││ │ │ │
%  ╩┘└┘┴└─┘┴└─┘  ─┴┘└─┘┴─┘  ─┴┘└─┘└─┘└─┘┴ ┴└─┘┘└┘ ┴ └─┘ 

\begin{document}
\pagenumbering{roman}
%\newpage
%\setcounter{page}{1}
% ---- PRIMERA PÁGINA DE PORTADA ----
\begin{center}
    \capstartfalse  % Omite el caption de la primera imagen
    \begin{figure}
        \centering
        \epsfig{file=0_Portada/EscudoUN,scale=1}
    \end{figure}
    \thispagestyle{empty}
    % \vspace*{0.2cm}
    \setstretch{1.5}\huge\textbf{Desarrollo de un procedimiento de calibración de sonómetros
    y calibradores acústicos de conformidad con las normas \mbox{IEC 61672--3} e \mbox{IEC 60942}}\\[4.0cm]
    \setstretch{1.1}
    \Large\textbf{Juan Felipe Maldonado Pedraza}\\[4.0cm]
    \small Universidad Nacional de Colombia\\
    Facultad de Ingeniería, Departamento de Ingeniería Eléctrica y Electrónica\\
    Bogotá, Colombia\\
    2021
\end{center}

\newpage{\pagestyle{empty}\cleardoublepage}
\newpage

% ---- SEGUNDA PÁGINA DE PORTADA ----
\begin{center}
    \thispagestyle{empty}
    \setstretch{1.3}\huge\textbf{Desarrollo de un procedimiento de calibración de sonómetros
    y calibradores acústicos de conformidad con las normas \mbox{IEC 61672--3} e \mbox{IEC 60942}}\\[3.0cm]
    \setstretch{1.1}
    \Large\textbf{Juan Felipe Maldonado Pedraza}\\[3.0cm]
    \small Trabajo de grado presentado como requisito parcial para optar al título de:\\
    \textbf{Master en Automatización Industrial}\\[2.0cm]
    Director:\\
    Ph. D.\ Leonardo Enrique Bermeo Clavijo\\[2.0cm]
    Línea de investigación:\\
    Automatización en procesos de metrología\\
    % Grupo de Investigación:\\
    % Nombrar el grupo en caso que sea posible\\[2.5cm]
    Universidad Nacional de Colombia\\
    Facultad de Ingeniería\\
    Bogotá, Colombia\\
    2022
\end{center}

\newpage{\pagestyle{empty}\cleardoublepage}

% ---- DEDICATORIA O LEMA ----
\newpage
\thispagestyle{empty}\normalsize \vspace*{3cm}

\begin{flushright}
    \begin{minipage}{8cm}
        \setlength{\parskip}{2em}  % Espacio entre párrafos
        \noindent
        \small
        \begin{center}
            ``Con Dios está la sabiduría y la fortaleza;\\suyo es el consejo y la inteligencia.''
        \end{center}
        \begin{flushright}
            Job 12:13 (RVG)
        \end{flushright}
    \end{minipage}
\end{flushright}

\newpage{\pagestyle{empty}\cleardoublepage}

% ---- AGRADECIMIENTOS ----
\newpage
\thispagestyle{empty}\normalsize \vspace*{3cm}

%\textbf{\LARGE Agradecimientos} \vspace{2em}
\chapter*{Agradecimientos}
\addcontentsline{toc}{chapter}{Agradecimientos}
Al Dios de mi supremo gozo, quien me ciñe de entendimiento y vigor para culminar trabajos como este.
A Él sea dado el honor y las gracias porque, a pesar de mí, su favor no me ha faltado jamás.

A mis hermanos en la fe y a mi familia, quienes toleraron mi ausencia y no obstante conté con su apoyo mientras
dedicaba mi tiempo a terminar este trabajo.

A mi director de trabajo de grado, por su importante orientación en el desarrollo del proyecto.

A todos ellos, gracias.

\newpage{\pagestyle{empty}\cleardoublepage}

% ---- RESUMEN ----
\newpage
%\textbf{\LARGE Resumen} \vspace{2em}
\chapter*{Resumen}
\addcontentsline{toc}{chapter}{Resumen}
En este documento se describe el desarrollo de un sistema de calibración de instrumentos acústicos de conformidad con
las
normas internacionales \mbox{IEC 60942} e \mbox{IEC 61672--3}.
Se inicia con una breve revisión de los sistemas desarrollados hasta ahora, analizando sus ventajas y oportunidades de
mejora.
En seguida se hace un estudio del marco normativo y de la naturaleza de los equipos bajo prueba, y se determina la
instrumentación necesaria para el sistema de calibración.
El sistema de calibración de sonómetros es controlado por una aplicación que emplea reconocimiento de imágenes, por lo
que después del marco normativo se explica el método diseñado para el reconocimiento de caracteres numéricos, y se
incluyen
resultados del procesamiento y desempeño del clasificador.
Luego, se introduce una sección en la que se describen los detalles de implementación de las aplicaciones de
\emph{software}
codificadas en Python;
particularmente, se presenta el modelo GRAFCET que fue la base del desarrollo de la aplicación para calibradores
acústicos.
A continuación, se explica el diseño e implementación del método propuesto para modelar la variabilidad de un valor
de medición
como un proceso estocástico usando cadenas de Markov y se muestra un ejemplo de una matriz de transición obtenida y el
cálculo del valor esperado.
Se concluye el documento subrayando los logros alcanzados, las recomendaciones de uso del sistema de calibración
y sugerencias de trabajo futuro.

\textbf{\small Palabras clave: Calibración, metrología, sonómetros, calibradores acústicos, automatización,
    visión de máquina, cadenas de Markov} \vspace{2em}

\vfill
\pagebreak

%\textbf{\LARGE \textsf{Abstract}} \vspace{2em}
\chapter*{Abstract}
This document describes the development of a calibration system for acoustic instruments in accordance with the
international standards \mbox{IEC 60942} and \mbox{IEC 61672--3}.
It begins with a brief review of the systems developed so far, analyzing their advantages and opportunities for
improvement.
Next, a study of the regulatory framework and the nature of the equipment under test is made, and the necessary
instrumentation for the calibration system is determined.
The sound level meter calibration system is controlled by an application that uses image recognition, so after the
normative framework, the method designed for the recognition of numerical characters is explained, and processing
results
and performance of the classifier are included.
Then, a section is introduced describing the implementation details of software applications coded in Python.
In particular, the GRAFCET model is presented, which was the basis for the development of the application for acoustic
calibrators.
Next, the design and implementation of the proposed method to model the variability of a measurement value as a
stochastic
process using Markov chains is explained, and an example of a transition matrix obtained and the calculation of the
expected value is shown.
The document is concluded highlighting the achievements, the recommendations for the use of the calibration system, and
suggestions for future work.

\textbf{\small Keywords: Calibration, metrology, sound level meters, acoustic calibrators, automation,
    computer vision, Markov chains}
\vfill
\cleardoublepage
\setlength{\jot}{10pt}  % Espacio interlineal por defecto en el entorno align
\tableofcontents
\addcontentsline{toc}{chapter}{Índice general}
\cleardoublepage
\listoffigures  % Índice de figuras
\addcontentsline{toc}{chapter}{Índice de figuras}  % Incluye la lista de figuras en el índice de contenido
\cleardoublepage
\listoftables % Índice de tablas
\addcontentsline{toc}{chapter}{Índice de cuadros} % Incluye la lista de tablas en el índice de contenido
\cleardoublepage
\capstarttrue  % Inicia el caption de los flotantes
\pagenumbering{arabic}
\chapter{Introducción}
%TODO
% Importancia del trabajo en el contexto
% Propósito del trabajo
% Principales logros del trabajo
% Visión general del trabajo
% Síntesis y organización del documento


\section{Planteamiento del problema}
En la actualidad, la creciente contaminación acústica amerita la implementación de redes de monitoreo continuo de ruido
o mediciones puntuales empleando instrumentos adecuados (entre estos los sonómetros y calibradores acústicos), con el
propósito de cuantizar los niveles de ruido ambiental, de emisión de ruido de fuentes sonoras específicas y de
exposición sonora, para luego comparar con los niveles máximos permitidos por la normativa nacional e internacional
relacionada y tomar decisiones al respecto.
Para garantizar la confiabilidad de tales mediciones o asegurar la validez de sus resultados, en Colombia, entidades
como el Instituto de Hidrología, Meteorología y Estudios Ambientales (IDEAM) exigen que las organizaciones que prestan
estos servicios cuenten con sonómetros calibrados periódicamente bajo el estándar internacional
\mbox{IEC 61672--3}~\citeyearpar{IEC_TC29_2013_3}, y con calibradores acústicos calibrados periódicamente de acuerdo con
el estándar internacional \mbox{IEC 60942, Anexo B}~\citeyearpar{IEC_TC29_2017}, por parte de un organismo de evaluación
de la conformidad (OEC), en este caso un laboratorio de calibración acreditado por el Organismo Nacional de Acreditación
de Colombia (ONAC) en \mbox{ISO 17025}~\citeyearpar{ISO_CASCO_2017}, con el fin de verificar que estos instrumentos
continúan cumpliendo las especificaciones normalizadas según su clase.

Los calibradores acústicos normalmente son empleados como patrones de trabajo en las verificaciones de campo periódicas
de los sonómetros.
Para estos calibradores, el Anexo B de la norma \mbox{IEC 60942} establece tres pruebas que evalúan el funcionamiento
del calibrador en tres magnitudes:
1) Nivel de presión acústica,
2) frecuencia y
3) distorsión armónica más ruido (THD+N).

Por su parte, la norma \mbox{IEC 61672--3} describe una serie de pruebas acústicas y eléctricas que se realizan a
sonómetros integradores clase 1 y 2, cuyo propósito es comprobar el funcionamiento del sonómetro en:
1) La sensibilidad de su micrófono (para lo cual se usa un calibrador acústico calibrado previamente y que esté en
conformidad con las especificaciones de la \mbox{IEC 60942}).
2) Las redes de ponderación frecuencial A, C y Z. 3) Las ponderaciones temporales F (\emph{fast}) y S (\emph{slow}).
4) El rango lineal de niveles.
5) La medición de niveles promediados en el tiempo, niveles de exposición sonora y niveles pico.
6) La indicación de sobrecarga.
7) La exposición a largos periodos de medición y a niveles de sonido elevados.
Dicha comprobación se hace comparando con las especificaciones definidas en la norma \mbox{IEC 61672-1:2013}, según la
clase del sonómetro.

A continuación se presenta una revisión de los avances en el desarrollo de sistemas de calibración de sonómetros,
considerando especialmente su versatilidad (que el alcance de su aplicación abarque sonómetros de cualquier marca), su
actualidad (que siga los lineamientos de la última versión de la norma internacional), su funcionalidad (que ofrezca
herramientas útiles como procesamiento, análisis, almacenamiento y presentación de datos que faciliten la gestión
metrológica) y su costo.


\section{Antecedentes}

\subsection{Sistemas de calibración comerciales desarrollados por fabricantes}
\begin{figure}[!h]
    \caption{Estación de medición para la calibración de instrumentos acústicos de medida en el laboratorio AP146 en
    Polonia.}
    \label{fig:AP146Laboratory}
    \centering
    \includegraphics[width=0.8\textwidth]{1_Intro/AP146Laboratory}
    \caption*{\footnotesize Tomado de~\cite{Podgorski2016}}
\end{figure}

Es común encontrar que las organizaciones fabricantes de sonómetros adicionalmente presten servicios de calibración
acreditada (e.g.\ Svantek, Brüel \& Kjær y 01dB), por lo que prácticamente también se pueden denominar OEC\@.
Pero estos OEC en especial, tienen la posibilidad de implementar procedimientos automáticos de calibración sin
limitaciones, puesto que tienen acceso a los comandos de control específicos de sus modelos de sonómetros.
Un ejemplo de eso se encuentra en el artículo de~\cite{Podgorski2016}, en el que se hace una descripción del laboratorio
AP146 en Polonia que pertenece a Svantek y está acreditado por el Centro Polaco de Acreditación (PCA).
En ese artículo se presenta el desarrollo de un \emph{software} de calibración que han validado, desde el que se envían
instrucciones a los instrumentos que generan las señales y a los que las miden, también es posible generar
automáticamente reportes de resultados.
No obstante, sólo es posible usar este \emph{software} con los sonómetros que fabrica la compañía.
Para sonómetros de otras marcas las configuraciones deben hacerse manualmente y por ende el tiempo total de calibración
llega a ser por lo menos el doble.
La figura~\ref{fig:AP146Laboratory} muestra la estación de medición de este laboratorio.

\begin{figure}[!h]
    \caption{Sistema de calibración de sonómetros Type 3630-A desarrollado por Brüel \& Kjær.}
    \label{fig:BK3630A}
    \centering
    \includegraphics[width=0.8\textwidth]{1_Intro/BK3630A}
    \caption*{\footnotesize Tomado de~\cite{BruelKjaer2000}}
\end{figure}

La figura~\ref{fig:BK3630A} muestra el Type 3630-A, uno de los sistemas de calibración más completo y compacto.
Fue desarrollado por Brüel \& Kjær~\citeyearpar{BruelKjaer2000}, que permite la calibración periódica y la estimación
de incertidumbre de medida para sonómetros Brüel \& Kjær y también de otras marcas.
Además, ofrece la posibilidad de calibrar en modo automático (si el sonómetro dispone de una interfaz serial),
semiautomático (si el sonómetro tiene una salida análoga que corresponda satisfactoriamente con la indicación en
pantalla) y manual (si el sonómetro no cuenta con ninguna salida análoga), con secuencias predefinidas o personalizadas
por el usuario.
Tiene una base de datos de clientes e instrumentos integrada, lo que permite la trazabilidad de los intervalos de
calibración de los patrones de trabajo.
Es capaz de generar certificados de calibración con detallados reportes de resultados.
Permite calibrar dosímetros, calibradores acústicos y filtros.
Sin embargo, por defecto, únicamente tiene disponibles los ensayos de acuerdo con la norma \mbox{IEC 60651} y la
\mbox{IEC 60804}, que quedaron obsoletas rápidamente con el avance tecnológico~\citep{Beyers2014}.
Los ensayos de acuerdo con la \mbox{IEC 61672--3} están disponibles sólo como complementos que el usuario debe comprar.

\subsection{Sistemas de calibración desarrollados por otras organizaciones}
Los anteriores sistemas de calibración desarrollados por los fabricantes más prominentes son sofisticados y costosos,
por lo que los demás OEC, que en su mayoría son organizaciones independientes, deben implementar su propia estación de
trabajo y desarrollar su propio procedimiento de calibración.
El hecho que se desconozca la codificación de control del sonómetro propia de cada fabricante sugiere que lo más seguro
es que este procedimiento sea manual.

Para abordar esta problemática, hay un avance en la automatización del procedimiento de calibración, extendiendo el
alcance a cualquier sonómetro sin importar su modelo o marca, este es el trabajo realizado por~\cite{Zhong2010},
en el que desarrollaron un sistema de calibración implementando reconocimiento de imágenes.
De modo que el resultado es obtenido automáticamente de la indicación en la pantalla del sonómetro, superando así la
limitación más común en los sistemas automáticos de calibración de los fabricantes.
Tal funcionamiento hace de este un sistema mucho más versátil, con el que igualmente se pueden generar de forma
automática los reportes con los resultados de calibración.
Sin embargo, dada su fecha de publicación (2010), está basado en la versión anterior de la
\mbox{IEC 61672--3}~\citeyearpar{IEC_TC29_2013_3}.
Y, por otro lado, utiliza un método de calibración distinto, en el que se requiere una cámara anecóica para calibrar en
campo libre las frecuencias de $\qty{500}{\Hz}$ a $\qty{20}{\kHz}$, un acoplador activo para calibrar en campo de
presión las frecuencias de $\qty{10}{\Hz}$ a $\qty{500}{\Hz}$ y un micrófono patrón de laboratorio para obtener las
respuestas de referencia de los campos, lo que eleva el costo.

La anterior revisión de sistemas de calibración desarrollados por fabricantes y por otros laboratorios indica que hace
falta un sistema de calibración de sonómetros actualizado, versátil, y de bajo costo, pero que no comprometa los
resultados.
Es en esa vía que se planteó este trabajo y se da cumplimiento a los siguientes objetivos.


\section{Objetivos}

\textbf{General}

Desarrollar un sistema de calibración periódica de sonómetros y calibradores acústicos de conformidad con las normas
\mbox{IEC 61672-3:2013} e {IEC 60942:2017}.
\vfill
\pagebreak

\textbf{Específicos}

\begin{enumerate}
    \item Formular un modelo en GRAFCET como base para el desarrollo de un sistema de calibración periódica de
    calibradores acústicos.
    \item Implementar las secuencias de comando (a través de bus GPIB) para configurar parámetros de señal y, a su vez,
    recibir resultados de los instrumentos de medición.
    \item Desarrollar un método de reconocimiento de imágenes para detectar los niveles instantáneos ponderados en
    tiempo y en frecuencia desde la pantalla del sonómetro.
    \item Desarrollar un método que permita tener en cuenta la variabilidad de los niveles en pantalla instantáneos
    ponderados en tiempo y en frecuencia del objetivo 3, (mediante mediciones de larga duración), para la estimación
    del mesurando y de la incertidumbre de medición.
\end{enumerate}

\subsection{Alcance de los objetivos}
El sistema de calibración se implementó para ejecutar las pruebas de calibración de los numerales 9.3 (apoyado en la
IEC 60942), 13, 14 y 16 de la \mbox{IEC 61672--3}.
Los indicadores de interés son los niveles instantáneos con ponderación temporal (\emph{slow} o \emph{fast}) y
ponderación frecuencial ($A$, $C$, o $Z$), i.e. $L_{AF}$,$L_{AS}$,$L_{CF}$,$L_{CS}$,$L_{ZF}$ o $L_{ZS}$, dependiendo
de la prueba y según estén disponibles en el sonómetro sujetos al periodo de actualización de la pantalla del sonómetro.


\section{Estructura del documento}
Antes de introducir el desarrollo técnico de los objetivos, este documento inicia con un capítulo de metodología e
instrumentación en el que se describen el marco normativo y los equipos utilizados en el sistema desarrollado.
Luego, en cumplimiento del tercer objetivo, el capítulo dos explica el diseño e implementación del algoritmo
para el reconocimiento de los caracteres numéricos que representan los niveles de sonido mostrados en la pantalla del
sonómetro.
En el capítulo cuatro se presenta el GRAFCET del objetivo uno y los detalles de implementación del objetivo dos en las
aplicaciones desarrolladas en Python.
En el último capítulo se explican las consideraciones teóricas-experimentales que forman el punto de partida del método
del cuarto objetivo.
Asimismo, se incluye el marco teórico de las cadenas de Markov que son la herramienta esencial del
método y se expone un resultado de ejemplo.
Finalmente, se concluye el documento con una revisión del trabajo realizado y sugerencias para desarrollo futuro.
%  ╔╦╗┌─┐┌┬┐┌─┐┌┬┐┌─┐┬  ┌─┐┌─┐┬┌─┐  ┌─┐  ╦┌┐┌┌─┐┌┬┐┬─┐┬ ┬┌┬┐┌─┐┌┐┌┌┬┐┌─┐┌─┐┬┌─┐┌┐┌
%  ║║║├┤  │ │ │ │││ ││  │ ││ ┬│├─┤  ├┤   ║│││└─┐ │ ├┬┘│ ││││├┤ │││ │ ├─┤│  ││ ││││
%  ╩ ╩└─┘ ┴ └─┘─┴┘└─┘┴─┘└─┘└─┘┴┴ ┴  └─┘  ╩┘└┘└─┘ ┴ ┴└─└─┘┴ ┴└─┘┘└┘ ┴ ┴ ┴└─┘┴└─┘┘└┘

\chapter{Metodología e instrumentación}

En este capítulo se hace una descripción de los instrumentos bajo calibración (calibradores acústicos y sonómetros), seguida de un resumen de los lineamientos de las normas internacionales para las calibraciones periódicas, incluyendo expresiones matemáticas para el cálculo de los voltajes de prueba y otras consideraciones prácticas para los ensayos.
Luego se presentan los patrones y otros instrumentos importantes necesarios para la calibración junto con los esquemas de interconexión.
Finalmente se explican brevemente los comandos para el control remoto de los instrumentos de medición.


\section{Instrumentos bajo calibración}

\subsection{Calibradores acústicos}
De acuerdo con la normativa internacional, un calibrador acústico es un dispositivo diseñado para producir uno o más niveles de presión sonora conocidos (en $\unit{\dB}$ referenciados a $\qty{20}{\micro\Pa}$) a una o más frecuencias especificadas (en $\unit{\Hz}$) cuando se acopla a modelos específicos de micrófono en configuraciones específicas \citepalias{IEC_TC29_2017}.
%
\begin{figure}[!h]
    \caption{Calibrador acústico multifución Brüel \& Kjær 4226.}
    \label{fig:bruel_4226}
    \centering
    \includegraphics[width=0.35\textwidth]{2_Metodología/Figs/bruel4226}
    \caption*{\footnotesize Tomado de \scriptsize
    \url{https://www.transcat.com/bruel-kjaer-4226-acoustic-calibrator-94-104-and-114db-used}}
\end{figure}

Normalmente, la señal senoidal es generada por algún transductor, como un altavoz o, en el caso de los pistófonos, un pistón mecánico cuyo movimiento genera en la cavidad una velocidad de volumen conocida.
Como ejemplo, en la figura~\ref{fig:bruel_4226} se muestra un calibrador acústico multifunción usado como referencia en muchos laboratorios: el Brüel \& Kjær 4226, que es capaz de generar $\qtylist{94; 104; 114}{\dB}$ en las frecuencias de octava desde $\qty{31.5}{\Hz}$ hasta $\qty{16}{\kHz}$, más la frecuencia de $\qty{12.5}{\Hz}$.

\begin{figure}[!h]
    \caption{Calibrador acústico Brüel \& Kjær 4231 acoplado al micrófono de un sonómetro Brüel \& Kjær 2250.}
    \label{fig:bruel_4231_coupled}
    \centering
    \includegraphics[width=0.45\textwidth]{2_Metodología/Figs/bruel4231coupled}
    \caption*{\footnotesize Tomado de \scriptsize \url{https://www.bksv.com/en/knowledge/blog/sound/getting-started-sound-level-meter}}
\end{figure}
%
Generalmente los calibradores acústicos son empleados para determinar la sensibilidad en campo de presión (típicamente en $\nicefrac{\unit{\mV}}{\unit{\Pa}}$ o en $\unit{\decibel}$ referenciados a $\qty{1}{\V}$) de modelos especificados de micrófonos en configuraciones dadas, pero también es utilizado para verificar o ajustar la sensibilidad de algún dispositivo o sistema de medición acústica.
Un ejemplo de calibrador acoplado para comprobar la indicación de un sonómetro se muestra en la figura~\ref{fig:bruel_4231_coupled}.

La norma \mbox{IEC 60942}~\citeyearpar{IEC_TC29_2017} establece una clasificación de los calibradores según sus especificaciones (límites de aceptación), de la más a la menos restrictiva: Clase LS (\emph{laboratory standard}), clase 1 o clase 2.
La comprobación de que cierto modelo de calibrador cumple con todas las especificaciones normalizadas según su clase la realiza una organización independiente acreditada para hacer pruebas de aprobación de modelo de acuerdo con los lineamientos de la \mbox{IEC 60942} (\mbox{Anexo A}~\citeyear{IEC_TC29_2017}).
Adicionalmente, un usuario de un calibrador acústico debería calibrar periódicamente su instrumento para garantizar la trazabilidad a los estándares nacionales y la confiabilidad de sus resultados.
Esta calibración periódica es llevada a cabo por organismos evaluadores de la conformidad acreditados en \mbox{ISO 17025}~\citeyearpar{ISO_CASCO_2017} para realizar los ensayos periódicos de acuerdo con la \mbox{IEC 60942} (\mbox{Anexo B}~\citeyear{IEC_TC29_2017}).
\vfill

\begin{figure}[!hp]
    \caption{Configuraciones de \emph{hardware} del sonómetro Brüel \& Kjær 2250.}
    \label{fig:bruel_2250_set}
    \centering
    \includegraphics[width=\textwidth]{2_Metodología/Figs/bruel2250set}
    \caption*{\footnotesize Tomado del Manual de Instrucciones \citepalias{Bruel2016}}
\end{figure}

\subsection{Sonómetros integradores}
Brüel \& Kjær, uno de los fabricantes más prominentes de sonómetros define consistentemente los conceptos básicos sobre dichos instrumentos en uno de sus artículos \citepalias{Bruel2021}.
Básicamente, un sonómetro es un instrumento diseñado para medir niveles de sonido de una forma estandarizada;
su respuesta al sonido se asemeja a la del oído humano y proporciona medidas de niveles de presión sonora objetivas y reproducibles.
Generalmente, los sonómetros son empleados en el monitoreo de ruido proveniente de diversas fuentes sonoras, como plantas industriales, tráfico rodado, aeronáutico o ferroviario, conciertos, etc.
Como se puede ver en la figura~\ref{fig:bruel_2250_set}, un sonómetro típico consta de un micrófono, un preamplificador, una unidad de procesamiento de señal (interna) y una pantalla.
Regularmente el preamplificador hace parte del cuerpo del sonómetro, pero no siempre es el caso;
un sonómetro podría estar provisto de cables de extensión que separen el preamplificador de la unidad de procesamiento.

En cuanto al flujo de señal, el micrófono es un transductor electroacústico que transforma la señal acústica en una señal eléctrica.
La mayoría de los micrófonos empleados en mediciones acústicas son de condensador, y gracias a su construcción es el mejor tipo para garantizar precisión, estabilidad y confiabilidad en los resultados.\ No obstante, la señal eléctrica proporcionada por un micrófono es de baja amplitud (aún con micrófonos de alta gama cuya sensibilidad se encuentra típicamente en el orden de los $50\,\nicefrac{\unit{\mV}}{\unit{\Pa}}$, por lo que se requiere una amplificación para que la unidad de procesamiento manipule la señal en un nivel adecuado, este es el objetivo del preamplificador.
Luego, en la unidad de procesamiento se ejecutan diferentes cálculos a partir de la señal.
Los mínimos requeridos por la norma internacional \mbox{IEC 61672--1}~\citeyearpar{IEC_TC29_2013_1} y utilizados en este proyecto se detallan a continuación.

\begin{figure}[!h]
    \caption{Gráfico de las ponderaciones frecuenciales $A$, $C$ y $Z$.}
    \label{fig:frequency_weightings}
    \centering
    \includegraphics[width=0.54\textwidth]{2_Metodología/Figs/frequency-weighings}
    \caption*{\footnotesize Tomado de \citepalias{Bruel2021}.}
\end{figure}
%
\textbf{Ponderación frecuencial:} Diferencia (como una función especificada de la frecuencia) entre el nivel de la señal ponderada en frecuencia indicado en el dispositivo de presentación de resultados y el nivel correspondiente de una señal de entrada sinusoidal de amplitud constante \citepalias{IEC_TC29_2013_1}.
En la figura~\ref{fig:frequency_weightings} se pueden ver gráficamente las ponderaciones frecuenciales.

Estas ponderaciones frecuenciales estandarizadas $A$, $C$, o $Z$, para las bandas de tercio de octavas están definidas en la \mbox{IEC61672--11} (\mbox{Tabla }~\citeyear{IEC_TC29_2013_1}).
En concreto, cada una de estas ponderaciones modifican la respuesta del sonómetro frente a diferentes frecuencias de sonido.
Por ejemplo, la ponderación $A$ asemeja la respuesta en frecuencia al comportamiento del oído humano en en un rango medio de niveles, tomando como referencia la curva de igual sonoridad de $\qty{40}{\dB}$~\citep{Fletcher1933}, por tal motivo es el más empleado en ruido ambiental y ocupacional.
Pero el oído humano no tiene un comportamiento lineal, y la percepción del sonido varía con el nivel.
La ponderación $C$ está basada en la curva de igual sonoridad de $\qty{100}{\dB}$, por eso esta es empleada en la evaluación de niveles pico de sonidos altos.
Finalmente, la ponderación \emph{zero} $(Z)$ es completamente plana en todo el rango de frecuencias (sin tener en cuenta la respuesta del micrófono).

\textbf{Ponderación temporal:} Es una función exponencial temporal que modifica la respuesta temporal del sonómetro frente a las variaciones en el nivel de presión sonora.
Una comparación entre las respuestas en el tiempo de cada ponderación temporal se muestra en la siguiente figura.
%
\begin{figure}[!h]
    \caption{Gráfico de las respuestas en el tiempo de las ponderaciones temporales \emph{fast}, \emph{slow} e \emph{impulse}.}
    \label{fig:time_weightings}
    \centering
    \includegraphics[width=0.55\textwidth]{2_Metodología/Figs/time-weightings}
    \caption*{\footnotesize Tomado de \citepalias{Bruel2021}.}
\end{figure}

Esta función exponencial obedece a una constante de tiempo especificada que depende de la ponderación temporal elegida, bien sea F (\emph{fast}, $\tau_F = \qty{125}{\ms}$), S (\emph{slow}, $\tau_S = \qty{1}{\s}$) o I (\emph{impulse}, $\tau_I = \qty{35}{\ms}$).
Por lo tanto, tal como lo define la norma, para una señal con ponderación $X$, el nivel de sonido con ponderación temporal $Y$ está dado por la siguiente ecuación:
%
\begin{equation}
    \label{eq:time_weighted_level}
    L_{XY}(t) = 10 \log\left(\frac{\frac{1}{\tau_Y}\,
    \int_{-\infty}^t p_X^2\left(\xi\right)\,e^{\nicefrac{-\left(t - \xi\right)}{\tau_Y}}\,\mathrm{d}\xi}
    {p_0^2}\right) \unit{\dB}.
\end{equation}
%
Donde $\tau_Y$ es la constante de tiempo en segundos de la ponderación temporal, $\xi$ es una variable ficticia del tiempo de integración desde un instante de tiempo en el pasado $(-\infty)$ hasta el instante de observación $t$, $p_X\left(\xi\right)$ es la señal de presión acústica instantánea con ponderación frecuencial $X$, y $p_0$ es el valor de referencia de $\qty{20}{\micro\Pa}$.

Consecuentemente, un nivel de sonido, objeto de evaluación en las pruebas aquí implementadas puede ser $L_{AF}$, $L_{AS}$, $L_{CF}$, $L_{CS}$, $L_{ZF}$ o $L_{ZS}$ para las ponderaciones frecuenciales $A$, $C$, o $Z$ y para las ponderaciones temporales \emph{fast} o \emph{slow}.
El resultado de la medición de nivel de sonido es mostrado directamente en la pantalla del sonómetro o alguna otra herramienta de visualización como una interfaz web.
En algunos sonómetros según su tecnología y disposiciones del fabricante, el resultado de medición es enviado vía serial o en forma de una señal DC o AC de amplitud proporcional al nivel de sonido.

La norma \mbox{IEC 61672--1}~\citeyearpar{IEC_TC29_2013_1} establece una clasificación de los sonómetros según sus especificaciones: clase 1 o clase 2.
La comprobación de que cierto modelo de sonómetro cumple con todas las especificaciones normalizadas, según su clase, la realiza una organización independiente acreditada para hacer pruebas de aprobación de modelo de acuerdo con los lineamientos de la \mbox{IEC61672--22}~\citeyearpar{IEC_TC29_2013_2}.
Pero también un usuario de un sonómetro debería calibrar periódicamente su instrumento para garantizar la trazabilidad a los estándares nacionales y la confiabilidad de sus resultados.
Esta calibración periódica es llevada a cabo por organismos evaluadores de la conformidad acreditados en \mbox{ISO 17025}~\citeyearpar{ISO_CASCO_2017} para realizar los ensayos periódicos de acuerdo con la \mbox{IE61672--3-3}~\citeyearpar{IEC_TC29_2013_3}.

Adicionalmente, la sensibilidad del transductor (micrófono) y la respuesta de los circuitos electrónicos puede variar con el paso del tiempo presentando unas pequeñas derivas o también pueden verse afectadas por las condiciones ambientales como la temperatura y la humedad.
Por esto, es una buena práctica verificar regularmente la sensibilidad del sonómetro (preferiblemente antes y después de cada campaña de medición).
De este modo, el sonómetro será ajustado a un nivel de referencia conocido emitido por un calibrador acústico cuyo nivel tenga trazabilidad metrológica.


\section{Métodos normalizados}

Las especificaciones y metodología de calibración de instrumentación acústica y de vibraciones son normalizadas por el comité técnico 29 de la Comisión Electrotécnica Internacional (IEC) en colaboración con la Organización Internacional de Metrología Legal (OIML).
A continuación, un resumen del proceso de calibración de calibradores acústicos y sonómetros, con especial enfoque en los pasos operativos, más que en las disposiciones preliminares de las normas.

\subsection{Descripción general de la calibración periódica de calibradores acústicos de acuerdo con la \mbox{IEC 60942:2017}}
\label{subsec:acoustic_calibrators_calibration_description}

En la siguiente figura se presenta un diagrama que describe en general el proceso de calibración de calibradores acústicos, el cual es explicado en detalle a continuación.
%
\begin{figure}[h]
    \caption{Diagrama de flujo general de la calibración periódica de calibradores acústicos.}
    \label{fig:acoustic_calibrator_calibration_flowchart}
    \centering
    \begin{tikzpicture}[font=\scriptsize, minimum width=2cm, minimum height=0.5cm]
        \node (start) at (-5, 0) [draw, terminal]{INICIO};
        \node (process1) [draw, process, align=center, below=0.3cm of start]{Acoplar calibrador \\ acústico patrón};
        \node (process2) [draw, process, align=center, below=0.3cm of process1]{Verificar calibrador \\ acústico patrón \\ $\mathrm{Acople} = 1$};
        \node (intersection1) [draw, circle, fill=black, inner sep=0pt, minimum size=2pt, below=0.3cm of process2]{};
        \node (intersection2) [draw, circle, fill=black, inner sep=0pt, minimum size=2pt, right=2.6cm of intersection1]{};
        \node (process3) [draw, process, align=center, below=0.3cm of intersection1]{Medir ruido \\ de fondo $(N)$};
        \node (decision1) [draw, conditional, below=0.3cm of process3]{$N \le L_{\mathrm{spec}} - \qty{30}{\dB}$};
        \node (process4) [draw, process, align=center,
            below=0.3cm of decision1]{Medir magnitudes \\ Voltaje, frecuencia y THD+N \\ $\mathrm{Acople\,} += 1$};
        \node (decision2) [draw, conditional, below=0.3cm of process4]{$\mathrm{Acople} == 4$};
        \node (process5) [draw, process, align=center, right=1.2cm of decision1]{Desacoplar calibrador \\ patrón, rotarlo y \\ volverlo a acoplar};
        \node (process6) [draw, process, align=center, below=0.3cm of decision2]{Calcular promedios \\ de magnitudes};
        \node (ankor1) [below right=0.3cm and 4.1cm of process6]{};
        \node (ankor2) [right=6.8cm of start]{};
        \node (process7) [draw, process, align=center,
            below=0.3cm of ankor2]{Desacoplar calibrador\\ patrón, acoplar \\ calibrador del cliente. \\ $\mathrm{Acople} = 1$};
        \node (intersection3) [draw, circle, fill=black, inner sep=0pt, minimum size=2pt, below=0.3cm of process7]{};
        \node (intersection4) [draw, circle, fill=black, inner sep=0pt, minimum size=2pt, right=2.6cm of intersection3]{};
        \node (process8) [draw, process, align=center, below=0.3cm of intersection3]{Medir ruido \\ de fondo $(N)$};
        \node (decision3) [draw, conditional, below=0.3cm of process8]{$N \le L_{\mathrm{spec}} - \qty{30}{\dB}$};
        \node (process9) [draw, process, align=center,
            below=0.3cm of decision3]{Medir magnitudes: \\ Nivel de presión acústica (por \\ comparación),
            frecuencia y THD+N \\ $\mathrm{Acople\,} += 1$};
        \node (decision4) [draw, conditional, below=0.3cm of process9]{$\mathrm{Acople} == 4$};
        \node (process10) [draw, process, align=center, right=1.2cm of decision3]{Desacoplar calibrador \\ del cliente, rotarlo y \\ volverlo a acoplar};
        \node (process11) [draw, process, align=center, below=0.3cm of decision4]{Calcular promedios \\ de magnitudes};
        \node (finish) [draw, terminal, below=0.3cm of process11]{FIN};

        \draw [arrow] (start.south) -- (process1);
        \draw [arrow] (process1) -- (process2);
        \draw [thick] (process2) -- (intersection1);
        \draw [arrow] (intersection1) -- (process3);
        \draw [arrow] (process3) -- (decision1);
        \draw [arrow] (decision1.south) -- node[at start, right]{Sí} (process4);
        \draw [arrow] (decision1.east) -| node[at start, above]{No} (intersection2);
        \draw [arrow] (intersection2) -- (intersection1);
        \draw [arrow] (process4) -- (decision2);
        \draw [arrow] (decision2.east) -| node[at start, above]{No}  (process5.south);
        \draw [arrow] (process5) |- (intersection2);
        \draw [arrow] (decision2.south) -- node[at start, right]{Sí} (process6);
        \draw [thick] (process6) |- (ankor1.center);
        \draw [thick] (ankor1.center) |- (ankor2.center);
        \draw [arrow] (ankor2.center) -- (process7);
        \draw [thick] (process7) -- (intersection3);
        \draw [arrow] (intersection4) -- (intersection3);
        \draw [arrow] (intersection3) -- (process8);
        \draw [arrow] (process8) -- (decision3);
        \draw [arrow] (decision3.east) -| node[at start, above]{No} (intersection4);
        \draw [arrow] (decision3.south) -- node[at start, right]{Sí} (process9);
        \draw [arrow] (process9) -- (decision4);
        \draw [arrow] (decision4.east) -| node[at start, above]{No} (process10);
        \draw [arrow] (process10) |- (intersection4);
        \draw [arrow] (decision4.south) -- node[at start, right]{Sí} (process11);
        \draw [arrow] (process11) -- (finish);
    \end{tikzpicture}
    \caption*{\footnotesize Fuente: Elaboración propia.}
\end{figure}

Tal como se describe en la \mbox{IEC 60942} (\mbox{Anexo B}~\citeyear{IEC_TC29_2017}), el calibrador acústico o pistófono con todos sus accesorios necesarios (como adoptadores o barómetro) debe ser entregado junto con el manual de instrucciones, si este es requerido por el laboratorio de calibración.
Luego, se hace una inspección visual del calibrador acústico, verificando que todos los controles están funcionando y que la fuente de alimentación está operando dentro de los límites especificados en el manual de instrucciones.
En seguida, se toman en cuenta o se realizan las siguientes secciones.

\textbf{Orientación para los ensayos}.
Si en el manual de instrucciones se especifica alguna orientación del calibrador acústico, esta debe ser la utilizada en la calibración periódica.

\textbf{Ruido ambiental}.
Para evitar que el ruido ambiental afecte las mediciones, las pruebas sólo se realizan si el nivel de presión sonora con el calibrador acoplado al micrófono (pero con el calibrador apagado) es por lo menos $\qty{30}{\dB}$ por debajo del nivel especificado que se está midiendo.

\textbf{Influencia de las condiciones ambientales}.
Cuando es apropiado, la información suministrada en el manual de instrucciones sobre la influencia de la presión estática debe ser aplicada para corregir el nivel de presión medido a la presión estática de referencia.

\textbf{Nivel de presión sonora}.
Después de acoplar el calibrador acústico al micrófono, se debe dejar el tiempo de estabilización indicado en el manual de instrucciones.
Luego, el nivel de presión sonora generado por el calibrador debe ser medido como un promedio de los valores instantáneos obtenidos durante un periodo $t_{\mathrm{op}}$ de entre $\qtylist{20;25}{\s}$ de operación.

Para medir el nivel de presión sonora hay propuestos dos métodos en la norma internacional: Usando un micrófono de referencia o usando un calibrador acústico de referencia para comparación.
En este proyecto se utiliza el segundo, en el que el nivel del calibrador bajo prueba es determinado por comparación contra el nivel generado por un calibrador acústico calibrado cuya trazabilidad metrológica esté establecida.
Como la señal a analizar es eléctrica, el nivel de presión se determina como una diferencia logarítmica:
\begin{equation}
    \label{eq:spl_from_voltage}
    L = L_{\mathrm{ref}} + 20\,\log\left(\frac{\bar{v}}{\bar{v}_{\mathrm{ref}}}\right).
\end{equation}
%
En que $L$ es el nivel de presión sonora del calibrador acústico bajo calibración, $L_{\mathrm{ref}}$ es el nivel de presión certificado del calibrador acústico patrón, y $\bar{v}$ y $\bar{v}_{\mathrm{ref}}$ son el voltaje medio medido con el calibrador del cliente y con el calibrador de referencia durante el tiempo $t_{\mathrm{op}}$ respectivamente.

El nivel de presión sonora debe ser medido al menos tres veces, cada vez acoplando el micrófono y el calibrador acústico antes de la medición y desacoplándolo después.
En cada nuevo acoplamiento se debe rotar el micrófono sobre su eje.
La diferencia absoluta entre el nivel medido medio y el nivel especificado no debe exceder los límites establecidos en la \mbox{IEC 60942} (\mbox{Tabla 2}~\citeyear{IEC_TC29_2017}), según la clase del calibrador y la frecuencia medida.
La medición de nivel de presión sonora debe ser repetida para cada combinación de nivel y frecuencia que indique el manual de instrucciones que cumple con las especificaciones de la norma.

\textbf{Frecuencia}, debe ser medida con el calibrador acoplado al micrófono como un promedio de los valores instantáneos obtenidos durante el tiempo $t_{\mathrm{op}}$, para cada frecuencia disponible en el calibrador, de la cual se indique en el manual que cumple con las especificaciones de la norma.
El valor absoluto de la diferencia porcentual a cada frecuencia medida (ver ecuación~\ref{eq:porcentual_difference}) y la correspondiente frecuencia especificada no debe exceder los límites establecidos en la \mbox{IEC 60942} (\mbox{Tabla 4}~\citeyear{IEC_TC29_2017}), según la clase del calibrador.
%
\begin{equation}
    \label{eq:porcentual_difference}
    \%error = \left|\frac{\bar{f}}{f_{\mathrm{spec}}} - 1\right| \times 100.
\end{equation}
%
En que $\bar{f}$ es la frecuencia media medida durante el tiempo $t_{\mathrm{op}}$ y $f_{\mathrm{spec}}$ es la frecuencia especificada bajo calibración.

\textbf{Distorsión armónica total más ruido (THD+N)}, la distorsión de la señal generada por el calibrador debe medirse con un ancho de banda de $\qty{22.4}{\Hz}$ a $\qty{22.4}{\kHz}$, como un promedio de los valores instantáneos obtenidos durante el tiempo $t_{\mathrm{op}}$, en los niveles máximo y mínimos disponibles a cada frecuencia de los que se indique en el manual que cumple con las especificaciones de la norma.
La THD+N puede ser medida utilizando un filtro de rechazo (medidor de factor de distorsión) o un analizador FFT. La THD+N medida no debe exceder los límites establecidos en la \mbox{IEC 60942} (\mbox{Tabla 7}~\citeyear{IEC_TC29_2017}), según la clase del calibrador.
Es obligatorio que la magnitud medida sea no sólo distorsión armónica total, sino distorsión armónica total \emph{más} ruido, reportada en porcentaje ($\%$).

\subsection{Descripción general de las pruebas periódicas seleccionadas de acuerdo con la \mbox{IEC 61672-3:2013}}
\label{subsec:slm_calibration_description}

\tikzmath{\x1 =1; \x2 = 3/5; \x3 = 3.1/6;}
\definecolor{block_blue}{HTML}{d4e1f5}
\begin{figure}[h]
    \caption{Diagrama de bloques del proceso de medición en la calibración periódica de sonómetros de acuerdo con la \mbox{IEC 61672-3:2013}.}
    \label{fig:slm_calibration_flowchart}
    \centering
    \begin{tikzpicture}[font=\scriptsize, minimum width=1cm, minimum height=1.5cm, align=center]
        \node (process1) at (-10, 0) [draw, process]{Recepción de \\ items de \\ calibración};
        \node (process2) [draw, process, right=\x1cm of process1]{Inspección \\ visual};
        \node (process3) [draw, process, right=\x1cm of process2,
            fill=block_blue]{Verificación \\ del calibrador \\ acústico};
        \node (process4) [draw, process, right=\x1cm of process3, fill=block_blue]{Verificación \\ de fuente de \\ alimentación};
        \node (process5) [draw, process, right=\x1cm of process4]{Ruido \\ intrínseco};
        \node (process6) [draw, process, right=\x1cm of process5]{Indicación a la \\ frecuencia de \\ comprobación de \\ la calibración};
        \node (process7) [draw, process, below=0.3cm of process6]{Una ponderación \\ frecuencial con \\ señales acústicas};
        \node (process8) [draw, process, left=\x2cm of process7, fill=block_blue]{Verificación \\ de fuente de \\ alimentación};
        \node (process9) [draw, process, left=\x2cm of process8, fill=block_blue]{Ponderaciones \\ frecuenciales y \\ temporales a $\qty{1}{\kHz}$};
        \node (process10) [draw, process, left=\x2cm of process9, fill=block_blue]{Ponderación \\ frecuencial con \\ señales eléctricas};
        \node (process11) [draw, process, left=\x2cm of process10, fill=block_blue]{Linealidad en el \\ rango de niveles \\ de referencia};
        \node (process12) [draw, process, below=0.3cm of process1]{Respuesta a \\ trenes de \\ onda};
        \node (process13) [draw, process, below=0.3cm of process12]{Nivel de \\ sonido $C$ \\ de pico};
        \node (process14) [draw, process, right=\x3cm of process13]{Indicación \\ de \\ sobrecarga};
        \node (process15) [draw, process, right=\x3cm of process14]{Estabilidad \\ a niveles \\ elevados};
        \node (process16) [draw, process, right=\x3cm of process15]{Estabilidad a \\ largo plazo};
        \node (process17) [draw, process, right=\x3cm of process16, fill=block_blue]{Verificación \\ de fuente de \\ alimentación};
        \node (process18) [draw, process, right=\x3cm of process17, fill=block_blue]{Estimación de \\ incertidumbre de \\ medición};
        \node (process19) [draw, process, below=0.3cm of process7]{Generación de \\ certificado de \\ calibración};

        \draw [arrow] (process1) -- (process2);
        \draw [arrow] (process2) -- (process3);
        \draw [arrow] (process3) -- (process4);
        \draw [arrow] (process4) -- (process5);
        \draw [arrow] (process5) -- (process6);
        \draw [arrow] (process6.east) -- +(0.3cm, 0cm) |- (process7);
        \draw [arrow] (process7) -- (process8);
        \draw [arrow] (process8) -- (process9);
        \draw [arrow] (process9) -- (process10);
        \draw [arrow] (process10) -- (process11);
        \draw [arrow] (process11) -- (process12);
        \draw [arrow] (process12.west) -- +(-0.3cm, 0cm) |- (process13);
        \draw [arrow] (process13) -- (process14);
        \draw [arrow] (process14) -- (process15);
        \draw [arrow] (process15) -- (process16);
        \draw [arrow] (process16) -- (process17);
        \draw [arrow] (process17) -- (process18);
        \draw [arrow] (process18) -- (process19);
    \end{tikzpicture}
    \caption*{\footnotesize Fuente: Elaboración propia.}
\end{figure}
%
En el diagrama de bloques de la figura~\ref{fig:slm_calibration_flowchart} se presenta de forma general el proceso de calibración periódica de sonómetros.
Los bloques resaltados en azul son los procesos objeto de automatización en este proyecto\footnote{Aún con la automatización de estas actividades, hay tareas que deben se realizadas manualmente por el usuario dadas las limitaciones físicas o de la naturaleza misma de los equipos. Por ejemplo, el usuario debe ajustar la orientación del calibrador acústico, encenderlo o apagarlo, configurar en el sonómetro los indicadores en pantalla, etc.}.
A continuación se describen en detalle las etapas en el proceso de calibración.

En principio, de conformidad con la \mbox{IEC 661672--3}~\citeyearpar{IEC_TC29_2013_3}, el sonómetro con todos sus accesorios necesarios (como preamplificador, micrófono, cable de extensión o adaptador de impedancia) debe ser entregado junto con el manual de instrucciones, si este es requerido por el laboratorio de calibración.
Toda la información necesaria para los ensayos periódicos debe estar disponible, como correcciones de campo libre, rangos de medición, niveles de referencia, etc.
Se debe contar con un calibrador acústico conforme con las especificaciones de la \mbox{IEC 60942}~\citeyearpar{IEC_TC29_2017} según su clase, ya sea suministrado por el cliente o por el laboratorio.

Luego, se hace una inspección preliminar del sonómetro y todos sus accesorios, verificando que todos los controles están funcionando, que la pantalla está en buen estado, que no haya acumulación de material extraño en la rejilla o membrana del micrófono y que otros elementos esenciales estén en un funcionamiento adecuado.
Después se verifica que la fuente de alimentación está operando dentro de los límites especificados en el manual de instrucciones.
La fuente de alimentación será verificada nuevamente después de los ensayos con señales acústicas y después de los ensayos con señales eléctricas.
A continuación, se detallan las pruebas que serán efectuadas.
Para todas las pruebas eléctricas se emplea el dispositivo de entrada (acoplador de impedancia) recomendado por el fabricante del sonómetro o uno que tenga una capacitancia similar que emule adecuadamente la carga del micrófono en el preamplificador.

\subsubsection{Indicación a la frecuencia de comprobación de la calibración}
El calibrador acústico entregado por el cliente o proporcionado por el laboratorio se acopla al micrófono del sonómetro, y, si es necesario, se ajusta el sonómetro para indicar el nivel de presión acústica requerido en las condiciones ambientales en las que se realizan los ensayos.
Las indicaciones antes y después del ajuste deben registrarse.
Se debe tomar en cuenta el efecto de la presión estática sobre el calibrador acústico empleado.
Este calibrador ya debió haber sido verificado previamente, según el procedimiento descrito en la sección~\ref{subsec:acoustic_calibrators_calibration_description}.

Después de haber ajustado el sonómetro en respuesta al nivel generado por el calibrador, un paso necesario (antes de continuar con las otras pruebas) es determinar el voltaje que produce una indicación del nivel de referencia.
La siguiente ecuación es empleada para determinar ese voltaje:
%
\begin{equation}
    \label{eq:reference_voltage}
    v_{\mathrm{ref}} = 10\,\hat{\mkern6mu}\left(\frac{L_{v,l}+\nicefrac{\left(L_{v,u} - L_{v,l}\right)}{2}}{20}\right).
\end{equation}
%
Donde $v_{\mathrm{ref}}$ es el voltaje medio en una escala logarítmica que produce una indicación del nivel de referencia, $L_{v,l}$ es el nivel de voltaje inferior del intervalo que produce una indicación del nivel de referencia, y $L_{v,u}$ el nivel de voltaje superior.
Los niveles de voltaje son referenciados a $\qty{1}{\V}$.
En concreto, este voltaje de referencia es el voltaje en la mitad (en una escala logarítmica) de un intervalo de voltajes que producen todos una misma indicación del nivel de referencia.
Con el voltaje de referencia se calculan los voltajes correspondientes a los niveles de señal en las demás pruebas.

\subsubsection{Ponderaciones frecuenciales y temporales a 1 kHz}
Se utiliza una señal eléctrica continua de $\qty{1}{\kHz}$ con una amplitud tal que produzca una indicación del nivel de referencia en el sonómetro $\left(\text{i.e. } v_{\mathrm{ref}}\right)$  y se siguen los pasos a continuación:
%
\begin{description}
    \item[PFT-1] Registrar el nivel indicado en las ponderaciones frecuenciales $A$, $C$ y $Z$ (según estén disponibles) con el sonómetro ajustado en ponderación temporal $F$ o nivel promediado en el tiempo\footnote{Como en este trabajo se busca reconocer el resultado instantáneo mostrado en pantalla, el nivel elegido preferiblemente es el que tiene ponderación temporal F.}; i.e. $L_{AF}, L_{CF}, L_{ZF}, L_{Aeq}, L_{Ceq}$ o $L_{Zeq}$.

    \item[PFT-2\label{itm:time_frequency_weight_registration}] Registrar el nivel indicado en las ponderaciones temporales F y S, y el nivel promediado en el tiempo\footnote{El nivel promediado en el tiempo sólo sería posible registrarlo automáticamente si este es mostrado en pantalla.} (según estén disponibles) con el sonómetro ajustado en ponderación frecuencial $A$; i.e. $L_{AF}, L_{AS}$ o $L_{Aeq}$.

    \item[PFT-3] Calcular las desviaciones de los niveles ponderados en frecuencia $C$ y $Z$ respecto al ponderado en frecuencia $A$ del paso 1.
    Estas desviaciones no deben superar los límites de aceptación de $\pm\qty{0.2}{\dB}$.

    \item[PFT-4] Calcular las desviaciones del nivel promediado en el tiempo y del nivel con ponderación temporal S, respecto al nivel con ponderación temporal F del paso~\ref{itm:time_frequency_weight_registration}.
    Estas desviaciones no deben superar los límites de aceptación de $\pm\qty{0.1}{\dB}$.
\end{description}

\subsubsection{Ponderaciones frecuenciales con señales eléctricas}

Se utilizan señales eléctricas sinusoidales continuas para todas las ponderaciones frecuenciales reguladas en la \mbox{IEC 61672--1}~\citeyearpar{IEC_TC29_2013_1} (que estén disponibles en el sonómetro).
Y se siguen los pasos a continuación:
%
\begin{description}
    \item[PFSE-1] Se ajusta el sonómetro para mostrar niveles de sonido con ponderación temporal F, niveles promediados en el tiempo o niveles de exposición sonora \footnote{Como en este trabajo se busca reconocer el resultado instantáneo mostrado en pantalla, el nivel elegido es preferiblemente el que tiene ponderación temporal F.}.

    \item[PFSE-2] Se ajusta el sonómetro en el rango de niveles de referencia y se envía una señal de $\qty{1}{\kHz}$ cuya amplitud produzca una indicación en el sonómetro que sea $\qty{45}{\dB}$ menos que el límite superior indicado en el manual de instrucciones para el rango de funcionamiento lineal a $\qty{1}{\kHz}$.

    Para automatizar este paso, se usa la siguiente ecuación:
%
    \begin{equation}
        \label{eq:voltage_1khz}
        v_{\qty{1}{\kHz}} = 10\,\hat{\mkern6mu}
        \left(\frac{20\,\log\left(v_{\mathrm{ref}}\right) + L_{\mathrm{ref}} - \left(L_{u@\qty{1}{\kHz}} - 45\right)}{20}\right).
    \end{equation}
%
    En que $v_{\qty{1}{\kHz}}$ es el voltaje que produce una indicación de $\qty{45}{\dB}$ menos que el límite superior del rango lineal a $\qty{1}{\kHz}$; $v_{\mathrm{ref}}$ es el voltaje que produce una indicación del nivel de referencia $L_{\mathrm{ref}}$; y, $L_{u@\qty{1}{\kHz}}$ es el límite superior del rango lineal a $\qty{1}{\kHz}$ especificado en el manual de instrucciones.

    \item[PFSE-3] Se registran los niveles de las señales de entrada y las correspondientes indicaciones.
    Para sonómetros clase 1 en las nueve frecuencias nominales en intervalos de octava de $\qty{63}{\Hz}$ a $\qty{16}{\kHz}$.
    Para sonómetros clase 2 en las ocho frecuencias nominales en intervalos de octava de $\qty{63}{\Hz}$ a $\qty{8}{\kHz}$.

    En frecuencias diferentes a $\qty{1}{\kHz}$ el voltaje de la señal de entrada se determina mediante
%
    \begin{equation}
        \label{eq:voltage_by_frequency}
        v_f = 10\,\hat{\mkern6mu}\left(\frac{20\,\log\left(v_{\qty{1}{\kHz}}\right) - W_{X,f}}{20}\right).
    \end{equation}
%
    En que $v_f$ es el voltaje de la señal a la frecuencia $f$, $v_{\qty{1}{\kHz}}$ es el voltaje del paso anterior y $W_{X,f}$ es el factor de la ponderación frecuencial elegida $X$ para la frecuencia $f$.

    \item[PFSE-4] Se calculan las ponderaciones frecuenciales relativas como $L_{f} - L_{\qty{1}{\kHz}}$, i.e.\ el nivel indicado a una frecuencia de ensayo menos el nivel indicado a $\qty{1}{\kHz}$.

    \item[PFSE-5] Aplicar factores de corrección a las ponderaciones frecuenciales relativas del paso anterior que den cuenta de:
%
    \begin{description}
        \item[PFSE-5.1] La desviación de la respuesta en frecuencia en campo libre o para incidencia aleatorio de un micrófono en la dirección de referencia respecto a una respuesta en frecuencia uniforme.
        \item[PFSE-5.2] Los efectos de las reflexiones en la carcasa del sonómetro y de la difracción del sonido alrededor del micrófono y del amplificador.
        \item[PFSE-5.3] Si aplica, la influencia de la pantalla antiviento y de cualquier accesorio que sea parte de la configuración normal del sonómetro.
    \end{description}

    \item[PFSE-6] Las ponderaciones frecuenciales relativas corregidas son las desviaciones respecto a los objetivos de diseño según la ponderación frecuencial bajo calibración y no deben exceder los límites de aceptación dados en la \mbox{IEC 61672--1} (\mbox{Tabla 3}~\citeyear{IEC_TC29_2013_1}).
\end{description}

\subsubsection{Linealidad de nivel en el rango de niveles de referencia}
Se utilizan señales eléctricas sinusoidales continuas a una frecuencia de $\qty{8}{\kHz}$ con el sonómetro ajustado en el rango de niveles de referencia, en ponderación frecuencial $A$, y con ponderación temporal F o un nivel promediado en el tiempo, i.e. $L_{AF}$ o $L_{Aeq}$, y se siguen los pasos a continuación:
%
\begin{description}
    \item[LNRR-1] Comenzar con una señal de entrada cuya amplitud produce el punto de partida para los ensayos de linealidad a $\qty{8}{\kHz}$ especificado en el manual de instrucciones.
    Y registrar el nivel indicado.

    \item[LNRR-2] Aumentar el nivel de la señal de entrada en saltos de $\qty{5}{\dB}$ desde el punto de partida hasta un nivel que se encuentre dentro de $\qty{5}{\dB}$ por debajo del límite superior del rango de funcionamiento lineal a $\qty{8}{\kHz}$ especificado en el manual.
    Luego, aumentar en saltos de $\qty{1}{\dB}$ hasta, pero sin incluir, la primera indicación de sobrecarga.
    Se deben registrar las indicaciones del sonómetro en cada punto.

    \item[LNRR-3] Disminuir el nivel de la señal de entrada en saltos de $\qty{5}{\dB}$ desde el punto de partida hasta un nivel que se encuentre dentro de $\qty{5}{\dB}$ por encima del límite inferior del rango de funcionamiento lineal a $\qty{8}{\kHz}$ especificado en el manual.
    Luego, disminuir en saltos de $\qty{1}{\dB}$ hasta, pero sin incluir, la primera indicación de ``por debajo del rango".
    Se deben registrar las indicaciones del sonómetro en cada punto.

    \item[LNRR-4] Calcular las desviaciones de nivel como la diferencia entre el nivel indicado y el nivel previsto.
    Estas desviaciones no deben superar los límites de $\pm\qty{0.8}{\dB}$ para la clase 1 o de $\pm\qty{1.1}{\dB}$ para la clase 2.
\end{description}

Para automatizar esta prueba, el voltaje en el punto de partida se determina como
%
\begin{equation}
    \label{eq:voltaje_start_point}
    v_{L_{\mathrm{start}}} = 10\,\hat{\mkern6mu}
    \left(\frac{20\,\log\left(v_{\mathrm{ref}}\right) +
    L_{\mathrm{start}} - L_{\mathrm{ref}} -
    W_{A,\qty{8}{\kHz}} - E_{A,\qty{8}{\kHz}}}{20}\right).
\end{equation}
%
En que $v_{L_{\mathrm{start}}}$ es el voltaje que causa una indicación del nivel en el punto de partida $L_{\mathrm{start}}$; $v_{\mathrm{ref}}$ es el voltaje que produce una indicación del nivel de referencia, $W_{A,\qty{8}{\kHz}}$ es el factor estandarizado de la ponderación $A$ en la frecuencia de $\qty{8}{\kHz}$ que tiene el valor de $\qty{-1.1}{\dB}$ y $E_{A,\qty{8}{\kHz}}$ es la ponderación relativa a $\qty{1}{\kHz}$ (sin corregir), obtenida en la prueba de ponderaciones frecuenciales en la frecuencia de $\qty{8}{\kHz}$ en la ponderación frecuencial $A$.%, y $R_{\mathrm{FF},\qty{8}{\kHz}}$ es la corrección de campo libre para la frecuencia de $\qty{8}{\kHz}$.

Luego, a partir de ese voltaje en el punto de partida, el voltaje $v_{L_{\mathrm{prev}}}$ para cada punto de calibración o nivel previsto $L_{\mathrm{prev}}$ a lo largo del rango de niveles, se calcula como:
%
\begin{equation}
    \label{eq:voltage_linearity}
    v_{L_{\mathrm{prev}}} = 10\,\hat{\mkern6mu}
    \left(\frac{20\,\log\left(v_{L_{\mathrm{start}}}\right) + L_{\mathrm{prev}} - L_{\mathrm{start}}}{20}\right).
\end{equation}


\section{Instrumentación}\label{sec:instrumentacion}

Los instrumentos presentados a continuación fueron elegidos teniendo en cuenta los requisitos metrológicos para obtener resultados confiables y trazables, como también asegurando que tengan las prestaciones mínimas para implementar el control remoto desde un ordenador.

\begin{figure}[!h]
    \caption{Esquema de conexiones de los instrumentos para la calibración periódica de calibradores acústicos.}
    \label{fig:IEC60942_connections}
    \begin{subfigure}[t]{0.59\textwidth}
        \centering
        \includegraphics[width=0.95\textwidth]{2_Metodología/Figs/IEC60942connections}
    \end{subfigure}
    \hfill
    \begin{subfigure}[t]{0.4\textwidth}
        \centering
        \begin{tikzpicture}
            \node [draw]{\vbox{\scriptsize{
                \begin{enumerate}
                    \item Multímetro digital Keysight 34461A
                    \item Convertidor BNC-Banana
                    \item Analizador de audio Keysight U8903B
                    \item Módulo de poder GRAS 12AK
                    \item Calibrador acústico Brüel \& Kjær 4231
                    \item Micrófono patrón de trabajo GRAS 40CE
                    \item Preamplificador 01dB PRE22
                    \item Computador
                    \item Interfaz GPIB Keysight 82357B
                \end{enumerate}}
            \textbf{\color{blue} —} Conexión GPIB \\
            \textbf{\color{orange} —} Conexión Lemo 7 \\
            \textbf{—} Conexión BNC}};
        \end{tikzpicture}
    \end{subfigure}
    \caption*{\footnotesize Fuente: Elaboración propia.}
\end{figure}

\subsection{Patrones e instrumentos para la calibración periódica de calibradores acústicos}

La interconexión propuesta de los instrumentos empleados en la calibración de calibradores acústicos se presenta en la figura~\ref{fig:IEC60942_connections}.
A continuación se describen las características de cada instrumento.

\begin{figure}[!h]
    \caption{Patrones acústicos para la calibración de calibradores acústicos.}
    \centering
    \begin{subfigure}[t]{0.49\textwidth}
        \centering
        \includegraphics[height=4cm]{2_Metodología/Figs/bruel4231}
        \caption{Calibrador acústico Brüel \& Kjær 4231 usado como patrón de laboratorio.}
        \label{fig:bruel_4231}
    \end{subfigure}
    \hfill
    \begin{subfigure}[t]{0.49\textwidth}
        \centering
        \includegraphics[height=4cm]{2_Metodología/Figs/gras40CE}
        \caption{Micrófono patrón de trabajo GRAS 40CE.}
        \label{fig:gras_40CE}
    \end{subfigure}
\end{figure}
%
\begin{figure}[!h]
    \caption{Instrumentos para adecuación de la señal eléctrica.}
    \centering
    \begin{subfigure}[t]{0.6\textwidth}
        \centering
        \includegraphics[height=7cm]{2_Metodología/Figs/PRE2201dB}
        \caption{Vista inferior del micrófono GRAS 40CE (izquierda abajo),
            preamplificador para micrófonos de $\nicefrac{1}{2}''$ 01dB PRE22 (medio)
            y adaptador de impedancia 01dB ADP12 con su terminal de aterrizaje (derecha).}
        \label{fig:PRE22_01dB}
    \end{subfigure}
    \hfill
    \begin{subfigure}[t]{0.38\textwidth}
        \centering
        \includegraphics[height=7cm]{2_Metodología/Figs/gras12AK}
        \caption{Módulo de poder para preamplificadores y micrófonos GRAS 12AK.}
        \label{fig:gras_12AK}
    \end{subfigure}
\end{figure}

Como bien indica la norma \mbox{IEC 60942}~\citeyearpar{IEC_TC29_2017}, uno de los posibles métodos para calibrar calibradores acústicos es por comparación contra un calibrador patrón.\ Ese es el método elegido en este trabajo.
El calibrador acústico patrón elegido debe ser de las especificaciones más altas posibles, y su elección también determina el alcance de calibración del laboratorio.
Para este trabajo se emplea el calibrador Brüel \& Kjær 4231 (ver figura~\ref{fig:bruel_4231}), el cual es de clase LS y tiene disponibles dos niveles de presión sonora ($\qtylist{94;114}{\dB}$) a $\qty{1}{\kHz}$.
La trazabilidad de este calibrador se mantiene directamente con el fabricante.

El método requiere también un micrófono de referencia con el cuál se pueda transformar la señal acústica en una señal eléctrica para que pueda ser analizada posteriormente en amplitud, frecuencia y distorsión armónica más ruido (THD+N).
Se empleó el micrófono GRAS 40CE (ver figura~\ref{fig:gras_40CE}), el cual es un micrófono de campo libre con una sensibilidad típica de $\num{40}\,\nicefrac{\unit{\mV}}{\unit{\Pa}}$ y cuenta con su certificado de calibración de fábrica, en el que es posible determinar la diferencia entre las respuestas de campo libre y de campo de presión a $\qty{1}{\kHz}$.

La señal eléctrica del micrófono debe ser adecuada antes de medirla, por lo que se usa un preamplificador 01dB PRE22 (ver figura~\ref{fig:PRE22_01dB}).
Para polarizar el preamplificador se usa un módulo de poder GRAS 12AK, el cual también puede darle mayor ganancia a la señal para mejorar la relación señal a ruido, aplicarle un filtro paso alto para eliminar la interferencia de baja frecuencia y permite hacer el acople de impedancias apropiado para conectar la señal a la entrada de los instrumentos de medición.
Este módulo se muestra en la figura~\ref{fig:gras_12AK}.

\begin{figure}[!h]
    \caption{Instrumentos de medición para la calibración periódica de calibradores acústicos.}
    \centering
    \begin{subfigure}[t]{0.45\textwidth}
        \centering
        \includegraphics[height=6cm]{2_Metodología/Figs/keysight34461A}
        \caption{Multímetro digital Keysight 34461A.}
        \label{fig:keysight_34461A}
    \end{subfigure}
    \hfill
    \begin{subfigure}[t]{0.45\textwidth}
        \centering
        \includegraphics[height=6cm]{2_Metodología/Figs/keysightU8903B}
        \caption{Analizador de audio Keysight U8903B.}
        \label{fig:keysight_U8903B}
    \end{subfigure}
\end{figure}
%
Para medir el voltaje AC y la frecuencia se requiere un multímetro digital con suficientes dígitos de resolución que aporten la precisión requerida y no afecte negativamente la incertidumbre de medición.
El multímetro empleado es el Keysight 34461A de $6\nicefrac{1}{2}$ dígitos (mostrado en la figura~\ref{fig:keysight_34461A}), que, aunque su impacto es proporcional con la escala de medición, en $\unit{\dB}$ resulta ser despreciable.
En $\unit{\Hz}$, debido a la escala lineal, tendrá un impacto mayor, pero sigue siendo despreciable.
Cabe resaltar que, dependiendo de la escala, no todos los dígitos están disponibles en la pantalla del multímetro, pero sí pueden adquirirse todos siempre por la interfaz remota.
Este multímetro mantiene su trazabilidad a los patrones nacionales.

En cuanto a la THD+N, podría usarse una interfaz de sonido con un software de análisis en frecuencia, pero establecer su trazabilidad metrológica es poco factible.
Se requiere un instrumento que igualmente pueda ser calibrado por un laboratorio acreditado para esta magnitud inusual.
Tal es el analizador de audio Keysight U8903B, que cuenta con una resolución de $7$ dígitos, disponibles completamente sólo mediante la interfaz remota.
Este instrumento cuenta con bastantes prestaciones especializadas para el análisis de audio.
Entre estas, unas importantes para los propósitos de este trabajo son: Establecer la impedancia de entrada, tipo de entrada (balanceada o no balanceada), desacople DC de la señal de entrada, filtros paso alto o paso bajo para la entrada, frecuencia de muestreo variable y cálculo de estadísticas en tiempo real.
Este analizador se muestra en la figura~\ref{fig:keysight_U8903B}.

Ambos instrumentos de Keysight cuentan con la interfaz de comunicación GPIB, mediante la cual estos pueden conectarse a un sólo puerto USB del computador para su control con instrucciones SCPI\@.

\subsection{Patrones e instrumentos para la calibración periódica de sonómetros}

\begin{figure}[!h]
    \caption{Esquema de conexiones de los instrumentos para la calibración periódica de sonómetros.}
    \label{fig:IEC61672_connections}
    \begin{subfigure}[t]{0.59\textwidth}
        \centering
        \includegraphics[width=0.95\textwidth]{2_Metodología/Figs/IEC61672-3connections}
    \end{subfigure}
    \hfill
    \begin{subfigure}[t]{0.4\textwidth}
        \centering
        \begin{tikzpicture}
            \node [draw]{\vbox{\scriptsize{
                \begin{enumerate}
                    \item Generador de señales Keysight 33511B
                    \item Atenuador programable ACOEM OUT-1694000
                    \item Adaptador de impedancia 01dB ADP12
                    \item Sonómetro
                    \item Cámara
                    \item Computador
                    \item Interfaz GPIB Keysight 82357B
                \end{enumerate}}
            \textbf{\color{blue} —} Conexión GPIB \\
            \textbf{—} Conexión BNC}};
        \end{tikzpicture}
    \end{subfigure}
    \caption*{\footnotesize Fuente: Elaboración propia.}
\end{figure}

La interconexión propuesta para la calibración de sonómetros se presenta en la figura~\ref{fig:IEC61672_connections}.
A continuación se describe cada instrumento.
%
\clearpage

\begin{figure}[!h]
    \caption{Instrumentos utilizados en la calibración periódica de sonómetros.}
    \centering
    \begin{subfigure}[t]{0.49\textwidth}
        \centering
        \includegraphics[height=6cm]{2_Metodología/Figs/keysight33511B}
        \caption{Generador de funciones arbitrarias Keysight 33511B.}
        \label{fig:keysight_33511B}
    \end{subfigure}
    \hfill
    \begin{subfigure}[t]{0.49\textwidth}
        \centering
        \includegraphics[height=6cm]{2_Metodología/Figs/decadebox}
        \caption{Atenuador programable ACOEM OUT-1694000.}
        \label{fig:decade_box}
    \end{subfigure}
\end{figure}
%
El flujo de señal inicia en el generador de señales.
En este trabajo se empleó el generador de funciones arbitrarias Keysight 33511B (véase figura~\ref{fig:keysight_33511B}), que tiene excelentes prestaciones como la definición personalizada de formas de onda y un desempeño superior dada su baja distorsión armónica (típicamente $\num{0.04}\%$), amplio ancho de banda e intervalo de voltaje, bajo efecto de \emph{jitter}, su filtro \emph{anti-alias}, su resolución de amplitud de $\qty{16}{\bit}$ y de frecuencia de $\qty{1}{\micro\Hz}$, respuesta en frecuencia plana ($\pm\qty{0.10}{\dB}$ en todo el rango inferior a $\qty{100}{\kHz}$), alta precisión en amplitud ($\pm1\%$ del valor establecido $\pm\qty{1}{\mV}$) y en frecuencia ($\pm\qty{2}{ppm}$ del valor establecido $\pm\qty{15}{\pico\Hz}$).
El generador 33511B también cuenta con la interfaz de comunicación GPIB para su control remoto.
Este generador cumple la función de patrón de medición, generando señales eléctricas para \emph{simular} los niveles de presión sonora, y mantiene su trazabilidad a los patrones nacionales.

A pesar de que el generador tiene un rango de voltaje AC amplio ($\qty{1}{\mVpp}$ a $\qty{10}{\Vpp}$), para la mayoría de aplicaciones en sonómetros (cuyos micrófonos tienen sensibilidades típicas entre $\qtyrange{40}{50}{\mV}$), con el fin de alcanzar los niveles más altos (cercanos a los $\qty{140}{\dB}$), se requieren voltajes del orden de $\qty{13}{\Vrms}$ aproximadamente, y para los niveles más bajos (cercanos a los $\qty{23}{\dB}$), se requieren voltajes del orden de $\qty{7}{\uVrms}$.
Por tal motivo se hace necesario un dispositivo adicional que amplifique la señal hasta al menos $\qty{6}{\dB}$ más y que sea capaz de atenuarla hasta al menos $\qty{50}{\dB}$.
Para esto se usó un \emph{Decade Box} o atenuador programable fabricado por ACOEM, el cual se muestra en la figura~\ref{fig:decade_box}.

Finalmente, el flujo de señal termina en el adaptador de impedancias.
Este debe ser preferiblemente el recomendado en el manual de instrucciones del sonómetro, o en su defecto debe usarse uno cuya capacitancia se equipare a la capacitancia del micrófono del sonómetro y que por supuesto tenga el mismo diámetro y rosca.
Como ejemplo, para sonómetros marca 01dB, se utiliza el adaptador 01dB ADP12, que se muestra en la figura~\ref{fig:PRE22_01dB}.

Adicionalmente, para capturar el valor de medición indicado en la pantalla del sonómetro, se usa una cámara web conectada al computador.

\subsection{Comandos SCPI}
\label{subsec:scpi_commands}
En esta sección se explica de forma general el protocolo de comandos estándar para instrumentos programables (SCPI), tomando como guía de referencia el manual del generador Keysight 33511B~\citeyearpar{Keysight2015}.

Los comandos SCPI son un lenguaje basado en ASCII para instrumentos de medición y de pruebas.
Estos comandos están basados en una estructura jerárquica conocida como sistema de árbol, en el que los comandos asociados están agrupados bajo un nodo o raíz común, formando así los subsistemas.
Por ejemplo, una parte de la estructura del sistema \texttt{\small OUTPut} es la siguiente:
%
\begin{Verbatim}[fontsize=\footnotesize]
OUTPut:
    SYNC {OFF|0|ON|1}
    SYNC:
        MODE {NORMal|CARRier}
        POLarity {NORMal|INVerted}
\end{Verbatim}

Las líneas sin sangría son los sistemas raíces y cada nivel de sangría corresponde al nivel del subsistema en la jerarquía.
El ``\textbf{:}" separa una palabra clave de otra en un nivel más bajo.
Las letras en mayúsculas son abreviaciones de las palabras clave de los comandos, que también pueden ser usadas opcionalmente, quitando las letras en minúsculas de la palabra clave.
Las llaves (\textbf{\{\}}), en realidad indican que se esperan los parámetros para un comando dado, los cuales pueden tomar valores numéricos, booleanos, cadenas de texto o palabras clave que denotan valores preestablecidos.
Si un comando espera más de un parámetro, estos son separados por comas.
Por ejemplo, para habilitar la salida del generador ajustándolo en una forma de onda senoidal de $\qty{1}{\kHz}$, con una amplitud de $\qty{50}{\mV}$ y un \emph{offset} DC de $\qty{0}{\V}$, se enviaría el siguiente comando: \texttt{\small APPL:SIN 1E3,50E-3,0.0}.

También se puede reducir la extensión de los códigos usando ``\textbf{;}" para separar instrucciones de un mismo sistema de mayor nivel.
Por ejemplo, la instrucción \texttt{\small TRIG:SOUR BUS; COUNT 30} logra el mismo efecto que las instrucciones \texttt{\small TRIG:SOUR BUS} y \texttt{\small TRIG:COUNT 30} enviadas una tras otra.

Se puede consultar el valor actual de la mayoría de parámetros de sistemas o subsistemas usando ``\textbf{?}". Por ejemplo, con el comando \texttt{\small :CALC:STAT:DATA1?} se obtiene el primer valor estadístico calculado por el analizador de audio, según se haya configurado previamente.

Cada instrumento de Keysight usado en este trabajo cuenta con un conjunto específico de sistemas y subsistemas para realizar las funciones particulares según su naturaleza, los cuales pueden ser consultados en las guías de referencia de comandos SCPI de los respectivos manuales o, de una forma mucho más interactiva, en la aplicación \emph{Command Expert} de Keysight.
También hay comandos comunes a la mayoría de instrumentos como los siguientes: \texttt{\small *IDN?} (consulta la información de identificación del instrumento) y \texttt{\small *TST?} (ejecuta la secuencia de autoverificación del instrumento).

El control remoto de los instrumentos con comandos SCPI se implementó en Python con la librería \texttt{\small PyVisa}~\citeyearpar{PyVisa2022}, la cual requiere que en el computador esté instalada por lo menos la librería VISA de \emph{National Instruments} o, como en este desarrollo, la \emph{Keysight IO Library Suite} \citepalias{Keysight2022}, las cuales son totalmente gratuitas.

\include{3_Reconocimiento/Reconocimiento}
%  ╦┌┬┐┌─┐┬  ┌─┐┌┬┐┌─┐┌┐┌┌┬┐┌─┐┌─┐┬┌─┐┌┐┌
%  ║│││├─┘│  ├┤ │││├┤ │││ │ ├─┤│  ││ ││││
%  ╩┴ ┴┴  ┴─┘└─┘┴ ┴└─┘┘└┘ ┴ ┴ ┴└─┘┴└─┘┘└┘

\chapter{Implementación de los procedimientos de calibración}

En este capítulo se describe el desarrollo y arquitectura de software de las aplicaciones diseñadas para la calibración periódica de calibradores acústicos de acuerdo con la \mbox{IEC 60942}~\citeyearpar{IEC_TC29_2017} (tomando como base un modelamiento en GRAFCET) y de sonómetros de acuerdo con la norma \mbox{IEC 61672--3}~\citeyearpar{IEC_TC29_2013_1} (usando el sistema de reconocimiento de caracteres del capítulo~\ref{ch:image_recognition}).
El desarrollo del \emph{software} está publicado en el siguiente repositorio de GitHub:

    {\footnotesize\url{https://github.com/jfBranch/unal-acoustic-metrology.git}}


\section{Automatización de la calibración periódica de calibradores acústicos}
\label{sec:acoustic_calibrators_automation}
Siguiendo el método normalizado descrito en la sección~\ref{subsec:acoustic_calibrators_calibration_description} y el algoritmo de la figura~\ref{fig:acoustic_calibrator_calibration_flowchart}, se modeló la secuencia de calibración en un gráfico de etapas y transiciones explicado en la siguiente sección.

\subsection{GRAFCET descriptivo del proceso}
Para dar mayor claridad, en la figura~\ref{fig:GRAFCET_principal} se presenta la secuencia principal en la que se hacen llamadas a subrutinas mediante las etapas macro.
Los GRAFCET de las subrutinas se muestran en las figuras~\ref{fig:GRAFCET_micsens} a la~\ref{fig:GRAFCET_noise_dut}.
\vfill
\clearpage

\begin{figure}[!hp]
    \caption{GRAFCET de la rutina principal de la calibración periódica de calibradores acústicos.}
    \label{fig:GRAFCET_principal}
    \centering
    \includegraphics[height=0.92\textheight]{4_Implementación/principalGRAFCET}
    \caption*{\footnotesize Fuente: Elaboración propia.}
\end{figure}
%
\begin{sidewaysfigure}[!hp]
    \caption{GRAFCET's de las subrutinas para la calibración periódica de calibradores acústicos.}
    \label{fig:GRAFCET_subrutines}
    \begin{subfigure}[t]{0.48\textwidth}
        \caption{Subrutina para la etapa macro $1$: Medición preeliminar de sensibilidad del micrófono.}
        \label{fig:GRAFCET_micsens}
        \centering
        \begin{tikzpicture}[font=\scriptsize]
            \begin{Encap}{0,0}{M1}{Mic Sensibility Measurement}
                \Etape[0,0]{E1}
                \ActionX{XE1}{\tiny $\texttt{CoupledRefCal} \defeq 1$}
                \ActionActiv{XE1}
                \Action{XE1}{\tiny \texttt{RefCalPower} $\defeq 1$}
                \ActionEvenement{XE1}{\tiny $\uparrow\texttt{CoupledRefCal}$}
                \TransitionRecept[VXE1]{E1}{\tiny $\texttt{RefCalPower} \cdot \texttt{CoupledRefCal} \cdot \texttt{Ok}$}
                \node(TE1') [below=0cm of VXE1]{}; \node(VTE1')[below=2em of TE1']{}; \draw (TE1') -- (VTE1');
                \DecaleNoeudx[11]{TE1'}{TE1r}
                \DecaleNoeudx[11]{VTE1}{VTE1r}
                \ConvOU{TE1'}{TE1r}{L20}
                \Etape[L20]{20}
                \ActionX{X20}{\tiny \texttt{MeasMicSens}}
                \ActionCond{X20}{\tiny $\qty{10}{\s}$/\texttt{X20}}
                \Action{X20}{\tiny \texttt{MeasFinished}}
                \ActionCond{X20}{\tiny $\qty{5}{\s}$/\texttt{MeasMicSens}}
                \DivOU{X20}{-4/L20a, 5/L20b}
                \Transition[L20a]{20a}
                \node[right=0.1 of L20a, align=left]{\tiny $\texttt{MeasFinished} \cdot$ \\ \tiny $[\qty{380}{\mV} \le \mathit{Sens} \le \qty{460}{\mV}]$};
                \Transition[L20b]{20b}
                \node [right=0.1 of L20b, align=left]
                {\tiny $\texttt{MeasFinished} \cdot$ \\ \tiny $([\qty{380}{\mV} \ge \mathit{Sens}] +$ \\ \tiny $[\mathit{Sens} \ge \qty{480}{\mV}])$};
                \Etape[VT20b]{21}
                \ActionX{X21}{\tiny \texttt{RepeatDialog}}
                \DivOU{X21}{-3/L21a,3/L21b}
                \TransitionRecept[L21a]{21a}{\tiny $\texttt{X21} \cdot \overline{\texttt{Ok}}$}
                \TransitionRecept[L21b]{21b}{\tiny $\texttt{X21} \cdot \texttt{Ok}$}
                \node (paso) [right=1em of TE1r]{};
                \Lien{T21b}{paso}{VTE1r}
                \Etape[VT21a]{22}
                \node[right=0.2 of X22, align=center]{\textit{``Suspende la} \\ \textit{secuencia principal''}};
                \DecaleNoeudy[8.8]{L20a}{nS1}
                \Etape[nS1]{S1}
                \draw (L20a) -- (XS1);
            \end{Encap}
        \end{tikzpicture}
        \caption*{\footnotesize Fuente: Elaboración propia.}
    \end{subfigure}
    \hfill
    \begin{subfigure}[t]{0.48\textwidth}
        \caption{Subrutina para la etapa macro $2$: Medición de ruido de fondo a $\qty{5}{\degree}$ con el calibrador patrón.}
        \label{fig:GRAFCET_standard_noise0}
        \centering
        \begin{tikzpicture}[font=\scriptsize]
            \begin{Encap}{0,0}{M2}{Background Noise Standard Calibrator}
                \Etape[0,0]{E2}
                \ActionX{XE2}{\tiny $\texttt{RefCalOrient} \defeq 0$}
                \ActionActiv{XE2}
                \Action{XE2}{\tiny \texttt{RefCalPower} $\defeq 0$}
                \ActionEvenement{XE2}{\tiny $\uparrow\texttt{RotatedRefCal}$}
                \TransitionRecept[VXE2]{E2}{\tiny $\overline{\texttt{RefCalPower}} \cdot \texttt{CoupledRefCal} \cdot \texttt{Ok}$}
                \node(TE2') [below=0cm of VXE2]{}; \node(VTE2')[below=2em of TE2']{}; \draw (TE2') -- (VTE2');
                \DecaleNoeudx[11]{TE2'}{TE2r}
                \DecaleNoeudx[11]{VTE2}{VTE2r}
                \ConvOU{TE2'}{TE2r}{L23}
                \Etape[L23]{23}
                \ActionX{X23}{\tiny \texttt{MeasBackNoise}}
                \ActionCond{X23}{\tiny $\qty{3}{\s}$/\texttt{X23}}
                \Action{X23}{\tiny \texttt{MeasFinished}}
                \ActionCond{X23}{\tiny $\qty{7}{\s}$/\texttt{MeasBackNoise}}
                \DivOU{X23}{-4/L23a, 4.5/L23b}
                \Transition[L23a]{23a}
                \node[right=0.1 of L23a, align=left]
                {\tiny $\texttt{MeasFinished} \cdot$ \\ \tiny $[\mathit{Noise} \le L_{\mathrm{spec}} - \qty{30}{\dB}]$};
                \Transition[L23b]{23b}
                \node [right=0.1 of L23b, align=left]
                {\tiny $\texttt{MeasFinished} \cdot$ \\ \tiny $[\mathit{Noise} \ge L_{\mathrm{spec}} - \qty{30}{\dB}]$};
                \Etape[VT23b]{24}
                \ActionX{X24}{\tiny \texttt{RepeatDialog}}
                \DivOU{X24}{-3/L24a,3/L24b}
                \TransitionRecept[L24a]{24a}{\tiny $\texttt{X24} \cdot \overline{\texttt{Ok}}$}
                \TransitionRecept[L24b]{24b}{\tiny $\texttt{X24} \cdot \texttt{Ok}$}
                \node (paso) [right=2em of TE2r]{};
                \Lien{T24b}{paso}{VTE2r}
                \Etape[VT24a]{25}
                \node[right=0.2 of X25, align=center]{\textit{``Suspende la} \\ \textit{secuencia principal''}};
                \DecaleNoeudy[8.8]{L23a}{nS2}
                \Etape[nS2]{S2}
                \draw (L23a) -- (XS2);
            \end{Encap}
        \end{tikzpicture}
        \caption*{\footnotesize Fuente: Elaboración propia.}
    \end{subfigure}
\end{sidewaysfigure}
%
\begin{sidewaysfigure}[!hp]
    \ContinuedFloat
    \caption{GRAFCET's de las subrutinas para la calibración periódica de calibradores acústicos (continuación).}
    \begin{subfigure}[t]{0.33\textwidth}
        \caption{Subrutina para las etapas macro $\numlist{3;6;9}$: Medición de magnitudes con el calibrador patrón.}
        \label{fig:GRAFCET_standard_quantities}
        \centering
        \begin{tikzpicture}[font=\scriptsize]
            \begin{Encap}{0,0}{M3 (M6 o M9)}{Quantities Measurement Standard Cal}
                \EtapeTransition[0,0]{E3}{\tiny $\texttt{RefCalPower} \defeq 1$}{\tiny $\texttt{RefCalPower} \cdot \texttt{Ok}$}
                \ActionActiv{XE3}
                \LienTE[3]{TE3}
                \EtapeTransition[VTE3]{26}{\tiny \texttt{MeasQuantities}}{\tiny \texttt{MeasFinished}}
                \ActionCond{X26}{\tiny $\qty{18}{\s}$/\texttt{X26}}
                \Action{X26}{\tiny \texttt{MeasFinished}}
                \ActionCond{X26}{\tiny $\qty{30}{\s}$/\texttt{MeasQuantities}}
                \Etape{S3}
            \end{Encap}
        \end{tikzpicture}
        \caption*{\footnotesize Fuente: Elaboración propia.}
    \end{subfigure}
    \hfill
    \begin{subfigure}[t]{0.55\textwidth}
        \caption{Subrutina para las etapas macro $\numlist{5; 8}$: Medición de ruido de fondo a $\qty{120}{\degree}$ y $\qty{240}{\degree}$ con el calibrador patrón.}
        \label{fig:GRAFCET_noise_standard}
        \centering
        \begin{tikzpicture}[font=\scriptsize]
            \begin{Encap}{0,0}{M5 (M8)}{Background Noise Standard Calibrator}
                \Etape[0,0]{E5}
                \ActionX{XE5}{\tiny $\texttt{RefCalOrient} \defeq \texttt{RefCalOrient} + 120$}
                \ActionActiv{XE5}
                \Action{XE5}{\tiny \texttt{CoupledRefCal} $\defeq 1$}
                \ActionEvenement{XE5}{\tiny $\uparrow\texttt{RotatedRefCal}$}
                \TransitionRecept[VXE5]{E5}{\tiny $\overline{\texttt{RefCalPower}} \cdot \texttt{CoupledRefCal} \cdot \texttt{Ok}$}
                \node(TE5') [below=0cm of VXE5]{}; \node(VTE5')[below=2em of TE5']{}; \draw (TE5') -- (VTE5');
                \DecaleNoeudx[11]{TE5'}{TE5r}
                \DecaleNoeudx[11]{VTE5}{VTE5r}
                \ConvOU{TE5'}{TE5r}{L27}
                \Etape[L27]{27}
                \ActionX{X27}{\tiny \texttt{MeasBackNoise}}
                \ActionCond{X27}{\tiny $\qty{3}{\s}$/\texttt{X27}}
                \Action{X27}{\tiny \texttt{MeasFinished}}
                \ActionCond{X27}{\tiny $\qty{7}{\s}$/\texttt{MeasBackNoise}}
                \DivOU{X27}{-4/L27a, 4.5/L27b}
                \Transition[L27a]{27a}
                \node[right=0.1 of L27a, align=left]
                {\tiny $\texttt{MeasFinished} \cdot$ \\ \tiny $[\mathit{Noise} \le L_{\mathrm{spec}} - \qty{30}{\dB}]$};
                \Transition[L27b]{27b}
                \node [right=0.1 of L27b, align=left]
                {\tiny $\texttt{MeasFinished} \cdot$ \\ \tiny $[\mathit{Noise} \ge L_{\mathrm{spec}} - \qty{30}{\dB}]$};
                \Etape[VT27b]{28}
                \ActionX{X28}{\tiny \texttt{RepeatDialog}}
                \DivOU{X28}{-3/L28a,3/L28b}
                \TransitionRecept[L28a]{28a}{\tiny $\texttt{X28} \cdot \overline{\texttt{Ok}}$}
                \TransitionRecept[L28b]{28b}{\tiny $\texttt{X28} \cdot \texttt{Ok}$}
                \node (paso) [right=2em of TE5r]{};
                \Lien{T28b}{paso}{VTE5r}
                \Etape[VT28a]{29}
                \node[right=0.2 of X29, align=center]{\textit{``Suspende la} \\ \textit{secuencia principal''}};
                \DecaleNoeudy[8.8]{L27a}{nS5}
                \Etape[nS5]{S5}
                \draw (L27a) -- (XS5);
            \end{Encap}
        \end{tikzpicture}
        \caption*{\footnotesize Fuente: Elaboración propia.}
    \end{subfigure}
\end{sidewaysfigure}
%
\begin{sidewaysfigure}[!hp]
    \ContinuedFloat
    \caption{GRAFCET's de las subrutinas para la calibración periódica de calibradores acústicos (continuación).}
    \begin{subfigure}[t]{0.48\textwidth}
        \caption{Subrutina para la etapa macro $11$: Medición de ruido de fondo a $\qty{0}{\degree}$ con el calibrador bajo prueba.}
        \label{fig:GRAFCET_noise0_dut}
        \centering
        \begin{tikzpicture}[font=\scriptsize]
            \begin{Encap}{0,0}{M11}{Background Noise Customer Calibrator}
                \Etape[0,0]{E11}
                \ActionX{XE11}{\tiny $\texttt{CusCalOrient} \defeq 0$}
                \ActionActiv{XE11}
                \Action{XE11}{\tiny \texttt{CusCalPower} $\defeq 0$}
                \ActionEvenement{XE11}{\tiny $\uparrow\texttt{RotatedRefCal}$}
                \TransitionRecept[VXE11]{E11}{\tiny $\overline{\texttt{CusCalPower}} \cdot \texttt{CoupledCusCal} \cdot \texttt{Ok}$}
                \node(TE11') [below=0cm of VXE11]{}; \node(VTE11')[below=2em of TE11']{}; \draw (TE11') -- (VTE11');
                \DecaleNoeudx[11]{TE11'}{TE11r}
                \DecaleNoeudx[11]{VTE11}{VTE11r}
                \ConvOU{TE11'}{TE11r}{L30}
                \Etape[L30]{30}
                \ActionX{X30}{\tiny \texttt{MeasBackNoise}}
                \ActionCond{X30}{\tiny $\qty{3}{\s}$/\texttt{X30}}
                \Action{X30}{\tiny \texttt{MeasFinished}}
                \ActionCond{X30}{\tiny $\qty{7}{\s}$/\texttt{MeasBackNoise}}
                \DivOU{X30}{-4/L30a, 4.5/L30b}
                \Transition[L30a]{23a}
                \node[right=0.1 of L30a, align=left]
                {\tiny $\texttt{MeasFinished} \cdot$ \\ \tiny $[\mathit{Noise} \le L_{\mathrm{spec}} - \qty{30}{\dB}]$};
                \Transition[L30b]{30b}
                \node [right=0.1 of L30b, align=left]
                {\tiny $\texttt{MeasFinished} \cdot$ \\ \tiny $[\mathit{Noise} \ge L_{\mathrm{spec}} - \qty{30}{\dB}]$};
                \Etape[VT30b]{31}
                \ActionX{X31}{\tiny \texttt{RepeatDialog}}
                \DivOU{X31}{-3/L31a,3/L31b}
                \TransitionRecept[L31a]{31a}{\tiny $\texttt{X31} \cdot \overline{\texttt{Ok}}$}
                \TransitionRecept[L31b]{31b}{\tiny $\texttt{X31} \cdot \texttt{Ok}$}
                \node (paso) [right=2em of TE11r]{};
                \Lien{T31b}{paso}{VTE11r}
                \Etape[VT31a]{32}
                \node[right=0.2 of X32, align=center]{\textit{``Suspende la} \\ \textit{secuencia principal''}};
                \DecaleNoeudy[8.8]{L30a}{nS11}
                \Etape[nS11]{S11}
                \draw (L30a) -- (XS11);
            \end{Encap}
        \end{tikzpicture}
        \caption*{\footnotesize Fuente: Elaboración propia.}
    \end{subfigure}
    \hfill
    \begin{subfigure}[t]{0.48\textwidth}
        \caption{Subrutina para las etapas macro $\numlist{14;17}$: Medición de ruido de fondo a $\qty{120}{\degree}$ y $\qty{240}{\degree}$ con el calibrador bajo prueba.}
        \label{fig:GRAFCET_noise_dut}
        \centering
        \begin{tikzpicture}[font=\scriptsize]
            \begin{Encap}{0,0}{M14 (M17)}{Background Noise Customer Calibrator}
                \Etape[0,0]{E14}
                \ActionX{XE14}{\tiny $\texttt{CusCalOrient} \defeq \texttt{CusCalOrient} + 120$}
                \ActionActiv{XE14}
                \Action{XE14}{\tiny \texttt{CoupledCusCal} $\defeq 1$}
                \ActionEvenement{XE14}{\tiny $\uparrow\texttt{RotatedCusCal}$}
                \TransitionRecept[VXE14]{E14}{\tiny $\overline{\texttt{RefCalPower}} \cdot \texttt{CoupledCusCal} \cdot \texttt{Ok}$}
                \node(TE14') [below=0cm of VXE14]{}; \node(VTE14')[below=2em of TE14']{}; \draw (TE14') -- (VTE14');
                \DecaleNoeudx[11]{TE14'}{TE14r}
                \DecaleNoeudx[11]{VTE14}{VTE14r}
                \ConvOU{TE14'}{TE14r}{L34}
                \Etape[L34]{34}
                \ActionX{X34}{\tiny \texttt{MeasBackNoise}}
                \ActionCond{X34}{\tiny $\qty{3}{\s}$/\texttt{X34}}
                \Action{X34}{\tiny \texttt{MeasFinished}}
                \ActionCond{X34}{\tiny $\qty{7}{\s}$/\texttt{MeasBackNoise}}
                \DivOU{X34}{-4/L34a, 4.5/L34b}
                \Transition[L34a]{34a}
                \node[right=0.1 of L34a, align=left]
                {\tiny $\texttt{MeasFinished} \cdot$ \\ \tiny $[\mathit{Noise} \le L_{\mathrm{spec}} - \qty{30}{\dB}]$};
                \Transition[L34b]{34b}
                \node [right=0.1 of L34b, align=left]
                {\tiny $\texttt{MeasFinished} \cdot$ \\ \tiny $[\mathit{Noise} \ge L_{\mathrm{spec}} - \qty{30}{\dB}]$};
                \Etape[VT34b]{35}
                \ActionX{X35}{\tiny \texttt{RepeatDialog}}
                \DivOU{X35}{-3/L35a,3/L35b}
                \TransitionRecept[L35a]{35a}{\tiny $\texttt{X35} \cdot \overline{\texttt{Ok}}$}
                \TransitionRecept[L35b]{35b}{\tiny $\texttt{X35} \cdot \texttt{Ok}$}
                \node (paso) [right=2em of TE14r]{};
                \Lien{T35b}{paso}{VTE14r}
                \Etape[VT35a]{36}
                \node[right=0.2 of X36, align=center]{\textit{``Suspende la} \\ \textit{secuencia principal''}};
                \DecaleNoeudy[8.8]{L34a}{nS14}
                \Etape[nS14]{S14}
                \draw (L34a) -- (XS14);
            \end{Encap}
        \end{tikzpicture}
        \caption*{\footnotesize Fuente: Elaboración propia.}
    \end{subfigure}
\end{sidewaysfigure}
%
\clearpage
\begin{figure}[!ht]
    \ContinuedFloat
    \caption{GRAFCET's de las subrutinas para la calibración periódica de calibradores acústicos (continuación).}
    \begin{subfigure}[t]{\textwidth}
        \caption{Subrutina para las etapas macro $\numlist{12;15;18}$: Medición de magnitudes con el calibrador bajo prueba.}
        \label{fig:GRAFCET_dut_quantities}
        \centering
        \begin{tikzpicture}[font=\scriptsize]
            \begin{Encap}{0,0}{M12 (M15 o M18)}{Quantities Measurement Customer Cal}
                \EtapeTransition[0,0]{E12}{\tiny $\texttt{CusCalPower} \defeq 1$}{\tiny $\texttt{CusCalPower} \cdot \texttt{Ok}$}
                \ActionActiv{XE12}
                \LienTE[3]{TE12}
                \EtapeTransition[VTE12]{33}{\tiny \texttt{MeasQuantities}}{\tiny \texttt{MeasFinished}}
                \ActionCond{X33}{\tiny $\qty{18}{\s}$/\texttt{X33}}
                \Action{X33}{\tiny \texttt{MeasFinished}}
                \ActionCond{X33}{\tiny $\qty{30}{\s}$/\texttt{MeasQuantities}}
                \Etape{S12}
            \end{Encap}
        \end{tikzpicture}
        \caption*{\footnotesize Fuente: Elaboración propia.}
    \end{subfigure}
\end{figure}

En todos los GRAFCET, además de los operandos de cada etapa ({\scriptsize \texttt{X\#}}), se emplean los del siguiente cuadro:
%
\begin{table}[!h]
    \caption{Descripción de los operandos del GRAFCET de la secuencia principal.}
    \label{tab:principal_GRAFCET_operands}
    \centering
    \scriptsize
    \begin{tabularx}{\textwidth}{c|X}
        \hline
        \multicolumn{1}{c|}{\textbf{Operando}} & \multicolumn{1}{c}{\textbf{Descripción}}                                        \\ \hline
        {\tiny\texttt{RefCalOrient}}           & Orientación del calibrador acústico patrón.                                     \\ \hline
        {\tiny\texttt{RotatedRefCal}} & {\tiny\texttt{True}} si el calibrador patrón ya fue rotado o
            {\tiny\texttt{False}} si aún no ha sido rotado. \\ \hline
        {\tiny\texttt{RefCalPower}} & Estado del calibrador acústico patrón.
            {\tiny\texttt{True}} significa encender y
            {\tiny\texttt{False}}  es apagar.
        Esta acción es realizada manualmente por el operador. \\ \hline
        {\tiny\texttt{CoupledRefCal}} & Estado de acoplamiento del calibrador acústico patrón.
            {\tiny\texttt{True}} significa acoplar el calibrador al micrófono,
            {\tiny\texttt{False}}  es desacoplar el calibrador.
        Esta acción es realizada manualmente por el operador. \\ \hline
        {\tiny\texttt{CusCalOrient}}           & Orientación del calibrador acústico bajo prueba.                                \\ \hline
        {\tiny\texttt{RotatedCusCal}} & {\tiny\texttt{True}} si el calibrador bajo prueba ya fue rotado o
            {\tiny\texttt{False}} si aún no ha sido rotado. \\ \hline
        {\tiny\texttt{CusCalPower}} & Estado del calibrador acústico bajo calibración.
            {\tiny\texttt{True}} significa encender y
            {\tiny\texttt{False}}  es apagar.
        Esta acción es realizada manualmente por el operador. \\ \hline
        {\tiny\texttt{CoupledCusCal}} & Estado de acoplamiento del calibrador acústico bajo calibración.
            {\tiny\texttt{True}} significa acoplar el calibrador al micrófono,
            {\tiny\texttt{False}}  es desacoplar el calibrador.
        Esta acción es realizada manualmente por el operador. \\ \hline
        {\tiny\texttt{RepeatDialog}}           & Mostrar ventana emergente de diálogo con los botones ``Aceptar'' y ``Abortar''. \\ \hline
        {\tiny\texttt{Ok}} & Respuesta del usuario a un mensaje de diálogo con los botones ``Aceptar'' ({\tiny\texttt{True}})
        o ``Abortar'' ({\tiny\texttt{False}}). \\ \hline
        {\tiny\texttt{MeasMicSens}}            & Medir sensibilidad.
        Esta acción es automática.                                  \\ \hline
        {\tiny\texttt{MeasFinished}}           & Señal que indica que la medición en curso ha finalizado.                        \\ \hline
        {\tiny\texttt{MeasBackNoise}}          & Medir ruido de fondo.
        Esta acción es automática.                                \\ \hline
        {\tiny\texttt{MeasQuantities}}         & Medir magnitudes.
        Esta acción es automática.
    \end{tabularx}
\end{table}

\vfill
\clearpage

\subsection{Implementación en Python}
Si bien los GRAFCET's son empleados principalmente en aplicaciones mecánicas o eléctricas, es un lenguaje común que tiene el potencial para usarse como base para el desarrollo de software, dada su simplicidad y practicidad \citepalias{MHJSoftware2020}; a lo que se suma la facilidad en aprenderlo, lo cual es una ventaja a la hora de integrar grupos de trabajo en los que participan personas de diferentes disciplinas.

\begin{figure}[!h]
    \centering
    \caption{Representación gráfica del paradigma \emph{Model-View-Controller}.}
    \label{fig:model_view_controller}
    \begin{tikzpicture}[thick, minimum width=2cm, minimum height=0.8cm]
        \node (model) at (0,0) [draw, process, fill=softOrange] {Modelo};
        \node (database) [below=1cm of model] {Base de datos};
        \node (databaseNortheast2) [above=0.15cm of database.north east,  inner sep=0pt, minimum size=0pt]{};
        \node (databaseNorthwest2) [above=0.15cm of database.north west,  inner sep=0pt, minimum size=0pt]{};
        \draw (database.north west) -- (database.south west)
        .. controls +(-30:0.2cm) and +(-150:0.2cm) .. (database.south east)
        -- (database.north east) .. controls +(-150:0.2cm) and +(-30: 0.2cm) .. (database.north west)
        -- (databaseNorthwest2.center) .. controls +(-30:0.2cm) and +(-150:0.2cm) .. (databaseNortheast2.center) -- (database.north east);
        \draw (databaseNorthwest2.center) .. controls +(30:0.2cm) and +(150:0.2cm) .. (databaseNortheast2.center);
        \node (database_north1) [above=0.2cm of database.north, inner sep=0pt, minimum size=0pt]{};
        \node (controller) [draw, process, fill=blue_sky, right=1cm of model]{Controlador};
        \node (view) [draw, process, fill=soft_red, right=1cm of controller]{Vista};
        \node (display) [draw, process, below=1cm of view]{};
        \node (displaySouthwest1) [below left=0.2cm and 0.2cm of display.south west, inner sep=0pt, minimum size=0pt]{};
        \node (displaySoutheast1) [below right=0.2cm and 0.2cm of display.south east, inner sep=0pt, minimum size=0pt]{};
        \draw (display.south west) -- (displaySouthwest1.center) -- (displaySoutheast1.center) -- (display.south east);
        \draw [-stealth] (database_north1) -- (model);
        \draw [stealth-stealth] (model) -- (controller);
        \draw [stealth-stealth] (controller) -- (view);
        \draw [stealth-stealth] (view) -- (display);
    \end{tikzpicture}
    \caption*{\footnotesize Fuente: Elaboración propia.}
\end{figure}
%
Normalmente, el diseño de un GRAFCET es llevado a la realidad en los sistemas empleando controladores lógicos programables (PLC), a los que están conectados los sensores y actuadores presentes en el sistema.
Sin embargo, en este desarrollo (siguiendo el paradigma de la programación orientada a objetos) la función del PLC es desempeñada \emph{virtualmente} por dos objetos cuyas clases están diseñadas para ser el controlador y el modelo en el patrón \emph{Model-View-Controller} (MVC), comúnmente utilizado en el desarrollo de aplicaciones web con interfaz gráfica (véase la figura~\ref{fig:model_view_controller}).
Para crear un entorno en el que estas interacciones virtuales puedan ocurrir, las clases de los objetos fueron diseñadas heredando de las clases {\small \texttt{QObject}} y {\small \texttt{QThread}} de la librería {\small \texttt{PyQt5}}, que permiten emplear la tecnología multi-hilos.
De manera que las señales y acciones ocurren en hilos paralelos según sea adecuado, sin \emph{congelar} el funcionamiento de la interfaz gráfica y permitiendo también el procesamiento en ``segundo plano''.
Las señales conectadas a los \emph{slots} son valores de progreso de la medición general o de la prueba en ejecución, y los valores de medición obtenidos en tiempo real.

La principal ventaja que se obtiene al usar un GRAFCET como base en el desarrollo de un \emph{software} que automatice las rutinas de un sistema, es que al final la programación queda encapsulada, ordenada e incluso etiquetada según las etapas del GRAFCET. De modo que esto facilita el diseño de la arquitectura de \emph{software} y permite que en la práctica se pueda ir a etapas específicas reutilizando las subrutinas, lo cual hace de esta una aplicación versátil e intuitiva para la calibración de calibradores acústicos.

\vfill

\begin{figure}[!hp]
    \caption{Diagrama de clases de la aplicación desarrollada para la calibración periódica de calibradores acústicos.}
    \label{fig:uml_acoustic_calibrators}
    \centering
    \includegraphics[width=\textwidth]{4_Implementación/IEC60942uml}
    \caption*{\footnotesize Fuente: Elaboración propia.}
\end{figure}

\subsubsection{Arquitectura de software}
%
El diagrama de clases de la figura~\ref{fig:uml_acoustic_calibrators} es un resumen de las clases principales (con sus atributos y métodos) diseñadas para el desarrollo de la aplicación para la calibración de calibradores acústicos.
También se presentan las clases heredadas y el instanciamiento entre clases.

La escritura del código se hizo aplicando las buenas prácticas de programación como: Documentación de clases, métodos e instrucciones relevantes, uso de atributos o métodos protegidos y las directrices de la guía PEP 8.
La interfaz gráfica se diseñó en Qt Designer de tal forma que se logre un manejo intuitivo de las funciones de la aplicación usando diferentes recursos como íconos, barras de progreso, menús desplegables, barra de herramientas, etc.
Con Qt Designer fue posible generar el código base de Python para el \emph{view} que lanza la aplicación y muestra la interfaz tal como fue diseñada.

\subsubsection{Descripción de funcionamiento}
\begin{figure}[!hb]
    \centering
    \caption{Interfaz gráfica de usuario de la aplicación para calibradores acústicos. Se muestra la pestaña de \emph{Patrones}.}
    \label{fig:calibrator_gui_standards}
    \includegraphics[width=0.8\textwidth]{4_Implementación/Figs/calibrator_gui_standards}
    \caption*{\footnotesize Fuente: Elaboración propia.}
\end{figure}
%
En la figura~\ref{fig:calibrator_gui_standards} se muestra la interfaz de usuario diseñada.
La pestaña de \emph{Patrones} es la mostrada en la vista inicial.
En esta, el usuario ingresa la información básica de los patrones empleados en la calibración.
Para el multímetro y el analizador de audio, si el usuario digita información en los campos de modelo, se habilita la herramienta \includegraphics[height=12pt]{4_Implementación/Figs/searchInstruments}, la cual busca automáticamente los modelos indicados entre todos los equipos disponibles conectados por GPIB al computador, extrae la información de estos y rellena los campos faltantes.

Una vez la información de los patrones está completa, el usuario puede hacer clic en \emph{Guardar}.
Se habilita la herramienta \includegraphics[height=12pt]{4_Implementación/Figs/test}, con la que se ejecuta la secuencia de auto-verificación de los patrones conectados por GPIB, si está disponible.
El resultado de la verificación se muestra en el \emph{check box} correspondiente.

\begin{figure}[!h]
    \centering
    \caption{Interfaz gráfica de usuario de la aplicación para calibradores acústicos. Se muestra la pestaña de \emph{Información IBC}.}
    \label{fig:calibrator_gui_dut}
    \includegraphics[width=0.8\textwidth]{4_Implementación/Figs/calibrator_gui_dut}
    \caption*{\footnotesize Fuente: Elaboración propia.}
\end{figure}

A continuación, en la pestaña \emph{Información IBC} (ver figura~\ref{fig:calibrator_gui_dut}), el usuario ingresa toda la información del calibrador bajo verificación y del cliente, necesaria para la calibración y para el certificado de calibración.
Cuando la información esté completa, el usuario puede hacer clic en \emph{Guardar} y, si el resultado de auto-verificación de los patrones fue correcto, entonces se habilita el botón \includegraphics[height=12pt]{4_Implementación/Figs/play}.
Al hacer clic en este se cumple la condición para la transición desde la etapa $0$ a la $1$ de la rutina principal del GRAFCET (figura~\ref{fig:GRAFCET_principal}).
En seguida se llevan a cabo todas las acciones de las demás etapas y, en la medida que avanza la secuencia, se van mostrando instrucciones al usuario para las acciones manuales y los resultados se presentan en tiempo real en la pestaña \emph{Resultados} (ver figura~\ref{fig:calibrator_gui_results}).

\begin{figure}[!h]
    \centering
    \caption{Interfaz gráfica de usuario de la aplicación para calibradores acústicos. Se muestra la pestaña de \emph{Resultados}.}
    \label{fig:calibrator_gui_results}
    \includegraphics[width=0.8\textwidth]{4_implementación/Figs/calibrator_gui_results}
    \caption*{\footnotesize Fuente: Elaboración propia.}
\end{figure}

En la parte inferior de la ventana se incluye una barra de estado que indica la etapa actual (que tiene el \emph{token}), y dos barras de progreso, una para el progreso general de la calibración y otra para la medición en la etapa actual.
En cualquier momento de la calibración se puede hacer clic en \includegraphics[height=12pt]{4_Implementación/Figs/pause} para suspender temporalmente la secuencia y luego reanudarla haciendo clic nuevamente en \includegraphics[height=12pt]{4_Implementación/Figs/play}.
También se incluyeron en la interfaz otros botones y herramientas proyectando la aplicación a un desarrollo posterior que permita abrir y guardar sesiones de calibración, avanzar, retroceder etapas o ir a alguna específica del GRAFCET, y hasta generar automáticamente el certificado de calibración.


\section{Automatización de la calibración periódica de sonómetros}

De acuerdo con el método descrito en la sección~\ref{subsec:slm_calibration_description} y el diagrama de bloques de la figura~\ref{fig:slm_calibration_flowchart}, la aplicación se desarrolló con una metodología similar a la implementada para calibradores acústicos, como se explica en la siguiente sección.

\subsection{Implementación en Python}
El funcionamiento de la aplicación para calibradores acústicos asentó las bases para diseñar la aplicación para sonómetros.
Para aprovechar esas interacciones de los paradigmas \emph{model-view-controller} y GRAFCET que ocurren en el ambiente multihilos, se establecieron los siguientes pasos para la automatización de las pruebas realizadas en sonómetros;
estos pasos serían análogos a las etapas de un GRAFCET\@.

\begin{algorithm}[H]
    \caption{Pasos para la calibración periódica de sonómetros implementados en la aplicación desarrollada.}
    \label{alg:slm_calibration_steps}
    \scriptsize
    \DontPrintSemicolon
    \textbf{Paso 0.} Inicio, entrenar clasificador bayesiano. \;
    \textbf{Paso 1.} Verificar fuente de alimentación.\;
    \textbf{Paso 2.} Realizar prueba de indicación a la frecuencia de comprobación de la calibración.\;
    \textbf{Paso 3.} Verificar fuente de alimentación.\;
    \textbf{Paso 4.} Determinar el voltaje que produce una indicación del nivel de referencia.\;
    \textbf{Paso 5.} Realizar prueba de ponderación frecuencial $A$ con señales eléctricas.\;
    \textbf{Paso 6.} Realizar prueba de ponderación frecuencial $C$ con señales eléctricas.\;
    \textbf{Paso 7.} Realizar prueba de ponderación frecuencial $Z$ con señales eléctricas.\;
    \textbf{Paso 8.} Realizar prueba de ponderaciones frecuenciales a $\qty{1}{\kHz}$.\;
    \textbf{Paso 9.} Realizar prueba de ponderaciones temporales a $\qty{1}{\kHz}$.\;
    \textbf{Paso 10.} Realizar prueba de linealidad en el rango de niveles de referencia.\;
    \textbf{Paso 11.} Verificar fuente de alimentación.\;
    \textbf{Paso 12.} Fin\;
\end{algorithm}

%\begin{description}
%\item[Paso 0.] Inicio
%\item[Paso 1.] Verificar fuente de alimentación.
%\item[Paso 2.] Realizar prueba de indicación a la frecuencia de comprobación de la calibración.
%\item[Paso 3.] Verificar fuente de alimentación.
%\item[Paso 4.] Determinar el voltaje que produce una indicación del nivel de referencia.
%\item[Paso 5.] Realizar prueba de ponderación frecuencial $A$ con señales eléctricas.
%\item[Paso 6.] Realizar prueba de ponderación frecuencial $C$ con señales eléctricas.
%\item[Paso 7.] Realizar prueba de ponderación frecuencial $Z$ con señales eléctricas.
%\item[Paso 8.] Realizar prueba de ponderaciones frecuenciales a $\qty{1}{\kHz}$.
%\item[Paso 9.] Realizar prueba de ponderaciones temporales a $\qty{1}{\kHz}$.
%\item[Paso 10.] Realizar prueba de linealidad en el rango de niveles de referencia.
%\item[Paso 11.] Verificar fuente de alimentación.
%\item[Paso 12.] Fin
%\end{description}

Además de los hilos en los que el modelo hace sus operaciones y la vista envía y recibe las instrucciones del controlador, para esta aplicación fueron necesarios dos hilos más: uno que captura y muestra las imágenes del objeto de vídeo y otro que en segundo plano procesa, reconoce y guarda en disco en formato binario las imágenes de cada punto de calibración.
Todo esto a la vez que el modelo está corriendo temporizadores, enviando instrucciones a los instrumentos patrones y haciendo cálculos.
También se agregan otras señales Qt para indicar el arranque y parada de temporizadores, entregar el cuadro de vídeo capturado, indicar la liberación del objeto de vídeo y la finalización de guardado del vídeo, y enviar los mensajes de \emph{logging}.

Asignar los procesos de cálculo de los voltajes de prueba y temporizadores que se realizan en el modelo, y los procesos de reconocimiento de imágenes a hilos separados es un beneficio para el mantenimiento del \emph{software} y el desempeño de la aplicación.
Por ejemplo, el último objetivo de este proyecto, que consiste en el modelamiento de las cadenas de Markov, se puede implementar como un proceso adicional en el hilo de reconocimiento de imágenes, sin afectar los otros procesos.

\vfill

\begin{figure}[!hp]
    \caption{Diagrama de clases de la aplicación desarrollada para la calibración periódica de sonómetros.}
    \label{fig:uml_sonometers}
    \centering
    \includegraphics[width=\textwidth]{4_Implementación/IEC61672uml}
    \caption*{\footnotesize Fuente: Elaboración propia.}
\end{figure}
%
\begin{figure}[!h]
    \ContinuedFloat
    \centering
    \caption{Diagrama de clases de la aplicación desarrollada para la calibración periódica de sonómetros. (Continuación)}
    \begin{tikzpicture}[font=\tiny\ttfamily]
        \begin{class}[text width=1.6cm]{QMainWindow}{-1,0.9}\end{class}

        \begin{class}[text width=8cm]{SoundLevelMetersUI}{-1,0}
            \inherit{QMainWindow}
            \attribute{\textit{\guillemotleft some instance attributes corresponding to the GUI elements\guillemotright}}
            \operation{\textit{\guillemotleft a few instance methods for configuring the initial state of the GUI elements\guillemotright}}
        \end{class}

        \begin{class}[text width=4cm]{GUIController}{8,0}
            \attribute{\textit{\guillemotleft same attributes\guillemotright}}
            \operation{\textit{\guillemotleft same methods\guillemotright}}
        \end{class}

        \node (ACUINorth1)[below=0.2 of SoundLevelMetersUI.north east, inner sep=0pt, minimum size=0pt]{};
        \node (GUICWest4)[right=2.6 of ACUINorth1, inner sep=0pt, minimum size=0pt]{};
        \draw [umlcd style dashed line, <-] (ACUINorth1) -- node[pos=0.5,above,sloped]{\guillemotleft instantiate\guillemotright} (GUICWest4);

        \begin{class}[text width=2cm]{QObject}{-3,-4.5}\end{class}

        \begin{class}[text width=6cm]{VideoObject}{2, -2}
            \inherit{QObject}
            \attribute{+ frameCaptured: pyqtSignal \textit{\guillemotleft class attribute\guillemotright}}
            \attribute{+ cameraReleased: pyqtSignal \textit{\guillemotleft class attribute\guillemotright}}
            \attribute{+ loggingMsg: pyqtSignal \textit{\guillemotleft class attribute\guillemotright}}
            \attribute{+ run\_flag: bool \textit{\guillemotleft instance attribute\guillemotright}}
            \attribute{+ fps: int \textit{\guillemotleft instance attribute\guillemotright}}
            \attribute{+ frame\_size: tuple \textit{\guillemotleft instance attribute\guillemotright}}

            \operation{+ run() ->\,None \textit{\guillemotleft instance method\guillemotright}}
        \end{class}

        \begin{class}[text width=6cm]{VideoSaving}{2, -4.9}
            \inherit{QObject}
            \attribute{+ videoSaved: pyqtSignal \textit{\guillemotleft class attribute\guillemotright}}
            \attribute{+ loggingMsg: pyqtSignal \textit{\guillemotleft class attribute\guillemotright}}
            \attribute{+ frames: list \textit{\guillemotleft instance attribute\guillemotright}}
            \attribute{+ video\_writer: cv.VideoWriter \textit{\guillemotleft instance attribute\guillemotright}}

            \operation{+ save\_video() ->\,None \textit{\guillemotleft instance method\guillemotright}}
        \end{class}

        \begin{class}[text width=2cm]{QThread}{-3,-9}\end{class}

        \begin{class}[text width=6cm]{ImageProcessingThread}{2, -7.5}
            \inherit{QThread}
            \attribute{+ realTimeValues: pyqtSignal \textit{\guillemotleft class attribute\guillemotright}}
            \attribute{+ loggingMsg: pyqtSignal \textit{\guillemotleft class attribute\guillemotright}}
            \attribute{\# frames\_queue: Queue \textit{\guillemotleft instance attribute\guillemotright}}
            \attribute{\# TESTER: SLMPeriodicTester \textit{\guillemotleft instance attribute\guillemotright}}
            \attribute{\# stage: int \textit{\guillemotleft instance attribute\guillemotright}}
            \attribute{\# current\_f: int \textit{\guillemotleft instance attribute\guillemotright}}
            \attribute{\# fweightings: list \textit{\guillemotleft instance attribute\guillemotright}}
            \attribute{\# octave\_frequencies: list \textit{\guillemotleft instance attribute\guillemotright}}
            \attribute{+ fps: int \textit{\guillemotleft instance attribute\guillemotright}}

            \operation{+ run() ->\,None \textit{\guillemotleft instance method\guillemotright}}
        \end{class}

        \draw [umlcd style dashed line, <-](VideoObject.east) -| node[pos=0.3,above,sloped]{\guillemotleft instantiate\guillemotright} (GUIController.south);
        \draw [umlcd style dashed line, <-](VideoSaving.east) -| node[pos=0.3,above,sloped]{\guillemotleft instantiate\guillemotright} (GUIController.south);
        \draw [umlcd style dashed line, <-](ImageProcessingThread.east) -| node[pos=0.3,above,sloped]{\guillemotleft instantiate\guillemotright} (GUIController.south);
    \end{tikzpicture}
    \caption*{\footnotesize Fuente: Elaboración propia.}
\end{figure}

\subsubsection{Arquitectura de software}
El diagrama de clases de la figura~\ref{fig:uml_sonometers} es un resumen de las clases principales (con sus atributos y métodos) diseñadas para el desarrollo de la aplicación para la calibración de sonómetros.
También se presentan las clases heredadas y el instanciamiento entre clases.

La escritura del código se hizo aplicando las buenas prácticas de programación como: Documentación de clases, métodos e instrucciones relevantes, uso de atributos o métodos protegidos, aseguramiento de recursos para evitar condiciones de carrera entre hilos, colas para el procesamiento en segundo plano y las directrices de la guía PEP 8.
La interfaz gráfica se diseñó en Qt Designer de tal forma que se logre un manejo intuitivo de las funciones de la aplicación usando diferentes recursos como íconos, barras de progreso, menús desplegables, barra de herramientas, etc.
Con Qt Designer fue posible generar el código base de Python para el \emph{view} que lanza la aplicación y muestra la interfaz tal como fue diseñada.

\subsubsection{Descripción de funcionamiento}

\begin{figure}[!h]
    \centering
    \caption{Interfaz gráfica de usuario de la aplicación para sonómetros. Se muestra la pestaña de \emph{Patrones}.}
    \label{fig:slm_gui_standards}
    \includegraphics[width=\textwidth]{4_implementación/Figs/slm_gui_standards}
    \caption*{\footnotesize Fuente: Elaboración propia.}
\end{figure}

La aplicación para la calibración de sonómetros tiene un flujo de trabajo similar a la de calibradores acústicos.
En la figura~\ref{fig:slm_gui_standards} se muestra la interfaz gráfica de usuario diseñada.
En la vista inicial, la pestaña de \emph{Patrones} es la primera en mostrarse.
En esta, el usuario ingresa la información básica de los patrones empleados en la calibración.
Para el generador de señales y el multímetro digital, si el usuario digita información en los campos de modelo, se habilita la herramienta \includegraphics[height=12pt]{4_Implementación/Figs/searchInstruments}, la cual busca automáticamente los modelos indicados entre todos los equipos disponibles conectados por GPIB al computador, extrae la información de estos y rellena los campos faltantes.

Una vez la información de los patrones está completa, el usuario puede hacer clic en \emph{Guardar}.
Se habilita la herramienta \includegraphics[height=12pt]{4_Implementación/Figs/test}, con la que se ejecuta la secuencia de auto-verificación de los patrones conectados por GPIB, si está disponible.
El resultado de la verificación se muestra en el \emph{check box} correspondiente.

Dado que en esta aplicación se realizan más tareas y hay mayor interacción de los hilos de programación, fue conveniente incluir un {\footnotesize \texttt{QListWidget}} en el que se indexan registros de eventos con su correspondiente marca de tiempo, usando un código de colores (negro: mensaje informativo, verde: acción importante realizada y rojo: error o acción finalizada incorrectamente).
Este registro de eventos está siempre visible debajo del recuadro al que se transmiten los cuadros de vídeo.

\begin{figure}[!h]
    \centering
    \caption{Interfaz gráfica de usuario de la aplicación para sonómetros. Se muestra la pestaña de \emph{Información del IBC}.}
    \label{fig:slm_gui_dut}
    \includegraphics[width=\textwidth]{4_implementación/Figs/slm_gui_dut}
    \caption*{\footnotesize Fuente: Elaboración propia.}
\end{figure}

A continuación, en la pestaña \emph{Información IBC} (véase figura~\ref{fig:slm_gui_dut}), el usuario ingresa toda la información del sonómetro bajo verificación y del cliente, necesaria para la calibración y para el certificado de calibración.
Cuando la información esté completa, el usuario puede hacer clic en \emph{Guardar}.
Según la configuración indicada para el sonómetro, el programa buscará en una base de datos los factores de corrección de campo libre;
si estos no están disponibles, entonces el usuario puede crear un archivo separado por comas con los factores de corrección y cargarlos manualmente.
Luego, si el resultado de auto-verificación de los patrones fue correcto, entonces se habilita el botón \includegraphics[height=12pt]{4_Implementación/Figs/play}.
Al hacer clic en este se ejecuta el paso $0$ del algoritmo~\ref{alg:slm_calibration_steps}.
En seguida se llevan a cabo todas las acciones de los demás pasos y, en la medida que avanza la secuencia, se van mostrando instrucciones al usuario para las acciones manuales (ver figura~\ref{fig:slm_gui_results1}).

\vfill
\clearpage

\begin{figure}[!h]
    \centering
    \caption{Interfaz gráfica de usuario de la aplicación para sonómetros. Se muestra la pestaña de \emph{Pruebas preliminares} y un cuadro de diálogo con una instrucción.}
    \label{fig:slm_gui_results1}
    \includegraphics[width=0.9\textwidth]{4_implementación/Figs/slm_gui_results1}
    \caption*{\footnotesize Fuente: Elaboración propia.}
\end{figure}

\begin{figure}[!h]
    \centering
    \caption{Interfaz gráfica de usuario de la aplicación para sonómetros. Se muestra la pestaña de \emph{Pruebas preliminares} y la selección del área de interés sobre el vídeo.}
    \label{fig:slm_gui_results2}
    \includegraphics[width=0.9\textwidth]{4_implementación/Figs/slm_gui_results2}
    \caption*{\footnotesize Fuente: Elaboración propia.}
\end{figure}

\vfill
\clearpage

Al llegar al paso 5 (ponderación frecuencial $A$ con señales eléctricas), si no se ha activado la transmisión de vídeo, entonces esta iniciará y el usuario puede seleccionar el área de interés donde se encuentra el valor de medición que se desea reconocer (véase figura~\ref{fig:slm_gui_results2}).
Igualmente con las siguientes pruebas hasta el paso 10.
En la medida que avanzan las pruebas, los voltajes y frecuencias de las señales eléctricas se van ajustando automáticamente y se muestran en el historial los comandos SCPI enviados;
los resultados se presentan en tiempo real en sus campos correspondientes según la prueba en curso como se muestra en la siguiente figura.

\begin{figure}[!h]
    \centering
    \caption{Interfaz gráfica de usuario de la aplicación para sonómetros. Se muestra la pestaña de \emph{Ponderaciones frecuenciales y temporales} y la presentación de resultados.}
    \label{fig:slm_gui_results3}
    \includegraphics[width=\textwidth]{4_implementación/Figs/slm_gui_results3}
    \caption*{\footnotesize Fuente: Elaboración propia.}
\end{figure}

\vfill
\clearpage
%  ╔═╗┌─┐┌┬┐┌─┐┌┐┌┌─┐┌─┐  ┌┬┐┌─┐  ╔╦╗┌─┐┬─┐┬┌─┌─┐┬  ┬
%  ║  ├─┤ ││├┤ │││├─┤└─┐   ││├┤   ║║║├─┤├┬┘├┴┐│ │└┐┌┘
%  ╚═╝┴ ┴─┴┘└─┘┘└┘┴ ┴└─┘  ─┴┘└─┘  ╩ ╩┴ ┴┴└─┴ ┴└─┘ └┘ 

\chapter{Caracterización de la variabilidad del valor de medición usando modelos de Markov}

En esta sección se explica el diseño e implementación del método propuesto para tener en cuenta la variabilidad del
nivel indicado en la pantalla del sonómetro bajo calibración en el resultado de medición, expresado como el valor de
medición junto con la incertidumbre expandida de medición.
En primer lugar, se describen las consideraciones previas que dan validez al método propuesto y se introduce el
algoritmo que resume la ejecución del método en una serie de pasos.
Luego, se presenta el fundamento teórico suficiente de los procesos estocásticos modelados con cadenas de Markov y se
describen brevemente los detalles de la implementación del método.
Finalmente, se muestra y discute un ejemplo de un valor de medición obtenido con el método implementado y su
correspondiente incertidumbre de medida tipo A\@.

\section*{Consideraciones previas}
\addcontentsline{toc}{section}{Consideraciones previas}
El nivel instantáneo ponderado en tiempo y en frecuencia definido en la ecuación~\eqref{eq:time_weighted_level} no es
un indicador representativo del nivel de sonidos cuya presión tiene bastante variabilidad, ya que es muy susceptible a
las variaciones instantáneas en la presión acústica.
Por lo que normalmente en mediciones acústicas se evalúa el nivel de sonido promediado en el tiempo o nivel de sonido
continuo equivalente, definido en la
\mbox{IEC 61672--1}~\citeyearpar{IEC_TC29_2013_1} como
%
\begin{equation}
    \label{eq:equivalent_level}
    L_{Xeq,T} = 10\log{\left(\frac{\frac{1}{t_1 - t_2}\,\int_{t_2}^{t_1} p_X^2\left(\xi\right)\,\mathrm{d}\xi}{p_0^2}\right)}.
\end{equation}
%
En que $t_1$ y $t_2$ son el instante de tiempo final e inicial de integración correspondientemente.
El nivel continuo equivalente es un indicador mucho más confiable, ya que se trata de un nivel constante que, durante
todo el tiempo de integración, tiene la misma energía acústica que la señal de presión con sus variaciones instantáneas.

Ahora, medir el nivel equivalente requiere una intervención manual del usuario para ajustar en el sonómetro el tiempo de
integración y ver el resultado final, por lo que este indicador no es el más adecuado para la automatización empleando
reconocimiento de imágenes.
No obstante, la \mbox{IEC 61672--3}~\citeyearpar{IEC_TC29_2013_3} permite seleccionar entre un nivel ponderado en el
tiempo o un nivel promediado en el tiempo, como valor de medición en cada punto de calibración.
Para no comprometer la automatización y obtener un resultado de medición válido y representativo de todas las posibles
variaciones que pueda tener el nivel instantáneo ponderado en el tiempo, se propone tener en cuenta todas las muestras
del nivel ponderado en el tiempo y luego, el nivel presente en la mayor cantidad de muestras viene a ser una estimación
apropiada del nivel promediado en el tiempo (esto en condiciones controladas, como es el caso en laboratorios de
calibración).
Este efecto se puede comprobar analizando las expresiones matemáticas de cada indicador.
En primer lugar, de la ecuación~\eqref{eq:time_weighted_level} se extrae que la presión acústica con ponderación
temporal, expresada como una función del tiempo es
%
\begin{equation}
    p_{XY}^2(t) = \frac{1}{\tau_{Y}}\,\int_{-\infty}^t p_X^2\left(\xi\right)\,
    e^{\nicefrac{-\left(t - \xi\right)}{\tau_Y}}\,\mathrm{d}\xi.
\end{equation}

Esta presión instantánea ponderada en el tiempo no es la misma de la ecuación~\eqref{eq:equivalent_level}.
Para calcular un nivel continuo equivalente con ponderación temporal y frecuencial se requiere modificar la
ecuación~\eqref{eq:equivalent_level} remplazando $p_X^2(\xi)$ por $p_{XY}^2(t)$; en efecto, queda
%
\begin{align}
    L_{XYeq,T} &= 10\log{\left(\frac{\frac{1}{t_1 - t_2}\,\int_{t_2}^{t_1} p_{XY}^2(t)\,\mathrm{d}t}{p_0^2}\right)}; \nonumber \\
    &= 10\log{\left(\frac{\frac{1}{\tau_Y\,\left(t_1 - t_2\right)}\,
    \int_{t_2}^{t_1} \int_{-\infty}^t p_X^2\left(\xi\right)\,
    e^{\nicefrac{-\left(t - \xi\right)}{\tau_Y}}\,\mathrm{d}\xi\,\mathrm{d}t}{p_0^2}\right)}. \label{eq:equivalent_time_weighted_level}
\end{align}

De la ecuación~\eqref{eq:equivalent_time_weighted_level} se puede concluir que cuanto más tiempo permanezca estable
una presión acústica instantánea ponderada en tiempo, el nivel equivalente más se acercará al correspondiente nivel
instantáneo ponderado en tiempo, pues tiene mayor efecto en el resultado de la integral.

\section*{Cadenas de Markov de tiempo discreto}
\addcontentsline{toc}{section}{Cadenas de Markov de tiempo discreto}

\subsection*{Matriz de probabilidades de transición}
\addcontentsline{toc}{subsection}{Matriz de probabilidades de transición}

Tomando como guía el libro de Dobrow~\citeyearpar{Dobrow2016} y el de Gubner~\citeyearpar{Gubner2006}, una cadena de
Markov es un proceso aleatorio con la propiedad particular de que, dados unos valores del proceso desde el tiempo cero
hasta el tiempo actual, la probabilidad condicional del valor del proceso en algún tiempo futuro depende solo del valor
actual, es decir, el futuro y el pasado son condicionalmente independientes dado el presente.
Analíticamente, una cadena de Markov es una secuencia de variables aleatorias $X_0, X_1, \dots$ que toman valores del
espacio discreto de estados $\mathcal{S}$ con la propiedad de que
%
\begin{equation}
    \label{eq:markov_chain}
    P\left(X_{n + 1} = j\,|\,X_0 = x_0, \dots, X_{n - 1}
    = x_{n - 1}, X_n = i\right) = P\left(X_{n + 1} = j\,|\,X_n = i\right),
\end{equation}
%
Para todo $x_0, \dots, x_{n - 1}, i, j \in \mathcal{S}$ y $n \ge 0$.
Normalmente, se asume que todas las cadenas de Markov son homogéneas, en las que la probabilidad no depende de $n$.

En la ecuación~\eqref{eq:markov_chain} las probabilidades solamente dependen de $i$ y $j$, por lo que estas se pueden
organizar de forma matricial en $\mathbf{P}$, cuya $ji$-ésima entrada es
$p_{ij} = P\left(X_{n + 1} = j\,|\,X_n = i\right)$, la probabilidad de transición de un estado a otro en un paso.
La matriz de transición es una matriz cuadrada de $k \times k$ para los $k$ estados en el espacio $\mathcal{S}$.

La matriz de transición o matriz estocástica debe cumplir que $p_{ij} \ge 0 \quad \forall\,i,j$ y que
%
\begin{equation*}
    \sum_j p_{ij} = \sum_j P\left(X_{n + 1} = j\,|\,X_n = i\right)
    = \sum_j \frac{P\left(X_{n + 1} = j, X_n = i\right)}{P\left(X_n = i\right)}
    = \frac{P\left(X_n = i\right)}{P\left(X_n = i\right)} = 1.
\end{equation*}
%
Lo que indica que las transiciones en las cadenas de Markov son exhaustivas.

Teniendo en cuenta este fundamento conceptual, para cada punto de calibración se conforma una matriz de transición en
la que cada nivel diferente indicado en pantalla es un estado en la cadena de Markov.

\subsection*{Distribución de probabilidad estacionaria}
\addcontentsline{toc}{subsection}{Distribución de probabilidad estacionaria}

Una vez se conforma una matriz con las probabilidades de transición se puede usar álgebra matricial para hacer cálculos
con las probabilidades.
Uno de los más elementales es el cálculo de las probabilidades de transición del estado $i$ al $j$ en $n \ge 1$ pasos,
es decir $p_{ij}^{(n)} = P\left(X_n = j \,|\,X_0 = i\right)$, la probabilidad de que en $n$ pasos el proceso visite el
estado $j$ dado que ahora está en el estado $i$.
Cuando $n = 1$ las probabilidades son las mismas de $\mathbf{P}$, pero cuando $n \ge 1$ las probabilidades de transición
son los $ij$-ésimos elementos de la potencia $n$ de $\mathbf{P}$, denotada como $\mathbf{P}^n$.

Es de especial interés conocer el comportamiento del sistema en el largo plazo, lo cual está caracterizado por las
potencias de mayor orden de $\mathbf{P}$.
En la medida en que $n$ incrementa, el proceso alcanza un comportamiento límite y las probabilidades de transición
convergen a una distribución de equilibrio, conocida como distribución límite, la cual no depende de la distribución de
probabilidad inicial de los estados.
Para una cadena de Markov la distribución límite es la distribución de probabilidad $\boldsymbol{\lambda}$ con la propiedad de
que para todo $i$ y $j$
%
\begin{equation}
    \lim_{n\to\infty} \mathbf{P}_{ij}^n = \lambda_{j}.
\end{equation}

Si una cadena de Markov tiene una distribución límite, entonces esta es única.
Se puede encontrar la distribución límite simplemente tomando las potencias de mayor orden hasta obtener una matriz
$\boldsymbol{\Lambda}$ en la cual todas las filas son iguales a $\boldsymbol{\lambda}$, o también se pueden encontrar soluciones
exactas de forma analítica y teórica.
Cabe mencionar que la distribución límite también puede ser interpretada como la proporción de tiempo que el proceso
visita cada estado en el largo plazo

Ahora, si se asigna la distribución límite como la distribución inicial de la cadena de Markov, luego se encuentran las
probabilidades marginales $\boldsymbol{\lambda}\,\mathbf{P}$, y el resultado es el mismo vector $\boldsymbol{\lambda}$, entonces
esta distribución límite es llamada distribución estacionaria.
Es decir, una distribución estacionaria es una distribución de probabilidad $\boldsymbol{\pi}$ que satisface
$\boldsymbol{\pi} = \boldsymbol{\pi}\,\mathbf{P}$, o lo que es igual
%
\begin{equation}
    \label{eq:stationary_distribution}
    \pi_j = \sum_i \pi_j\, \mathbf{P}_{ij}, \quad \forall\, j.
\end{equation}

Aunque la distribución límite y la estacionaria están relacionadas, una cadena de Markov puede tener más de una
distribución estacionaria o ninguna, y esta no necesariamente es la distribución límite.
Pero si la cadena tiene una distribución límite, entonces esa distribución es estacionaria.
Esto depende directamente de la topología de la cadena;
si la cadena es regular entonces su matriz de transición $\mathbf{P}$ es regular (todos los valores de alguna de sus
potencias son positivos), y la distribución límite es igual a la estacionaria.

Para encontrar la distribución estacionaria cuando la matriz es regular, la forma más directa es resolver el sistema
lineal de ecuaciones que resulta de combinar la ecuación~\eqref{eq:stationary_distribution} y la restricción
$\sum_i \boldsymbol{\pi}_i = 1$.
La ecuación~\eqref{eq:stationary_distribution} puede escribirse matricialmente como
$\boldsymbol{\pi}\,(\mathbf{P} - \mathbf{I}) = \mathbf{0}$, o, para operar con vectores columna, como
$\left(\mathbf{P}^\intercal - \mathbf{I}\right)\,\boldsymbol{\pi} = \mathbf{0}$.
Finalmente, se puede escribir el sistema de ecuaciones matricialmente como
%
\begin{align}
    \mathbf{A}\,\boldsymbol{\pi} &= \mathbf{b}; \\
    \left[\begin{matrix}
              \left(\mathbf{P}^\intercal - \mathbf{I}\right) \\ 1 \cdots 1
    \end{matrix}\right]\,\boldsymbol{\pi} &= \left[\begin{matrix}
                                                       0 \\ 0 \\ \vdots \\ 1
    \end{matrix}\right].
\end{align}

Para encontrar soluciones para $\boldsymbol{\pi}$ se resuelve la ecuación
\begin{equation}
    \label{eq:pi_linear_system}
    \left(\mathbf{A}^\intercal\,\mathbf{A}\right)\,\boldsymbol{\pi} = \mathbf{A}^\intercal\,\mathbf{b}.
\end{equation}

\subsection*{Valor esperado}
\addcontentsline{toc}{subsection}{Valor esperado}

En un sistema discreto, el valor esperado calculado a partir de la distribución estacionaria es
%
\begin{equation}
    \mathrm{E}[X] = \sum_n x_n\,\pi_n,
\end{equation}

\section*{Algoritmo para la creación del modelo de Markov}
\addcontentsline{toc}{section}{Algoritmo para la creación del modelo de Markov}
El método propuesto consiste en tomar cada uno de los niveles instantáneos obtenidos como un estado en una cadena de
Markov.
Esta cadena caracteriza los cambios de un nivel a otro, considerando la variabilidad del nivel instantáneo como un
proceso estocástico.
Luego, con la distribución de probabilidad estacionaria se estima el valor esperado en el largo plazo, que lógicamente
corresponde al estado (nivel instantáneo) con mayor probabilidad de ocurrir.
Las probabilidades de transición de un estado a otro se determinan a partir de una serie de muestras de niveles
instantáneos obtenidas durante un tiempo de $25$ segundos aproximadamente, y la cantidad de muestras depende del periodo
de actualización del nivel instantáneo en la pantalla del sonómetro.
Por tanto, el valor esperado calculado con la cadena de Markov será probablemente el nivel instantáneo que más muestras
tuvo y será el valor de medición.
Este es un proceso que se realiza en cada punto de calibración y se resume en el siguiente algoritmo.

\begin{algorithm}[H]
    \caption{Algoritmo para el cálculo del valor esperado usando cadenas de Markov.}
    \label{alg:markov_expected_value}
    \scriptsize
    \DontPrintSemicolon
    \SetKwData{frames}{frames}
    \SetKwInOut{Output}{output}
    \KwData{\frames $\leftarrow$ \text{Vídeo de $\qty{25}{\s}$ del nivel instantáneo indicado en pantalla.}}
    \Output{Valor esperado}
    \BlankLine
    \textbf{\hyperref[sec:downsampling]{Paso 1}:} Submuestrear y reconocer cuadros después del tiempo de estabilización.\;
    \textbf{\hyperref[sec:transition_matrix]{Paso 2}:} Conformar matriz de transición de estados.\;
    \textbf{\hyperref[sec:stationary_distribution]{Paso 3}:} Calcular distribución de probabilidad estacionaria.\;
    \textbf{\hyperref[sec:expected_value]{Paso 4}:} Calcular valor esperado.\;
\end{algorithm}

Las operaciones efectuadas en cada paso, se describen y se ilustran mediante un ejemplo tomado de una ejecución del
algoritmo en la prueba de ponderación frecuencial $Z$ con señales eléctricas a la frecuencia de $\qty{63}{\Hz}$.

\section*{Paso 1: Submuestreo y reconocimiento de cuadros después de la estabilización}
\addcontentsline{toc}{section}{Paso 1: Submuestreo y reconocimiento de cuadros después de la estabilización}
\label{sec:downsampling}

Como el nivel con ponderación temporal \emph{Fast}, que tiene una constante de tiempo $\tau_F = \qty{125}{\ms}$,
requiere un tiempo transitorio hasta que el nivel se estabilice después de enviar la señal eléctrica.
Se determinó experimentalmente un tiempo de $\qty{2}{\s}$.
Los cuadros obtenidos durante este tiempo de estabilización no se tienen en cuenta para la matriz de transición,
pero sí para determinar el cuadro cero con el que se sincroniza la actualización del nivel instantáneo en pantalla con
los cuadros del vídeo.

Para determinar el cuadro cero primero se efectúa el algoritmo~\ref{alg:image_recongnition} en los cuadros del tiempo
de estabilización.
Con el vector de los valores numéricos reconocidos se encuentra el índice del primer cambio detectado de valor.
Luego, a los cuadros después de ese índice se les hace un submuestreo según la relación entre la tasa de cuadros por
segundo de la cámara y la tasa de actualización de la pantalla del sonómetro.
Los cuadros que quedan corresponden al instante exacto en el que hay una nueva muestra en pantalla.
Sin embargo, dado que puede presentarse en la pantalla LCD un efecto de solapamiento entre la muestra anterior y la
nueva, que afecta negativamente el reconocimiento de imágenes, se determinó experimentalmente tomar $2$ cuadros después
del cuadro en el que ocurre el cambio, para permitir la estabilización de la pantalla.

\tikzmath{\x1 = -0.2; \x2 = 20;}
\begin{figure}[!h]
    \caption{Serie adquirida de cuadros de estabilización (naranja), de medición (negro) y submuestreados con
        corrimiento de dos posiciones (rojo)}
    \label{fig:frames_series}
    \centering
    \begin{tikzpicture}[inner sep=1pt, minimum height=1.8cm]
        \node[process, draw=orange, align=center] (frame0) at (0,0) {\includegraphics[height=\x2pt]{5_MC_incertidumbre/Frames/frame(0)}\\[\x1em] idx: $0$\\[\x1em] \texttt{\small nan}};
        \node[process, draw=orange, align=center, right=0cm of frame0] (frame1) {\includegraphics[height=\x2pt]{5_MC_incertidumbre/Frames/frame(1)}\\[\x1em] idx: $1$\\[\x1em] \texttt{\small nan}};
        \node[process, text=orange, minimum height=0.5cm, above=0cm of frame1] (tstab){$t_{\mathrm{stab}} = \qty{2}{s} \quad \longrightarrow$};
        \node[process, draw=orange, align=center, right=0cm of frame1] (frame2) {\includegraphics[height=20pt]{5_MC_incertidumbre/Frames/frame(2)}\\[\x1em] idx: $2$\\[\x1em] \texttt{\small nan}};
        \node[process, right=0cm of frame2] (dots1) {$\cdots$};
        \node[process, draw=orange, align=center, right=0cm of dots1] (frame38) {\includegraphics[height=\x2pt]{5_MC_incertidumbre/Frames/frame(38)}\\[\x1em] idx: $38$\\[\x1em] \texttt{\small nan}};
        \node[process, draw=blue, thick, align=center, right=0cm of frame38] (frame39) {\includegraphics[height=\x2pt]{5_MC_incertidumbre/Frames/frame(39)}\\[\x1em] idx: $39$\\[\x1em] $32.6$};
        \node[process, text=blue, above=0cm of frame39, minimum height=0.5cm] (zeroframe) {cuadro cero};
        \node[process, draw=orange, align=center, right=0cm of frame39] (frame40) {\includegraphics[height=\x2pt]{5_MC_incertidumbre/Frames/frame(40)}\\[\x1em] idx: $40$\\[\x1em] $32.6$};
        \node[process, draw=red, thick, align=center, right=0cm of frame40] (frame41) {\includegraphics[height=\x2pt]{5_MC_incertidumbre/Frames/frame(41)}\\[\x1em] idx: $41$\\[\x1em] $32.6$};
        \node[process, draw=orange, align=center, below=0cm of frame0] (frame42) {\includegraphics[height=\x2pt]{5_MC_incertidumbre/Frames/frame(42)}\\[\x1em] idx: $42$\\[\x1em] $32.6$};
        \node[process, draw=orange, align=center, right=0cm of frame42] (frame43) {\includegraphics[height=\x2pt]{5_MC_incertidumbre/Frames/frame(43)}\\[\x1em] idx: $43$\\[\x1em] $32.6$};
        \node[process, draw=orange, align=center, right=0cm of frame43] (frame44) {\includegraphics[height=\x2pt]{5_MC_incertidumbre/Frames/frame(44)}\\[\x1em] idx: $44$\\[\x1em] $32.6$};
        \node[process, draw=orange, align=center, right=0cm of frame44] (frame45) {\includegraphics[height=\x2pt]{5_MC_incertidumbre/Frames/frame(45)}\\[\x1em] idx: $45$\\[\x1em] $32.6$};
        \node[process, draw=red, thick, align=center, right=0cm of frame45] (frame46) {\includegraphics[height=\x2pt]{5_MC_incertidumbre/Frames/frame(46)}\\[\x1em] idx: $46$\\[\x1em] $32.6$};
        \node[process, draw=orange, align=center, right=0cm of frame46] (frame47) {\includegraphics[height=\x2pt]{5_MC_incertidumbre/Frames/frame(47)}\\[\x1em] idx: $47$\\[\x1em] $32.6$};
        \node[process, draw=orange, align=center, right=0cm of frame47] (frame48) {\includegraphics[height=\x2pt]{5_MC_incertidumbre/Frames/frame(48)}\\[\x1em] idx: $48$\\[\x1em] $32.6$};
        \node[process, draw=orange, align=center, below=0cm of frame42] (frame49) {\includegraphics[height=\x2pt]{5_MC_incertidumbre/Frames/frame(49)}\\[\x1em] idx: $49$\\[\x1em] $33.2$};
        \node[process, draw=orange, align=center, right=0cm of frame49] (frame50) {\includegraphics[height=\x2pt]{5_MC_incertidumbre/Frames/frame(50)}\\[\x1em] idx: $50$\\[\x1em] $33.2$};
        \node[process, draw=red, thick, align=center, right=0cm of frame50] (frame51) {\includegraphics[height=\x2pt]{5_MC_incertidumbre/Frames/frame(51)}\\[\x1em] idx: $51$\\[\x1em] $33.2$};
        \node[process, right=0cm of frame51] (dots2) {$\cdots$};
        \node[process, draw=orange, align=center, right=0cm of dots2] (frame97) {\includegraphics[height=\x2pt]{5_MC_incertidumbre/Frames/frame(97)}\\[\x1em] idx: $97$\\[\x1em] $33.2$};
        \node[process, draw=orange, align=center, right=0cm of frame97] (frame98) {\includegraphics[height=\x2pt]{5_MC_incertidumbre/Frames/frame(98)}\\[\x1em] idx: $98$\\[\x1em] $33.3$};
        \node[process, draw=orange, align=center, right=0cm of frame98] (frame99) {\includegraphics[height=\x2pt]{5_MC_incertidumbre/Frames/frame(99)}\\[\x1em] idx: $99$\\[\x1em] $33.3$};
        \node[process, draw=orange, align=center, right=0cm of frame99] (frame100) {\includegraphics[height=\x2pt]{5_MC_incertidumbre/Frames/frame(100)}\\[\x1em] idx: $100$\\[\x1em] $33.3$};
        \node[process, below=0cm of frame50, minimum height=0.5cm] (tmeas) {$t_{\mathrm{meas}} = \qty{25}{s} \quad\longrightarrow$};
        \node[process, draw=red, thick, align=center, below=0.5cm of frame49] (frame101) {\includegraphics[height=\x2pt]{5_MC_incertidumbre/Frames/frame(101)}\\[\x1em] idx: $101$\\[\x1em] $33.3$};
        \node[process, draw, align=center, right=0cm of frame101] (frame102) {\includegraphics[height=\x2pt]{5_MC_incertidumbre/Frames/frame(102)}\\[\x1em] idx: $102$\\[\x1em] $-$};
        \node[process, draw, align=center, right=0cm of frame102] (frame103) {\includegraphics[height=\x2pt]{5_MC_incertidumbre/Frames/frame(103)}\\[\x1em] idx: $103$\\[\x1em] $-$};
        \node[process, draw, align=center, right=0cm of frame103] (frame104) {\includegraphics[height=\x2pt]{5_MC_incertidumbre/Frames/frame(104)}\\[\x1em] idx: $104$\\[\x1em] $-$};
        \node[process, draw, align=center, right=0cm of frame104] (frame105) {\includegraphics[height=\x2pt]{5_MC_incertidumbre/Frames/frame(105)}\\[\x1em] idx: $105$\\[\x1em] $-$};
        \node[process, draw=red, thick, align=center, right=0cm of frame105] (frame106) {\includegraphics[height=\x2pt]{5_MC_incertidumbre/Frames/frame(106)}\\[\x1em] idx: $106$\\[\x1em] $33.3$};
        \node[process, draw, align=center, right=0cm of frame106] (frame107) {\includegraphics[height=\x2pt]{5_MC_incertidumbre/Frames/frame(107)}\\[\x1em] idx: $107$\\[\x1em] $-$};
        \node[process, right=0cm of frame107] (dots3) {$\cdots$};
        \node[process, draw=red, thick, align=center, below=0cm of frame101] (frame111) {\includegraphics[height=\x2pt]{5_MC_incertidumbre/Frames/frame(111)}\\[\x1em] idx: $111$\\[\x1em] $33.2$};
        \node[process, draw, align=center, right=0cm of frame111] (frame112) {\includegraphics[height=\x2pt]{5_MC_incertidumbre/Frames/frame(112)}\\[\x1em] idx: $112$\\[\x1em] $-$};
        \node[process, right=0cm of frame112] (dots4) {$\cdots$};
        \node[process, draw=red, thick, align=center, right=0cm of dots4] (frame116) {\includegraphics[height=\x2pt]{5_MC_incertidumbre/Frames/frame(116)}\\[\x1em] idx: $116$\\[\x1em] $33.2$};
        \node[process, right=0cm of frame116] (dots5) {$\cdots$};
        \node[process, draw=red, thick, align=center, right=0cm of dots5] (frame121) {\includegraphics[height=\x2pt]{5_MC_incertidumbre/Frames/frame(121)}\\[\x1em] idx: $121$\\[\x1em] $33.1$};
        \node[process, right=0cm of frame121] (dots6) {$\cdots$};
        \node[process, right=0cm of dots6] (dots7) {$\cdots$};
        \node[process, draw=red, thick, align=center, right=0cm of dots7] (frame161) {\includegraphics[height=\x2pt]{5_MC_incertidumbre/Frames/frame(161)}\\[\x1em] idx: $161$\\[\x1em] $33.2$};
        \node[process, right=0cm of frame161] (dots8) {$\cdots$};
        \node[process, right=0cm of dots8] (dots9) {$\cdots$};
    \end{tikzpicture}
    \caption*{\footnotesize Fuente: Elaboración propia.}
\end{figure}

En seguida, en el nuevo vector de cuadros submuestreado se busca el índice del cuadro inmediatamente después al tiempo
de estabilización, que corresponde al producto del tiempo de estabilización y la tasa de actualización de la pantalla
del sonómetro.
Finalmente, se efectúa el reconocimiento de los cuadros submuestreados a partir del índice del cuadro de estabilización
con el algoritmo~\ref{alg:image_recongnition}.
El vector de valores numéricos reconocidos es el arreglo de muestras a partir del cual se conforma la matriz de
transición de estados en el \hyperref[sec:transition_matrix]{paso 2}.

\begin{figure}[!h]
    \caption{Cuadro capturado con solapamiento en el dígito decimal entre el número $2$ y el $3$.}
    \label{fig:wrong_frame}
    \centering
    \includegraphics[height=2cm]{5_MC_incertidumbre/Frames/wrong_frame}
    \caption*{\footnotesize Fuente: Elaboración propia.}
\end{figure}
%
Si en este punto aún resultara que el clasificador entregue un valor no numérico debido a un solapamiento como el de
la figura~\ref{fig:wrong_frame}, el valor no numérico es tomado como atípico y se elimina de la serie de muestras.

La figura~\ref{fig:frames_series} ilustra el proceso del paso 1 y en la siguiente figura se grafica el arreglo de
valores numéricos reconocidos.
%
\begin{figure}[!h]
    \caption{Serie de muestras de valores numéricos reconocidos.}
    \label{fig:recognized_samples}
    \includegraphics[width=\textwidth]{5_MC_incertidumbre/recognizedSamples}
    \caption*{\footnotesize Fuente: Elaboración propia.}
\end{figure}
\vfill

\section*{Paso 2: Conformar matriz de transición de estados}
\addcontentsline{toc}{section}{Paso 2: Conformar matriz de transición de estados}
\label{sec:transition_matrix}

\begin{code}
    \caption{Método estático para la construcción de una matriz de transición de estados a partir de una serie de muestras dada.}
    \label{code:build_transition_matrix}
    \centering
    \begin{minted}{python}
@staticmethod
def build_transition_matrix(samples: np.ndarray) -> np.ndarray:
    """
    Method for construction of the states transition matrix from a sequence of samples
    :param samples: A numpy one dimensional ndarray with the sequence of samples.
    :return: A numpy array that represents the transition matrix of the Markov model.
    """
    states = np.unique(samples)  # Remove repeated samples
    P = pd.DataFrame(data=np.zeros((states.shape[0], states.shape[0])), index=states, columns=states)  # Empty P
    for i in range(1, samples.shape[0]):  # Counts transitions from state to state
        P.loc[samples[i - 1], samples[i]] += 1
    P = P.div(P.sum(axis=1), axis=0)  # Computes probabilities
    return P
    \end{minted}
\end{code}
\begin{figure}[!h]
    \caption{Grafo de estados y probabilidades de transición que representa la matriz de transición~\eqref{eq:eg_transition_matrix}.}
    \label{fig:eg_automata}
    \centering
    \begin{tikzpicture}[thick,shorten >=1pt,node distance=6cm,on grid,auto]
        \node[state](q1){$\qty{33.1}{\dB}$};
        \node[state](q2)[below=of q1]{$\qty{33.2}{\dB}$};
        \node[state](q3)[right=of q1]{$\qty{33.3}{\dB}$};
        \node[state](q4)[below=of q3]{$\qty{33.4}{\dB}$};
        \draw[->, stealth-]
        (q1) edge [loop left] node {$\num{0.500}$} ()
        edge [bend right, swap] node {$\num{0.015}$} (q2)
        edge [swap] node {$\num{0.012}$} (q3)
        (q2) edge [loop left] node {$\num{0.712}$} ()
        edge [swap, pos=0.6] node {$\num{0.333}$} (q1)
        edge [bend right] node {$\num{0.395}$} (q3)
        edge [bend right, swap] node {$\num{0.286}$} (q4)
        (q3) edge [loop right] node {$\num{0.581}$} ()
        edge [bend right, swap] node {$\num{0.167}$} (q1)
        edge [bend right] node {$\num{0.250}$} (q2)
        edge [swap, pos=0.6] node {$\num{0.143}$} (q4)
        (q4) edge [loop right] node {$\num{0.571}$} ()
        edge [swap] node {$\num{0.023}$} (q2)
        edge [bend right, swap] node {$\num{0.012}$} (q3);
    \end{tikzpicture}
    \caption*{\footnotesize Fuente propia.}
\end{figure}

Para conformar la matriz de transición se escribió el método estático del código~\ref{code:build_transition_matrix}.
Una vez implementado correctamente en el desarrollo previo de la aplicación, se obtuvieron los resultados esperados.
Para el ejemplo ilustrado en el \hyperref[sec:downsampling]{paso 1}, la matriz de transición obtenida se presenta a
continuación.
Y el grafo correspondiente se representa en la figura~\ref{fig:eg_automata}.

\begin{equation}
    \label{eq:eg_transition_matrix}
    \mathbf{P} = \kbordermatrix{
        & \mathbf{\num{33,1}} & \mathbf{\num{33.2}} & \mathbf{\num{33.3}} & \mathbf{\num{33.4}} \\
        \mathbf{\num{33.1}} & \num{0.500} & \num{0.333} & \num{0.167} & \num{0.000} \\
        \mathbf{\num{33.2}} & \num{0.015} & \num{0.712} & \num{0.250} & \num{0.023} \\
        \mathbf{\num{33.3}} & \num{0.012} & \num{0.395} & \num{0.581} & \num{0.012} \\
        \mathbf{\num{33.4}} & \num{0.000} & \num{0.286} & \num{0.143} & \num{0.571} \\
    }.
\end{equation}

\section*{Paso 3: Calcular distribución de probabilidad estacionaria}
\addcontentsline{toc}{section}{Paso 3: Calcular distribución de probabilidad estacionaria}
\label{sec:stationary_distribution}

\begin{code}
    \caption{Método estático para el cálculo de la distribución límite de una cadena de Markov dada una matriz de transición.}
    \label{code:limit_dist}
    \centering
    \begin{minted}{python}
def limit_dist(P: np.ndarray) -> float:
    """
    Method to calculate the limit distribution with linear algebra solution using a given transition matrix P if
    the given matrix is a regular matrix, else, the calculation is performed with the high matrix powers of the
    transition matrix.
    :param P: The numpy array that represents de transition matrix.
    :return: The stationary distribution as a float number.
    """
    for n in range(2, 1001):
        if np.all(np.linalg.matrix_power(P, n) > 0):  # Check if it is a regular transition matrix
            # The matrix is regular, so limiting distribution exists and is the unique stationary distribution
            A = np.append(np.transpose(P) - np.identity(P.shape[0]), np.ones((1, P.shape[0])),
                          axis=0)  # Augmented A
            b = np.zeros((A.shape[0], 1))
            b[-1] = 1  # Augmented b
            PI = np.linalg.solve(np.transpose(A).dot(A), np.transpose(A).dot(b))  # Stationary distribution
            break
    else:
        PI = np.linalg.matrix_power(P, 1000)

    return PI
    \end{minted}
\end{code}

Para calcular la distribución límite dada una matriz de transición $\mathbf{P}$ se escribió
el método estático del código~\ref{code:limit_dist}.
Siguiendo con el ejemplo, al resolver el sistema lineal con el código implementado, la distribución estacionaria es
\begin{equation*}
    \boldsymbol{\pi} = \kbordermatrix{
        & \mathbf{\num{33.1}} &  \mathbf{\num{33.2}} &  \mathbf{\num{33.3}} & \mathbf{\num{33.4}} \\
        & \num{0.026} & \num{0.570} & \num{0.364} & \num{0.040}
    }.
\end{equation*}

\section*{Paso 4: Calcular valor esperado}
\addcontentsline{toc}{section}{Paso 4: Calcular valor esperado}
\label{sec:expected_value}

El valor esperado se obtiene directamente con
\mintinline{python}{expected_value = np.round(np.sum(np.array(P.index) * PI.T), 1)}.
Para el ejemplo, el valor esperado, redondeado a la misma cantidad de dígitos decimales de la precisión del sonómetro,
es $\qty{33.2}{\dB}$.

\section*{Evaluación tipo A de la incertidumbre típica}
\addcontentsline{toc}{section}{Evaluación tipo A de la incertidumbre típica}

De a cuerdo con la guía para la expresión de la incertidumbre de medida~\citep{ISO_TAG4_2008}, los valores de las
observaciones individuales $q_k$ difieren en razón de las variaciones aleatorias de las magnitudes de influencia o de
efectos aleatorios.
La varianza experimental de dichas $n$ observaciones está dada por:
%
\begin{equation}
    s^2\left( q_k \right) = \frac{1}{n - 1}\, \sum_{j = 1}^{n} \left( q_j - \bar{q} \right)^2,
\end{equation}
%
Que, junto con su raíz cuadrada positiva $s\left( q_k \right)$ (denominada desviación típica experimental), representan
la variabilidad de los valores $q_k$, es decir, su dispersion al rededor del valor esperado $\bar{q}$.
Luego, la mejor estimación de la varianza experimental de la media $\sigma^2\left( \bar{q} \right) = \sigma^2 / n$ es
%
\begin{equation}
    \label{eq:experimental_mean_variance}
    s^2\left( \bar{q} \right) = \frac{s^2\left( q_k \right)}{n},
\end{equation}
Que, junto con desviación típica experimental de la media $s\left( \bar{q} \right) = \sqrt{s^2\left( \bar{q} \right)}$
pueden ser utilizadas como medida de la incertidumbre de $\bar{q}$.

En concreto, en un modelo matemático de un mesurando $y$, para una magnitud de entrada $X_i$ obtenida a partir de $n$
observaciones repetidas e independientes $X_{i,k}$, la incertidumbre típica $u\left( x_i \right)$ de su estimación
$x_i = \bar{X}_i$ es $u\left( x_i \right) = s\left( \bar{X}_i \right)$, con $s^2\left( \bar{X}_i \right)$ calculada con
la ecuación~\eqref{eq:experimental_mean_variance}.
Esta incertidumbre típica $u\left( x_i \right)$ es llamada \emph{incertidumbre típica tipo A}.

El número de observaciones $n$ debe ser lo suficientemente grande para garantizar que $s^2\left( \bar{q} \right)$
proporcione una estimación fiable de la varianza $\sigma^2\left( \bar{q} \right)$.
La aplicación desarrollada permite cumplir esta consideración, puesto que en un tiempo de $\qty{25}{\s}$, con un periodo
de actualización de pantalla típico de $\qty{100}{\ms}$, se capturan $250$ muestras.
Ahora, en un cálculo posterior, cuando se determinan intervalos de confianza para la incertidumbre expandida, se debe
tomar en cuenta la diferencia entre $\sigma^2\left( \bar{q} \right)$ y $s\left( \bar{q} \right)$, ya que en muchos casos
puede ser que la distribución de probabilidad del mesurando (en este caso equivale a la distribución estacionaria) sea
muy distinta de una distribución normal.

La estimación de la incertidumbre típica tipo A puede hacerse rápidamente con la instrucción
\mintinline{python}{samples.std() / math.sqrt(samples.shape[0])}.
Para las muestras del ejemplo anterior se obtiene $s\left( \bar{q} \right) \approx \qty{0.004}{\dB}$.
%  ╔═╗┌─┐┌┐┌┌─┐┬  ┬ ┬┌─┐┬┌─┐┌┐┌┌─┐┌─┐
%  ║  │ │││││  │  │ │└─┐││ ││││├┤ └─┐
%  ╚═╝└─┘┘└┘└─┘┴─┘└─┘└─┘┴└─┘┘└┘└─┘└─┘

\chapter{Conclusiones y recomendaciones}
\section{Conclusiones}
En este trabajo se desarrollaron dos aplicaciones de \emph{software} diseñadas, una para realizar la calibración de calibradores
acústicos, y otra para sonómetros, de conformidad con los lineamientos del Anexo B de la \mbox{IEC 60942}~\citeyearpar{IEC_TC29_2017}
y de la \mbox{IEC 61672--3}~\citeyearpar{IEC_TC29_2013_3} respectivamente.
Estas aplicaciones integran el control de los instrumentos de medición (patrones y auxiliares) que conforman el sistema de
calibración junto con las secuencias predeterminadas.
Particularmente, para sonómetros, también se integra el procesamiento de imágenes para el reconocimiento de los valores
indicados en la pantalla del sonómetro y el modelamiento de la variabilidad de un resultado usando cadenas de Markov.

Dados los resultados obtenidos, las aplicaciones desarrolladas son una utilidad que mejora significativamente los procesos
de calibración de los laboratorios, ya que reducen el tiempo total de calibración en aproximadamente un $30\%$,
bajan la incertidumbre de medición (al incrementar los grados efectivos de libertad por el mayor número de muestras que
pueden obtenerse respecto a un proceso manual) y aumentan la robustez del procedimiento (asegurando los resultados contra
el error humano).

Para calibradores acústicos, la arquitectura de \emph{software} basada en el modelo de GRAFCET es una metodología
que permitió desarrollar una aplicación versátil e intuitiva, ofreciendo al usuario un mecanismo sencillo para identificar
y ordenar los pasos del proceso de calibración en etapas enumeradas.
A esto se suma que, tomar el GRAFCET como un tipo de paradigma de programación, facilita la implementación en Python
empleando multi-hilos.

El control de los instrumentos del estante de calibración mediante comandos SCPI es una de las herramientas fundamentales
en la aplicación desarrollada.
Utilizar estos comandos en las secuencias programadas hace innecesario que el usuario manipule directamente los controles
físicos de los instrumentos.
Esto contribuye al aseguramiento contra el error humano y aumenta la precisión de los parámetros de señal, ya que se pueden
ajustar y obtener todos los dígitos disponibles en el instrumento, lo que no es posible con la interfaz física.

El método de reconocimiento de imágenes implementado para la calibración de sonómetros siguió la estructura clásica de
adquisición (captura de fotogramas), preproceso (filtrado, \emph{padding} y escalamiento), segmentación
(binarización Otsu), extracción de características (descriptor SIFT simplificado), y reconocimiento (clasificador
bayesiano normal ingenuo).
Este método diseñado es lo bastante sencillo, pero suficiente para el reconocimiento en tiempo real con una base de datos
de entrenamiento relativamente pequeña y con el que se consiguen resultados confiables a un bajo costo computacional.

Finalmente, el modelamiento de la variabilidad de un resultado empleando cadenas de Markov, considerándola como un proceso
estocástico, permite estimar en el largo plazo cuál es el valor de medición tomando en cuenta la \emph{dinámica} del
cambio de un valor a otro (representada en las probabilidades de transición) y no solo como la media de todas las
muestras adquiridas.

\section{Recomendaciones y trabajo futuro}
Las aplicaciones fueron programadas con un manejo de excepciones básicas, sin embargo, conviene garantizar que estas
sean ejecutadas en las condiciones normales.
Como que las señales eléctricas enviadas o medidas se encuentren dentro del rango de los instrumentos utilizados,
lo cual se puede predecir con la sensibilidad del micrófono patrón y del micrófono del sonómetro, y con los límites
del rango lineal de medición del sonómetro.
Al usar la aplicación para la calibración de sonómetros, se recomienda adecuar un ambiente controlado para que la
iluminación, distancia focal, o ruido no afecte el reconocimiento de imágenes.

Tal como están desarrolladas las aplicaciones hasta ahora, cuentan con el funcionamiento básico necesario para llevar a
cabo las calibraciones de principio a fin.
Sin embargo, aún resta desarrollar utilidades que mejorarían las aplicaciones y las harían más adecuadas para el uso
cotidiano en los laboratorios.
Utilidades como guardar y cargar sesiones de calibración, retroceder, avanzar o navegar entre etapas, emitir certificado
de calibración y bloquear resultados.
El desarrollo de estas utilidades puede hacerse sobre lo que ya está, puesto que en el código se dejaron las bases para
esas funciones.
Por ejemplo, en la interfaz gráfica están los botones correspondientes, y la configuración básica de estos se ejecuta al
lanzar la aplicación.

Por ahora, las señales eléctricas son ajustadas y medidas sin tener en cuenta las correcciones proporcionadas en los
certificados de calibración de los patrones, por lo que los resultados aún no son trazables en el sentido estricto.
El trabajo posterior es entonces, aplicar las correcciones usando técnicas de interpolación o métodos numéricos a partir
de los resultados de calibración de los patrones.

Igualmente, la estimación de incertidumbre de medición en este trabajo llegó hasta la incertidumbre tipo A, pero para
reportar un resultado de medición completo, el trabajo debe complementarse estimando las demás componentes de incertidumbre
tipo B y calculando la incertidumbre expandida de medición.
Además, en este trabajo la incertidumbre tipo A se estimó a partir de la desviación experimental de la media, no del
valor esperado del proceso estocástico.
El trabajo futuro es tener en cuenta la distribución de probabilidad estacionaria del proceso estocástico en la estimación
de la incertidumbre tipo A y posteriormente en la determinación de los intervalos de confianza de la incertidumbre
expandida de medición, de tal modo que se relacione el modelamiento en cadenas de Markov con la incertidumbre de medición
asociada al valor de medición.
%  ╔═╗┌┐┌┌─┐─┐ ┬┌─┐┌─┐
%  ╠═╣│││├┤ ┌┴┬┘│ │└─┐
%  ╩ ╩┘└┘└─┘┴ └─└─┘└─┘

\begin{appendix}

\chapter{Anexo: Códigos de Python}

\begin{code}
	\caption{Código para presentar resultados del procesamiento de una muestra del número $5$ del conjunto de imágenes entrenamiento.}
	\label{code:test_samples_code}
	\centering
	\inputminted{python}{3_Reconocimiento/Codes/image_processing_test.py}
\end{code}

\begin{code}
	\caption{Código para realizar la extracción de características de los vectores de entrenamiento y para entrenar el clasificador.}
	\label{code:features_extracting}
	\centering
	\inputminted{python}{3_Reconocimiento/Codes/features_extracting.py}
\end{code}

\begin{code}
\caption{Código para realizar la extracción de características de los vectores de prueba y para realizar la clasificación.}
\label{code:sample_classifying}
\centering
\begin{minted}{python}
X = np.array(X)  # Convert test sample features vector to numpy array
# -------- DIMENSIONALITY REDUCTION BY KERNEL PCA --------- #
X = kpcaModel.transform(X)
# ------- CLASSIFY ---------#
y_hat = gnbClassifier.predict(X)

\end{minted}
\end{code}

\begin{code}
\caption{Código para generar la matriz de confusión del clasificador bayesiano normal.}
\label{code:confusion_matrix}
\centering
\begin{minted}{python}
from sklearn.metrics import confusion_matrix, ConfusionMatrixDisplay

# -------------- CLASSIFICATION ACCURACY --------------
cm = confusion_matrix(y_hat, y_test, labels=gnbClassifier.classes_)
disp = ConfusionMatrixDisplay(confusion_matrix=cm, display_labels=gnbClassifier.classes_)
disp.plot()
plt.show()
\end{minted}
\end{code}

\end{appendix}

\bibliographystyle{apalike-es}
\bibliography{Bibliografía}
\addcontentsline{toc}{chapter}{\numberline{}Bibliografía}
\end{document}